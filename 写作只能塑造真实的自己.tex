\documentclass[12pt,a5paper]{ctexbook}
\usepackage{ctex}
\setCJKmainfont{FZSSK.TTF}[BoldFont={FZXBSK.TTF},ItalicFont={FZKTK.TTF}]
\setCJKsansfont{FZHTK.TTF}
\setCJKmonofont{FZFSK.TTF}
\pagestyle{headings}
%导入版面设置的宏包
\usepackage{geometry}
\usepackage[hidelinks,colorlinks=true,linkcolor=black]{hyperref}
\usepackage{graphicx} %插入图片的宏包

%导入页眉页脚需要的宏包
\usepackage{fancyhdr}

%设置边距
%\geometry{top=2cm,bottom=3cm}

%设置页脚
%\lfoot{荣昌区国资经营中心}
\cfoot{\emph{第\thepage 页}}
%\rfoot{右页脚}

%设置目录深度
\setcounter{tocdepth}{1}

\setlength{\skip\footins}{1cm}	%脚注与正文之间的距离
%\setlength{\footnotesep}{0.7cm} %脚注之间的距离

\title{\Huge{\textbf{写作只能塑造真实的自己}}}
\author{\small{\emph{由「筷子小手」公众号 \space{} 整理}}}
\date{}


\begin{document}
\maketitle %%标题页
%\centerline{\emph{二〇二三年九月}} %%为了放在页面底部
%\centerline{\emph{中国 · 重庆}}

\thispagestyle{empty} %%封面页不要编页码
\newpage

\thispagestyle{empty}
\tableofcontents %%目录页
\newpage


\pagestyle{plain}
\setcounter{page}{1} %重新编页码

\chapter{为什么要写作}

\section{王小波:我为什么要写作}


\emph{本文选自王小波\footnote{王小波(1952年5月13日—1997年4月11日),男,中国当代学者、作家。代表作品有《黄金时代》《白银时代》《青铜时代》《黑铁时代》等。王小波生于北京,先后当过知青、民办教师、工人。1978年考入中国人民大学,1980年王小波与李银河结婚,同年发表处女作《地久天长》。1984年赴美匹兹堡大学东亚研究中心求学,留学期间游历了美国各地,并利用1986年暑假游历了西欧诸国。1988年回国,先后在北京大学,中国人民大学任教。1992年9月辞去教职,做自由撰稿人。1997年4月11日病逝于北京,年仅45岁。}的《我的精神家园》,主要收录了王小波的影评和他对社会环境与个人尊严的反思。}

\vspace{2em}

有人问一位登山家为什么要去登山—谁都知道登山这件事既危险,又没什么实际的好处,他回答道:“因为那座山峰在那里。”我喜欢这个答案,因为里面包含着幽默感—明明是自己想要登山,偏说是山在那里使他心里痒痒。除此之外,我还喜欢这位登山家干的事,没来由地往悬崖上爬。它会导致肌肉疼痛,还要冒摔出脑子的危险,所以一般人尽量避免爬山。用热力学的角度来看,这是个反熵\footnote{反熵这是热力学中的一个术语,这个词的意思简而言之就是说投入的多但释放出来的能量少,也就是“费力不讨好”的现象在物理学中表现的一种形式,王小波用反熵这个词来形容自己的写作行为。}的现象,所发趋害避利肯定反熵。

现在把登山和写作相提并论,势必要招致反对。这是因为最近十年来中国有过小说热、诗歌热、文化热,无论哪一种热都会导致大量的人投身写作,别人常把我看成此类人士中的一个,并且告诫我说,现在都是什么年月了,你还写小说(言下之意是眼下是经商热,我该下海去经商了)?但是我的情形不一样。前三种热发生时,我正在美国念书,丝毫没有受到感染。我们家的家训是不准孩子学文科,一律去学理工。因为这些缘故,立志写作在我身上是个不折不扣的反熵过程。我到现在也弄不明白自己为什么要干这件事,除了它是个反熵过程这一点。

有关我立志写作是个反熵过程,还有进一步解释的必要。写作是个笼统的字眼,还要看写什么东西。写畅销小说、爱情小诗等等热门东西,应该列入熵增过程之列。我写的东西一点不热门,不但挣不了钱,有时还要倒贴一些。严肃作家的“严肃”二字,就该做如此理解。据我所知,这世界上有名的严肃作家,大多是凑合也算不上。这样说明了以后,大家都能明白我确实在一个反熵过程中。

我父亲不让我们学文科。当然,他老人家也是屋内饮酒,门外劝水的人,自己也是个文科的教授,但是他坦白地承认自己择术不正,不足为训。我们兄弟姐妹五个就范此全学了理工科,只我哥哥例外。考虑到我父母脾气暴躁、吼声如雷,你得说这种选择是个熵增过程。而我哥哥那个例外是这么发生的:七八年考大学时,我哥哥是北京木城漳煤矿最强壮的青年矿工,吼起来比我爸爸音量还要大。无论是动手揍他,还是朝他吼叫,我爸爸自己都挺不好意思,所以就任凭他去学了哲学:在逻辑学界的泰斗沈有鼎先生的门下当了研究生。考虑到符号逻辑是个极专门的学科(这是从外行人看不懂得逻辑文章来说),它和理工科差不太多的。从以上的叙述,你可以弄明白我父亲的意思。他希望我们每个人都学一种外行人弄不懂而又是有功世道的专业,平平安安地度过一生。我父亲一生坎坷,他又最爱我们,这样的安排在他看来最自然不过。

我自己的情形是这样的:从小到大,身体不算强壮,吼起来音量也不够大,所以一直本分为人。尽管如此,我身上总有一股要写小说的危险情绪。插队的时候,我遇上一个很坏的家伙,我就编了一个故事,描写他从尾骨开始一寸寸变成了一条驴,并且把它写出来,以泄心头之愤。后来读了一些书,发现卡夫卡也写了个类似的故事,搞得我很不好意思。还有一个故事,女主人公长了蝙蝠的翅膀,并且头发是绿色的,生活在水下。这些二十岁前的作品我都烧掉了。在此一提是要说明这种危险倾向的由来。后来我一直抑制着这种倾向,念完了本科,到美国去留学。我哥哥也念完了硕士,也到美国去留学。我在那边又开始写小说,这种危险的倾向再也不能抑制了。

在美国时,我父亲去世了。回想他让我们读理科的事,觉得和美国发生的事不是一个逻辑。这让我想起了前苏联元帅图哈切夫斯基\footnote{米哈伊尔·尼古拉耶维奇·图哈切夫斯基,1893年2月16日—1937年6月11日,苏联最早的五名元帅之一,军事战略学家,有“红色拿破仑”之称号。}对大音乐家萧斯塔科奇\footnote{肖斯塔科维奇,1906年9月25日—1975年8月9日,全名迪米特里·迪米特里耶维奇·肖斯塔科维奇,出生于俄罗斯圣彼得堡,是苏联时期最重要的作曲家之一、20世纪世界著名作曲家之一,多次担任苏联作曲家协会的领导工作,被世界许多著名音乐学府都曾授予他荣誉称号。}说的话来:“我小的时候,很有音乐天才。只可惜我父亲没钱给我买把小提琴!假如有了那把小提琴,我现在就坐在你的乐池里。”这段话乍看不明其意,需要我提示一句:这次对话发生在苏联的三十年代,说完了没多久,图元帅就一命呜呼。那年头专毙元帅将军,不大毙小提琴手。我父亲在世时,一心一意地要给我们每人都弄把小提琴。这把小提琴就是理工农医任一门,只有文科不在其内,这和美国发生的事不一样,但是结论还是同一个——我该去干点别的,不该写小说。

有关美国的一切,可以用一句话来描述:“American's business is business.”这句话的意思就是说,那个国家永远是在经商热中,而且永远是一千度的白热。所以你要是看了前文之后以为那里有某种气氛会有助于人立志写作就错了。连我哥哥到了那里都后悔了,觉得不该学逻辑,应当学商科或者计算机。虽然他依旧未证出的逻辑定理,但是看到有钱人豪华的住房,也免不了唠叨几句他对妻儿的责任。

在美国有很强大的力是促使人去挣钱,比方说洋房,有些只有一片小草坪,有的有几百亩草坪,有的有几千亩草坪,所以仅就住房一项,就能产生无穷无尽的挣钱的动力。再比方说汽车,有无穷的档次和价格。你要是真有钱,可以考虑把肯尼迪遇刺时坐的汽车买来坐。还有人买下了前苏联的战斗机,驾着飞上天。在那个社会里,没有人受得了自己的孩子对同伴说:我爸爸穷。我要是有孩子,现在也准在那里挣钱。而写书在那里也不是个挣钱的行当,不信你到美国书店里看看,各种各样的书涨了架子,和超级市场里陈列的卫生纸一样多——假如有人出售苦心积虑一页页写出的卫生纸,肯定不是好行当。除此之外,还有好多人的书没有上架,窝在他自己的家里。我没有孩子,也不准备要。作为中国人,我是个极少见的现象。但是人有一张脸,树有一张皮,别人都有钱挣,自己却在干可疑的勾当,脸面上也过不去。

在美国时,有一次和一位华人教授聊天,他说他女儿很有出息,放着哈佛大学人类学系奖学金不要,自费去念一般的大学的law school,如此反潮流,真不愧是书香门第。其实这是舍小利而趋大利,受小害而避大害。不信你去问问律师挣多少钱,人类学家又挣多少钱。和我聊天的这位教授是个大学问家,特立独行之辈。一谈到了儿女,好像也不大特立独行了。

说完了美国、苏联,就该谈谈自己。到现在为止,我写了八年小说,也出了几本书,但是大家没怎么看到。除此之外,我还常收到谩骂性的退稿信,这时我总善意地想:写信的人准是领导那里挨了骂,找我撒气。

提起王小波,大家准会想到宋朝的四川拉杆子的那一位,想不起我身上。我还在反熵过程中。顺便说一句,人类的存在,文明的发展就是个反熵过程,但是这是说人类。具体说到自己,我的行为依旧无法解释。

再顺便说一句,处于反熵过程中,绝不只是我一个人。在美国,我遇上过支起摊来卖托洛斯基、格瓦拉、毛主席等人的书的家伙,我要和他说话,他先问我怕不怕联邦调查局——别的例子还很多。

在这些人身上,你就看不到水往低处流、苹果掉下地,狼把兔子吃掉的宏大的过程,看到的现象,相当于水往山上流,苹果飞上天,兔子吃掉狼。我还可以说,光有熵增现象不成。举例言之,大家都顺着一个自然的方向往下溜,最后准会在个低洼的地方汇齐,挤在一起像粪缸里的蛆。但是这也不能解释我的行为。我的行为是不能解释的,假如你把熵增现象看成金科玉律的话。

当然,如果硬要我用一句话直截了当地回答这个问题,那就是:\textbf{我相信我自己有文学才能,我应该做这件事。}但是这句话正如一个嫌疑犯说自己没杀人一样不可信。所以信不信由你罢。

\newpage

\section{乔治·奥威尔:我为什么写作}

\emph{《我为何写作》是2009年4月五南图书出版股份有限公司出版的图书,此为其中第一章内容,作者乔治·奥威尔\footnote{乔治·奥威尔(George Orwell,1903年6月25日—1950年1月21日),英国著名小说家、记者和社会评论家。他的代表作《动物庄园》和《1984》是反极权主义的经典名著,其中《1984》是20世纪影响最大的英语小说之一。},由黄伟翻译。}
\vspace{2em}

快16岁时,我突然发现了纯粹属于字词的快乐,就是说,字词的声音和组合。《失乐园》中的诗句:

\centerline{\emph{所以他艰难而吃力地}}
\centerline{\emph{向前:他艰难而吃力}}

现在看来并不是特别地精彩,那时却让我脊骨战栗:而把“he”(他)拼成“hee”也别有乐趣。至于描绘事物的需要,我已经都知道了。所以,我要写什么样的书就清楚了,可以说这就是我那时要写的书。我要写大部头的自然主义的小说,它们有着不幸的结局,充满细节描写和引人注目的比喻,到处都是辞藻华美的章节:这里词语的选用某种程度上为的是它们的声音。事实上我完成的第一部小说《缅甸岁月》(写它时我30岁,构思要早得多)就是这样的书。

我提供这些背景情况,是因为我觉得对一个作者的发展初期缺少了解,就无法评估他的写作动机。写作的主题由作者生活的时代所决定——至少在像我们这样大动荡、大变革的时代,这一点是不错的——但在开始写作之前,他在情感上已经获得某种态度,对此他永远也不能彻底摆脱。毫无疑问,他需要修炼性格,以免停滞在某些不成熟的阶段或陷于某种不正常的情绪:但是,如果他把早先接受的影响摆脱得干干净净,那么他的写作冲动也就被扼杀了。撇开谋生的需求不谈,我认为写作动机主要有四种,至少写作散文是这样。这些动机程度不同地存在于个个作者的心里,而就某一作者来说,各种动机所占比例将根据他生活环境的改变而时时发生变化。这些动机有:

\begin{itemize}
	\item 纯粹利己主义。想显得聪明,被人谈论,死后让人回忆,在儿时冷落过你,现已长大的你想在那些人面前出一口气,等等,等等。硬说这不是写作的一个动机,以及不是一个强有力的动机,那是骗人的鬼话。在这一点上,作家与科学家、艺术家、政治家、律师、军事家、成功的商人——一句话,与所有出人头地的人物没有什么两样。人类的绝大多数并不是自私透顶。大约30岁过后,他们放弃个人的抱负——许多情况的确表明,他们几乎放弃了作为个体存在的意识——而主要为别人活着,不然的话就得被单调乏味的事情憋死。但是,也有少数有才华而又固执的人,决心终身过自己的生活,作家属于这一类。应当说,严肃作家总体上要比记者更自负、更自我中心,尽管对钱的兴趣低一些。
	\item 审美热情。对于外部世界的美或在另一方面对于词语及其正确的组合具有的感觉。由一种发音作用于另一种发音产生的效果、一篇好散文的坚实力量或是一个好故事的叙述节奏带来的快乐。想把个人觉得有价值的不应错过的经验与人共享的希望。尽管在许多写作者那里,审美动机非常薄弱,然而,即使小册子的编写人或教科书作者也有自己心爱的词语,他喜欢它们不是出于实用目的;或许他会对排字的式样、纸面边缘空白的宽度等有着强烈的感觉。超出列车时刻表水平以上的任何一本书都不可能一点没有审美上的考虑。
	\item 历史冲动。希望看到事情的本来面目,发掘出真实的事件,并将它们储存起来留给子孙后代。
	\item 政治目的——采用“政治”一词的尽可能宽泛的含义。想把世界推向某一特定的方向,改变人们对于应努力争取的社会类型的观念。同样,任何一本书都不可能真正地摆脱政治倾向。那种认为艺术与政治毫不相干的观点本身就是一种政治态度。
\end{itemize}

看得出,这些不同的冲动彼此间是多么地不相容,以及它们面对不同的人、在不同的时间将出现怎样的波动。就本性而言——把你刚刚进入成年时具有的状态视作你的“本性”——我是一个更看重前三种动机而不是第四种动机的人。在和平年代,我也许会写那些华丽的或纯描写性的书,也许不会太多地意识到我的政治信仰。但事实上,我已被迫变成了那种写作小册子的人。起初我把5年的时间花在一项不合适的职业上(缅甸的印度皇家警察)。然后又经受了贫穷和失败。这增强了我对权力的本能憎恨,使我第一次完全意识到劳动阶级的存在,而且,在缅甸的工作也使得我对帝国主义的本质有了一些了解;但这些经历还不足以让我获得一个准确的政治方向。后来出现了希特勒、西班牙内战,等等。到了1935年底,我仍然不能做出坚定的抉择。

1936至1937年间,西班牙战争和其他一些事件扭转了局面,此后我知道了我的立场。自1936年后,我的严肃作品中每一行写下的文字,直接或间接地都是反对独裁主义和拥护我心目中的民主社会主义的。处在我们这样的时期,认为一个作者可以避开这些主题不写,在我看来是荒唐的。每个人都以这样或那样的方式写到它们。这只是一个选择什么立场以及采取何种方法的问题。一个人对他的政治倾向越自觉,也就越有可能不致因为行为的政治性而牺牲他真诚的审美与理性追求。

过去的10年里,我一直最想做的是使政治性写作成为艺术。我的出发点常常是一种党派感情,一种对不公正现象的不平之感。当我坐下来写一本书时,我并不对自己说,“我要写出一部艺术作品”。我之所以要写,是因为我想揭穿某些谎言,我要引起人们对某些事实的注意,我的初衷是让人们倾听。然而,如果写作不同时也是一种审美经验的话,我是不会去写书甚至给杂志写长篇文章的。谁要是有心检查一下我写的东西,就会看出,即便一篇彻头彻尾的宣传,也包含着大量在职业政治家看来不大相干的内容。我不能,也不想,彻底放弃童年获得的对世界的看法。只要能健康地活着,我就会对散文文体怀有强烈感觉,就会喜欢这地球表面上的东西,各种实在之物和那些零七八落的没用的消息就会给我带来快乐。企图压制我自己的那一面是无济于事的。要做的就是把自己根深蒂固的好恶与时代强加给我们的本质上属于公众的而非个人的种种行为协调起来。

这并不容易,在结构和语言方面都出现一些麻烦,而且真实性方面也面临新的问题。让我只举一例,这是个所遇到的较为原始的问题。我的那本关于西班牙内战的书《向卡塔洛尼亚致敬》,毫无疑问,是一部坦率的政治性著作,但我写作时却在总体上保持着某种超脱和对形式的关注。我的确做出极大努力说真话而又不触犯我的文学本能。但在其他内容中间保留了长长一章,这是为被指控与弗朗哥阴谋勾结的托洛茨基\footnote{托洛茨基主义是托洛茨基提出的马克思主义理论,在马克思主义里,托洛茨基主义是倾左的,托洛茨基领导的左翼反对派在1920年代的影响力日增,并于1928年被斯大林强力镇压。在被镇压后,该派仍在苏联国内秘密活动。托洛茨基最终流亡海外(土耳其、挪威、墨西哥)。}分子的辩护,里面充斥着报纸摘引之类的文字。显然,这样的章节,一两年后就不再吸引一般读者,会毁掉这本书。一位我尊敬的批评家为此开导了我一番。“你为什么把那些东西放进来?”他说,“你把本来不错的一本书弄成了新闻报道。”他说的是真话,但我只能这样做。我恰好得知英国人民不大可能被告知的事实:那些无辜的人们遭到错误的指控。如果我对此不感到气愤也就不会去写这本书了。

这个问题以这样或那样的形式一再出现。语言的问题要微妙一些,探讨它需要太长的时间。我只想说,近年来试图写作时少些形象性,多些准确性。我发现在任何情况下,一旦你精通了一种写作方式,你总要超越它。《兽园》是我第一部试图将政治目的同艺术目的熔炼成一个整体的小说——我完全清楚我在做什么。我已经有7年没写小说了,但我希望不久再写一部。这注定是一个失败,每本书都是一个失败,但我的确比较清楚地知道我要写什么样的书。

回头看看写出的这一两页,我明白这些话看上去好像我的全部写作动机只是一种关心公众利益的精神。我不想给人留下这样的最后印象。所有作家都自负、自私和懒惰,而在他们各种动机的最深处存在着一种神秘。写作一本书就是一场可怕的消耗战,好像经历了一次长期不愈的痛苦疾病。

要不是由于那不可抵挡、无法理解的魔鬼的驱使,谁也不会再去干这种事情。我们只知道这魔鬼就是让婴儿啼哭以引起注意的同一类本能。然而这话也不假:一个人除非长期不懈地致力于消除自己的个性,否则就写不出任何可读的东西。好的散文就像窗上的玻璃。我不能肯定地说我哪一种动机最强,但我知道应该听从哪一种。纵观我的写作经历,我看到不论在哪儿,只要缺少政治目的,我写的书就没有生气,我就会误入歧途,总是写出那些华而不实的章节、没有意义的句子、装饰性的形容词和空话。

\newpage
\section{余华:写作是去完成一个过去的愿望}

\emph{本篇摘自余华杂文集《我只知道人是什么》,原标题《永远不要被自己更愿意相信的东西影响》。}
\vspace{2em}

最好的阅读是怀着空白之心去阅读,赤条条来去无牵挂的那种阅读,什么都不要带上,这样的阅读会让自己变得越来越宽广,如果以先入为主的方式去阅读,就是挑食似的阅读,会让自己变得狭窄起来。

为什么不少当时争议很大的文学作品后来能成为经典,一代代流传下去?这是因为离开了它所处时代的是是非非,到了后来的读者和批评家那里,重要的是作品表达了什么,至于作者是个什么样的人不重要了。

这就是为什么我们在阅读古典文学作品或者过去时代文学作品的时候——比如鲁迅的作品时——我们可以怀着一颗空白之心去阅读,而阅读当代作品的时候很难怀有这样的空白之心。你有你的经验,你会觉得这部作品写得不符合你的生活经验,中国很大,经济发展不平衡,每个地方的风俗和文化也有差异,每个人的成长环境不一样以后,年龄不一样以后,经验也会不一样,这会导致带着过多的自己的经验去阅读一部作品,对这部作品的判断可能会走向另外一个方向。反过来带着空白之心去阅读,就会获得很多。阅读最终为了什么?最终是为了丰富自己,变化自己,而不是为了让自己原地踏步,始终如此,没有变化。

无论是读者、做研究的,还是做评论的,首先要做的是去读一部作品,而不是去研究一部作品。我上中学的时候,读的都是中心思想、段落大意之类的,用这种方式的话肯定是把一部作品毁掉了。阅读首先是感受到了什么,无论这种感受是喜欢还是不喜欢,欣赏还是不欣赏。读完以后有感受了,这种感受带来的是欣赏还是愤怒,都是重要的。然后再去研究为什么让我欣赏,为什么让我愤怒,为什么让我讨厌?研究应该是第二步的,应该是在阅读之后的。

说到写作时的画面感,我在写小说的时候肯定是有的,虽然我不会画画,我对绘画也没有像对音乐那么的喜爱。

还有一个原因,相对小说叙述而言,音乐叙述更近一点,两者都是流动的叙述,或者说是向前推进的叙述。而绘画也好雕塑也好,绘画是给你一个平面,雕塑是让你转一圈,所以我还是更喜欢音乐。但是小说也好,音乐也好,都是有画面感的。

我1992年底和张艺谋合作做《活着》电影的时候——这片子是一九九三年拍的——他那时候读了我的一个中篇小说《一九八六年》,他说我的小说里面全是电影画面,当时我并没有觉得我作品里面有那么多的画面,但是一个导演这么说,我就相信了。

至于九十年代写作的变化,《活着》和《许三观卖血记》为什么在今天如此受欢迎?昨天张清华还高谈阔论分析了一堆理由,听完我就忘了,没记住,昨天状态不好。其实我也不知道,我的感觉是这样,我当时写《活着》,有些人把《在细雨中呼喊》视为我写作风格的转变之作。是,它是已经转变了,因为它是长篇小说了。但是真正的转变还是从《活着》开始的,什么原因?就是换成了一个农民来讲述自己的故事,只能用一种最朴素的语言。

有位出版社的编辑告诉我,她的孩子,十三岁的时候读了《许三观卖血记》,喜欢;读了《活着》还是喜欢;读到《在细雨中呼喊》就读不懂了。她问我什么原因,我想《活着》和《许三观卖血记》受欢迎,尤其是《活着》,可能有这么个原因,故事是福贵自己来讲述的,只能用最为简单的汉语。我当时用成语都是小心翼翼,一部小说写下来没有一个成语浑身难受,总得用它几个,就用了家喻户晓的,所有人都会用的成语。可能就让大家都看得懂了,人人都看得懂了,从孩子到大人。

我昨天告诉张清华,这两本书为什么在今天这么受欢迎,尤其是《活着》,我觉得唯一的理由就是运气好,确实是运气好。我把话题扯开去,《兄弟》出版那年我去义乌,发现那里有很多“李光头”。当地的人告诉我,义乌的经济奇迹起来以后,上海、北京的经济学家、社会学家们去调查义乌奇迹,义乌人告诉他们三个字“胆子大”,就是胆子大,创造了义乌的奇迹。所以《活着》为什么现在受欢迎,也是三个字“运气好”,没有别的可以解释。

《第七天》在一个地方比《活着》受欢迎,就是翻译成维吾尔文以后,在维族地区很受欢迎,已经印了六次,《活着》只印了三次。在中文世界里,我其他的书不可能超过《活着》,以后也不可能,我这辈子再怎么写,把自己往死里写,也写不出像《活着》这么受读者欢迎的书了,老实坦白,我已经没有信心了。《活着》拥有了一代又一代的读者,当当网有大数据,前些日子他们告诉我,在当当网上购买《活着》的人里面有六成多是九五后。

我为什么写《第七天》,这是有延续性的。《活着》和《许三观卖血记》之后,长达十年之后出版的长篇小说是《兄弟》。《兄弟》出版的时候,我在后记里写得很清楚了,中国人四十年就经历了西方人四百年的动荡万变,这四十年对我来说是很重要的写作,而且我以后再也不会写这么大的作品了,用法语的说法叫“大河小说”或者“全景式的”,他们的评论里几乎都有大河小说和全景式的,法语世界的读者对这部小说极其喜爱。

我三十一岁写完《在细雨中呼喊》,三十二岁写完《活着》,三十五岁写完《许三观卖血记》,四十六岁写完《兄弟》。

我们这一代作家的经历比较特殊,我们同时代外国作家的朋友圈不会像我们这么杂乱。我二十岁出头刚开始写作,在浙江参加笔会时,认识了浙江的作家,当时跟我关系最好的两个作家,早就不写作了,都去经商了。我在成长和写作过程中,不断认识一些人,这些人一会儿干这个一会儿干那个,他们又会带来不同的朋友圈,有些人从政,有些人从商,有几个进了监狱,还在监狱给我打电话,我们在二十多岁时因为文学和艺术走到一起,后来分开了,各走各的路,这样的经历让我到了四五十岁时写作的欲望变化了,说白了就是想留下一个文学文本之外还想留下一个社会文本。

《兄弟》写完以后,我觉得不够,想再写一个,想用更加直接的方式写一个,于是写了一部非虚构的书,在中国台湾出版。写完这本非虚构的书之后,我还是觉得不够,中国这三十年来发生的奇奇怪怪的事情太多了,我有个愿望是把它集中写出来。用什么方式呢?如果用《兄弟》的方式篇幅比《兄弟》还要长。然后呢,有一天突然灵感光临了,一个人死了以后接到火葬场的电话,说他火化迟到了。我知道可以写这本书了,写一个死者的世界,死者们聚到一起的时候,也把自己在生的世界里的遭遇带到了一起,这样就可以用不长的篇幅把很多的故事集中写出来。

我虚构了一个候烧大厅,死者进去后要拿一个号,坐在那里等待自己的号被叫到,然后起身去火化。穷人挤在塑料椅子里,富人坐在宽敞的沙发区域,这个是我在银行办事的经验,进银行办事都要取一个号,拿普通号坐在塑料椅子里,拿VIP号的进入另一个区域,坐在沙发里,那里有茶有咖啡有饮料。我还虚构了一个进口炉子一个国产炉子,进口炉子是烧VIP死者的,国产炉子是烧普通死者的。昨天晚上收到别人给我发来的一个东西,关于八宝山的,八宝山有两个公墓,一个是革命公墓,一个是人民公墓,革命公墓里葬的都是干部,人民公墓里葬的都是群众。那里还真有进口炉子,还是从日本进口的,烧起来没有烟,全是高级干部在里面烧的。我写进口炉子时是瞎编的,我不知道有进口的,我没考察过,没想到真有。八宝山里面也是有等级制的,夫妻不是同一个级别的不能葬在一起,而是葬在不同的墓区。

“死无葬身之地”在我写“第一天”的时候就出现了,当时我知道这部小说可以写完了。我现在比较担心的——事实也正是如此——就是“死无葬身之地”翻译成其他语言之后不是这样了,已经不是我们中文里的“死无葬身之地”了。

把社会事件集中起来写,需要一个角度,这个角度在《第七天》里就是“死无葬身之地”,从一个死者的世界来对应一个活着的世界。假如没有死无葬身之地的话,这个小说很难写完,一方面是不知道写到最后是怎么回事,有了“死无葬身之地”之后也就有了小说的结尾;另一方面是很多故事可以集中到一起来写,死者们来到死无葬身之地的时候,也把各自生前的遭遇带到了一起。

这本书写了不少现实里发生过的奇奇怪怪的事情,但是写作的时候,运用它们的时候,不是那么容易的。我举个例子,杨飞是去殡仪馆以后才意识到自己没有墓地,那他烧了之后怎么办,没地方放,所以他出来了。路上遇上了鼠妹,然后去了死无葬身之地。还有几个人也在游荡,也去了死无葬身之地。所有的人都没有去过医院的太平间,只有李月珍和二十七个死去的婴儿,他们是从太平间去的死无葬身之地。我还写了李月珍和那些婴儿的失踪之谜,当地政府说他们已经火化了,紧急把别人的骨灰分出来一部分变成他们的骨灰,诸如此类的荒诞事。所以我不能让他们在太平间里自己坐起来自己走去,这样写很不负责任。那时候我想到那么多年来经常发生的一个事件——地陷,很符合这里的描写。所以我就让太平间陷下去,把他们震出来,有震动以后,李月珍带着这些婴儿在某种召唤下顺理成章地去了死无葬身之地。写这样一部小说的时候,事情不是简单的罗列,什么地方怎么处理是非常重要的。写完《第七天》以后,我觉得够了,接下来我不想再写这些了,我应该换换口味了。

《第七天》肯定有遗憾的地方,包括《兄弟》《许三观卖血记》《活着》和《在细雨中呼喊》都有遗憾的地方,每一部作品我都有遗憾的地方。至于写错了和用错了什么,就有人认为是硬伤,这个我认为不是那么回事。当年我写《活着》的时候,《活着》才十一万字,里面有个次要人物的名字写错了,前面叫这个后面叫那个了,后来是我的一个译者发现的,他怎么读都觉得这两个人是一个人,就写信问我,我读了一下原文,发现确实是一个人,然后改过来了。《许三观卖血记》要感谢《收获》的肖元敏,她真是一个好编辑,她在编辑的过程中给我打电话,那时候已经有电话了,她说从叙述上看,《许三观卖血记》写的应该是南方的小镇。我说是南方的小镇。她说你为什么不写“小巷”,写了“胡同”。我在北京住了很多年了,平时出门都是说什么胡同,我在写作的时候都不知不觉写成了胡同,肖元敏替我把“胡同”改回“小巷”。如果肖元敏不改回来,肯定又有人说是硬伤了。但是这种问题,并不能用来否定一部作品。因为作家是人,是人都会犯个错误什么的。《兄弟》有五十多万字,有时候写着写着就会犯错,张清华就找到了一个毛病,小说里面李光头说林红是他的梦中情人。张清华很温和地问我,“文革”的时候会说这样的话吗?我说当然不会说,忘了嘛,写着写着就忘记了。张清华问我为什么再版的时候不把它改一下呢?我说没有必要,假如五十年之后这本书还有人读的话,根本没人知道“文革”时候的人不会说这样的话的,今天在座的同学肯定也不知道那时候不会说这样的话,如果五十年之后没有人读了,我改了也白改。

写作有时候就是去完成一个过去的愿望。我年轻的时候读了川端康成\footnote{川端康成(かわばた やすなり,1899年6月14日—1972年4月16日),日本文学界“泰斗级”人物,新感觉派作家,著名小说家。1968年以《雪国》《古都》《千只鹤》三部代表作获得诺贝尔文学奖,亚洲第二位获诺贝尔文学奖的人。一生多旅行,心情苦闷忧郁,逐渐形成了感伤与孤独的性格,这种内心的痛苦与悲哀成为后来川端康成的文学底色。1972年4月16日,川端康成突然采取口含煤气管的自杀方式离开了人世,未留下纸质遗书。他早在1962年就说过:“自杀而无遗书,是最好不过的了。无言的死,就是无限的活。”}的中篇小说《温泉旅馆》,这是我读到的第一部里面没有主角的小说,里面的人物可以说都是配角。看上去《温泉旅馆》是一部传统小说,它的叙述很规矩,其实不是。传统小说有个套路,简单地说就是有主角和配角,但《温泉旅馆》不是,里面人物很多,每个人物的笔墨却都不多,有的人物好像只有一页纸就消失了,比如里面写到一个人,是专门糊窗户纸的,他糊窗户纸时跟那些侍女打情骂俏,有个女孩还爱上他了,他扬长而去的时候对那个女孩说,如果你想我了,就把窗户纸全捅破。《温泉旅馆》对我很有吸引力,我想以后有机会时也应该写一部没有主角的小说,大概五六年以后,我写作《世事如烟》的时候,已经写了几页纸了,小说的主角还没有在我脑子里出现,我突然想到当初读完《温泉旅馆》时留给自己的愿望,知道机会来了,于是我写下了一部没有主角的小说。略有遗憾的是《世事如烟》是一部中篇小说,其实我的野心更大,我想写一部没有主角的长篇小说,这个机会后来出现过,可是我没有把握住,就是去年出版的《第七天》,等我意识到这部长篇小说可以写成没有主角的小说时已经来不及了,因为我选择了第一人称,已经写到“第三天”了,“我”和父亲的故事已经是主线了,再变换人称或者角度的话叙述的感觉就会失去。现在看来,杨飞和他的父亲的故事还是写得多了点,我应该写得少一些,增加其他人物的笔墨,这样的话这部作品对我来说会更有意思。当然,对读者来说,他们可能更喜欢阅读像福贵和许三观这样的故事,自始至终的人物命运的故事,读者能够很快进入。但是对作家不一样,他有自己的写作理想,他想在某部作品中完成某个理想,而这样的理想往往是他二十多岁甚至十多岁时阅读经典作家作品时出现的。

我心想以后吧,以后肯定还会有机会。很多读者熟悉我的长篇小说,但是对我过去的中短篇小说不太了解,他们读完《第七天》后以为我是第一次写生死交界的小说,或者说是有关亡灵的小说,我的日文译者饭塚容\footnote{饭塚容,日本著名汉学家,现任日本中央大学文学系教授。他早在大学时代便开启了他的中国文学翻译生涯,已经翻译了余华、铁凝、苏童等40多名中国作家的80多部作品,为中国文化的在日传播做出了巨大贡献。}告诉我,他在翻译《第七天》的时候总是想到我过去的《世事如烟》。确实如此,《第七天》可以说是《世事如烟》的某种延续。

如何面对批评?这是作家不能回避的一个问题。我从《兄弟》到《第七天》,被人铺天盖地地批评了两轮,批评对我已经连雨点都不是了,没有什么作用了。但是有时候我对批评会有反思,为什么有那么多人来批评?尤其从《兄弟》开始,只要我出版一本新书,就会有猛烈的批评光临。刚开始可以把它理解为有某种动机,后来我觉得不应该这样,虽然批评我的文章中百分之九十都是胡扯,但是反过来想一想,赞扬我的文章里胡扯的比例不比这个低。同样都是胡扯,为什么赞扬你就觉得不错,批评你就不能接受?

优秀的文学评论给作家的感受是什么样的?应该是这样的:如果我站在这个山头,那么他就会在对面的那个山头;如果我在这个河边,那他就应该在对面的河边。作家读到以后,和他的想法完全不一样,但是又引发了某种一致性。这种一致性我可以用两部电影的画面来向你们解释,一部电影是安哲罗普洛斯的《永恒的一天》,里面有个人要离开了,他在收拾屋子准备离开的时候,正在放他的音乐。当这个音乐响起来,他家对面窗户里的某个人也放起同样的音乐。这个人每次放这个音乐,对面也响起这个音乐,对面那个人是谁他不知道,他们俩都放一样的音乐。还有一个是我儿子告诉我的,日本的一个动画片,有一个男孩,可能是经历过像你们一样曾经备受摧残的中学生活,考试考试考试,这个话题可能不适合在大学说,你们现在已经很成熟了,说一说也没关系。男孩不想活了,走上了自己教室所在的楼顶,准备往下跳的时候发现对面楼顶也有一个学生想往下跳,两个学生互相看了一会儿,最后决定不跳了。我觉得好的作家看到好的评论,好的评论家看到好的作品的感受就是这样。
那天的讨论会上,张清华以赞扬的口吻说了一句我在北师大的入校仪式上说过的话“我永远不会放弃对真理的追求”。虽然很矫情,但是他很感动。

我说这句话是有前因后果的,当时我和儿子一起——他高中毕业准备去美国上大学——在家里看了张艺谋的《金陵十三钗》,看完之后我们一起讨论,最后的结尾让妓女替女学生赴死让我们反感,难道妓女的生命就比女学生低贱?当时我儿子说了一番话让我很吃惊,孩子的成长让父母无法预料。他说的是罗素接受英国BBC的采访,记者最后请他对一千年以后的人说几句话,有关他的一生以及一生的感悟。罗素说了两点,一是关于智慧,二是关于道德。关于一,罗素说不管你是在研究什么事物,还是在思考任何观点,只问你自己,事实是什么,以及这些事实所证实的真理是什么。永远不要让自己被自己所更愿意相信的,或者认为人们相信了会对社会更加有益的东西所影响。只是单单地去审视,什么才是事实。

当时我儿子基本上把罗素的话复述出来了,我的理解就是永远不要放弃对真理的追求。当然我儿子的复述比我说得好多了,我这个说得很直白,我的是福贵说的,他的是罗素说的。接着我儿子说张艺谋已经把自己的想法当成真理了,然后说我也到了这个时候,要小心了。确实,当一个人成功以后,很容易把自己的想法当成真理。那么真理是什么呢?我今天不是对在座的老师说,是对你们学生说,真理是什么,真理不是自己的想法,也不是你们老师的想法,真理不是名人名言,也不是某种思想,它就是单纯的存在,它在某一个地方,你们要去寻找它,它才会出现,你们不去寻找,它就不会出现。或者说有点像灯塔那样,像飞机航道下面的地面雷达控制站,它并不是让你们产生一种什么思想之类的,它能做的就是把你们引向一个正确的方向,当你们去往这个正确的方向时,可以避免触礁或者空中险情。真理就是这样一种单纯的存在,你们要去寻找它,它才会有,然后它会引领你们。

\newpage

\section{史铁生:写作只能塑造真实的自己}

\emph{长篇哲思散文《病隙碎笔》,是史铁生最为经典的作品之一,是一部充满了深刻生命体验的人生笔记。以下是《病隙碎笔》中关于写作的摘抄。}
\vspace{2em}

1.我其实未必合适当作家,只不过命运把我弄到这一条路上来了。左右苍茫时,总也得有条路走,这路又不能再用腿去趟,便用笔去找。而这样的找,后来发现利于此一铁生,利于世间一颗最为躁动的心走向宁静。

我的写作因此与文学关系疏浅,或者竟是无关也可能。我只是走得不明不白,不由得唠叨;走得孤单寂寞,四下里张望;走得怵目惊心,便向着不知所终的方向祈祷。我仅仅算是一个写作者吧,与任何“学”都不沾边儿。学,是挺讲究的东西,尤其需要公认。数学、哲学、美学,还有文学,都不是打打闹闹的事。写作不然,没那么多规矩,痴人说梦也可,捕风捉影也行,满腹狐疑终无所归都能算数。当然,文责自负。

2.我想,何妨就把“文学”与“写作”分开,文学留给作家,写作单让给一些不守规矩的寻觅者。文学或有其更为高深广大的使命,值得仰望,写作则可平易些个,无辜而落生斯世者,尤其生来长去还是不大通透的一类,都可以不管不顾地走一走这条路。没别的意思,只是说写作可以跟文学不一样,不必拿种种成习去勉强它。

3.写作者,未必能够塑造真实的他人,只可能塑造真实的自己。——前人也这么说过。你靠什么来塑造他人?你只可能像我一样,以史铁生之心度他人之腹,以自己心中的阴暗去追查张三的阴暗,以自己心中的光明去拓展张三的光明,你只能以自己的血肉和心智去塑造。那么,与其说这是塑造,倒不如说是受造,与其说是写作者塑造了张三,莫如说是写作者经由张三而有了新在。

4.因此我向往着这样的写作—史铁生曾称之为“写作之夜”。当白昼的一切明智与迷障都消散了以后,黑夜要我用另一种眼睛看这世界。很可能是第五只眼睛,第三他不是外来者,第四他也没有特异功能,他是对生命意义不肯放松的累人的眼睛。如果还有什么别的眼睛,尽可都排在他面前,总之这是最后的眼睛,是对白昼表示怀疑而对黑夜素有期盼的眼睛。这样的写作或这样的眼睛,不看重成品,看重的是受造之中的那缕游魂,看重那游魂之种种可能的去向,看重那徘徊所携带的消息。因为,在这样的消息里,才能看清一个人,一个犹豫、困惑的人,一个受造者;比如说我才有可能看看史铁生到底是什么,并由此对他的未来保持住兴趣和信心。

幸亏写作可以这样,否则他轮椅下的路早也就走完了。有很多人问过我:史铁生从20岁上就困在屋子里,他哪儿来的那么多可写的?借此机会我也算作出回答:白昼的清晰是有限的,黑夜却是辽阔无边。

\newpage

\section{史铁生:写作就是要解决自己的问题}

\emph{原文出自《黄河文学》2006年第6-7期《史铁生:扶轮问路的哲人》}
\vspace{2em}

和歌:铁生要不是被固定在这儿的话,凭他的那种灵性和生命力,不定会在别的领域做出什么大事来呢。

周国平:我觉得他还是写作。

史铁生:最好是。但我觉得有种危险在那儿呀。我是个——用我奶奶的话,还有北京话说是——“怵窝子”,非常胆小,不敢到外面去。小时候我的性格就是这样。还有个朋友也说,你的这些东西可以总结成一个词:恐惧。我觉得他说得太好了。我从来是恐惧的,对这个世界。因为恐惧,才会对爱、宗教信仰呀,有着本能的向往。凭我的“怵窝子”,写作我可能根本就不敢想,写了也不敢拿出去。可能就会在七七、七八年跟着我的理工科同学去考个理工科大学,然后再去干个什么事儿。然后会尽力把它干好,但干不好,凭我的魄力,我还不能放弃它,去自己写作什么的,那我可就惨了。

周国平:(大笑)不会的!

和歌:您觉得在写作方面受哪些作家或是作品的影响比较大?

史铁生:好像没有……

和歌:就想听您说找不到师承,嘿嘿!

史铁生:其实我看的文学作品,小说并不多,就是现在我也几乎看不完一本书,除非是很短的一篇小说。因为我主要是看他的方式。他的方式就是他的态度,他看世界的态度。我一旦把这个看明白了,我就不要看他了。所以我说从我插队以来,一直到后来生病,我真是想弄清楚自己的问题,因为我自己的问题实在是太严重了,涉及要不要活下去的问题,一旦你觉得应该活下去,就要问为什么要活下去?这么付出我值吗?我是不是冒傻气呢?受一辈子罪还要活下去。就是这样的问题。其实我的写作一直是在这样的氛围中,别的我都不太关注。

和歌:一直是在追问。

史铁生:活得好又怎么样?万事顺利又怎么样?是不是还是荒诞的?这些事情我可能想得早些,因为我二十岁就已经瘫痪了,随之而来的必定是一个问题接着另一个问题:你要不要活下去?为什么要活下去?那这是肯定的。所以我觉得我写作是在回答我自己的问题。我得想!所以有时候我就想写一篇这样的东西,但不见得对别人有用。有时候要少读书,不用读那么多书。不如多想。古圣贤的时候没有多少书,事儿都是他们想出来的。

周国平:天才不用读太多的书,中等之才还是要读书,多受启发。

史铁生:我说的是有的时候不用过分强调。不读书是不行,那是许多高级脑子想出来的东西。

周国平:读书最有用的一点是推动你思考,引发你思考。

史铁生:还有就是支持你思考。就是说你有时候想到了,你不信,有一天你看到了,孔子也这么想,亚里士多德也这么想,你就信了。好,那就接着想。

和歌:就是说走在思考的正确的路上了,跟圣贤一致了。

史铁生:我最突出的感受就是,如果要是你自己想到过的问题,在读的时候撞上了,人家比你说得棒,比你想得完全,这个你就永远都记得住。而且一下子就通了。然后你就开始赞叹,人家名著就是不一样!

周国平:所以读书是在寻求自己的问题的回答,这才是真读书。不光是正在想的问题。实际上一个人的问题始终就是那么几个,差不多不变的。

史铁生:从各个角度来审视这几个问题。

周国平:但这得是优秀的人才会有这样的问题。

史铁生:博尔赫斯说过,可能世界上就只有一件事,所有的事都是它的不同侧面。

和歌:就像您也说过,所有的作家都是在从不同的角度写作同一个故事。

周国平:应该说每个作家都在写着同一个故事。那还是指那些真正伟大的作家,达到一定高度的,才能这么说。很多作家都不知道在说什么。

史铁生:对对对!他不想。

周国平:而且绝大多数作家是没有问题的。

史铁生:对,没有问题。比如说死的问题。我发现在医院里一般人都怕说这个问题。有一次我遇见一个诗人,我说到这个问题,他说你别说。我说你连死的问题都没想过你写什么诗呀?

周国平:这是个灵魂的问题。没有问题就没有灵魂。

史铁生:没有灵魂就没有问题,那就剩了有没有房子和车子的问题了。

和歌:灵魂就是生命的主题。剩下的就是些零散的东西。铁生的作品就是,没有弄出复杂的情节呀、虚构呀。

史铁生:就是庄子乘物游心。

和歌:就像存在主义的那种小说,比如说萨特的那种小说,它好像是有一个内核,其实主角是在木然地走,但最后有一个对于自我的存在的问题在那里。他木然地走是因为意识到荒诞。但看我们现在的许多小说,主人公是在那里活动,但他活动到最后,连个大的问题都没有。

史铁生:问题就在这里,没有问题。其实各行都是这样。你只要搞人文,搞科学,你提不出问题来就完了。爱因斯坦说了,你提问题比解决问题更重要。你提不出问题来,你干吗呢?

周国平:大师就是伟大的提问者。

史铁生:就是这个意思,在别人结束的地方你开始了。

和歌:找出一个缺,才能产生问题。

史铁生:所以有人问我,写作是怎么回事?其实我写作就是要解决自己的问题。苏格拉底说,要认识你自己,真是这么回事。没有别的原因。刚开始是为谋生,我想来想去只能做这个。开始写作呢就要像那么回事,带有模仿的意思,任何人写作可能刚开始都是这样。等你写到一定时候,你就是解决自己的问题,解决自己弄不明白的问题。

周国平:这时候一个真正的作家才诞生了,那以前都还是一个习作者。

和歌:国平也说他写的东西是自解自劝。

史铁生:就是这样。有时候你看,网上的小文章写得很好,那作者不以写作为生,偶尔写这个,但他是有问题的,他是从问题出发的。写多了的人尤其是要注意这个。据说有人一天要写一篇散文。我觉得这是每日大便一次的感觉!(众人大笑)这你怎么能保证每日一篇呢?他压着自己一定得写。

和歌:可现在网络写手一天必须得写一万多字,坐在马桶上还在写呢。

史铁生:好家伙,我也不理解。那也是一种能耐。

周国平:那种状态和写作没有关系,那是生产。

和歌:那是苦役犯。您现在想得最多的问题是什么问题?

史铁生:嗨,想得最多的还是那个问题。但那个问题确实很严重。所以我看书就特别杂,不光是看小说。我老想知道别人那么多故事干吗使?我看那个杂书,比较邪门的书。你比如说灵魂到底有没有?最近读到一本书,是美国一个人类学家,跟踪研究一个墨西哥的巫师。他本来想去分析研究人家,结果反被人家给改造了。那个挺邪乎的。存在这事儿不好说。我是不是全信,单说。就是说科学所圈定的那点儿东西,太简单太少了。你只是宇宙里的一种存在,因为我们的行动所具有的时间性、逻辑性,就把我们给框定成了一种时间性逻辑性的动物,我们就遵循一种方法,把它奉为圭臬奉为神圣。实际上存在的状态太多了。说起来又有一个问题,现代的社会是怎么活着都对。可能作为梦想你怎么活着都对,你自己的信念,你自己的梦想,都可以。但这里面还有一个社会问题、政治问题。这个说起来太长了。你的问题后面永远有问题。最后解决了?

周国平:真正的问题解决不了。

史铁生:永不解决的问题是真正的问题,那你说这岂不是荒诞吗?最后你发现作为一个永恒的过程而言,只有美是它最终的解答。别的没有,别的都很荒诞。只有美可以是不断超越的。

周国平:还有宗教,神秘。

史铁生:对,这都包含在里面。真正的美里面一定有这一层。所以,真、善、美,这三个字我一直觉得它们是递进的。先有真,说是什么就是什么,但这个东西是不够的,背后还有许多东西。所以要有善的标准。善的东西有时候可以容忍假。艺术可以虚构,那是善的东西。但善的东西走来走去,有时候也很荒诞。只有美在最后作支撑,所以美有时候很神圣。你到一个城市,它的美如何?就全说了。你是文盲,然后你是科盲,最后到美盲是极致。如果你是美盲,那反过去一看前面那几个肯定也全盲。

和歌:一旦到美盲就有点儿行尸走肉的意思了。

周国平:从个人来说,真善美可以说是递进的;从人类来说,可能是无奈的后退。真得不到,那我们来个善吧,主观性强一点儿;可是善也得不到,那还是人与人之间的关系问题;那就美吧,美我个体自己就能支配了。

史铁生:对对对,这很对。其实最后你就是……

和歌:是向内在的退。

史铁生:有句话说,穷则独善其身,达则兼济天下。到最高境界就只能独善其身,你不可能要求世界全都是怎么样的。刘小枫他们说的政治哲学,也有这意思。你不能用理想来要求一切,最后要靠政治来平衡,平衡大家伙儿。很多人在一块生活呢!我的那个“丁一”呀,有点儿不谋而合,或者说有了这些坚定了我的想法。丁一呀也是很好的理想。你说丁一有哪点儿不好?多跟几个人发生爱情有什么不好?爱情不是好东西吗?好东西为什么要限制在最小的范围?推而广之,有什么不好?但是不成。只要有三个人,就已然要出政治。一个人,独自的理想,两个人可以有爱情,三个人,就要出政治。它要平衡关系。你把理想放在政治那儿,就要出问题。戏剧呢,是一种艺术,是一种理想,就像爱情是一样,家庭就是现实。人要是老像戏剧一样地活着,就不成。不可能的可能,不现实的实现,在戏剧那儿可以,但不能拿到社会上。拿到社会上,就一定要坏,出娄子。顾城的事情就是这样,他想要强行地维系一个伊甸园,就不成。这里面的关系是要变的。谢烨一旦要走向现实,要想孩子怎么办?理想主义者就不干了。这就坏了。

和歌:他就崩溃了,是不是艺术家的那根弦更脆弱?

周国平:谁都不行,那种时候谁都是艺术家。谁都受不了。

史铁生:政治家让你厌烦,但不能没有。其实你想人类的矛盾就是这样。

和歌:顾城要是有点儿政治家的方式方法是不是会好一些?他直接就拿起屠刀了。

史铁生:他要能那么冷静,他就不是他了。

和歌:也就不会想到伊甸园了。

史铁生:他就不会想到去做这事。像哈姆雷特,既是艺术家,又面对了政治,他老是犹豫,老觉得这事不能干。

\newpage

\section{莫言:我为什么写作}
\emph{本文内容来自莫言在绍兴文理学院的讲演,摘自《传记文学》2012年第11期。}
\vspace{2em}

我今天演讲的题目叫“我为什么写作”。以我个人的经验看,一个作家从他写作的开始,一直到他写作的终止,在这个漫长的写作过程当中,他的写作目的并不是一成不变的,并不是说一开始确定了,然后一直没有变化,它是随着作家本身创作经验的丰富、社会的变迁、作家个人各方面的一些变化而变化的。刚开始的时候你拿起笔来写小说或者诗歌,一直到你写不动了为止,其间可能要经过很多次的变化和发展。

\subsection{为一天三顿吃饺子的幸福生活而写作}

我最初的文学动机跟鲁迅确实是有天壤之别的。鲁迅先生以国家为基准,以民族为基准,要把当时的中国的“铁屋子”凿开几个洞,放进几线光明来促进社会变革。

而我早年是农民,每年都在地上凿很多的洞。我很早就辍学,没有读过几本书。我的读书经验也在一些散文里零星提到过。因为当时的书很少,每个村庄里大概也就那么几部书,比如说老张家有一本残缺不全的《三国演义》,李大叔家可能有两册《西游记》,还有谁谁家还有几本什么书。当时这些书读完以后,我感到我已经把世界上所有的书都读完了。当兵以后,我才知道自己目光短浅,是井底的青蛙,看到的天空太小了。

我的一个邻居——山东大学的一个学生,学中文的,后来被划成“右派”——每天跟我在一起劳动。劳动的间隙里,他“右派”本性难改,就经常向我讲述他在济南上大学的时候所知道的作家故事。其中讲到一个作家——一个很有名的写红色经典的作家,说他的生活非常腐败,一天三顿都吃饺子,早晨、中午、晚上都吃饺子。在上世纪60、70年代的农村,每年只有到了春节大年夜里,才能吃一顿饺子,饺子分两种颜色,一种是白色的白面,一种是黑色的粗面。我想:“一个人竟然富裕到可以一天三顿吃饺子,这不是比毛主席的生活还要好吗?”我们经常产生一种幻想,饥肠辘辘时就想:“毛主席吃什么?”有人说肯定是每天早晨吃两根油条,有人说肯定是大白菜炖肥肉。我们都不敢想象毛主席一天三顿吃饺子,这个邻居居然说济南一个作家一天三顿吃饺子。我说:“如果我当了作家,是不是也可以一天三顿吃饺子?”他说:“那当然,只要你能够写出一本书来,出版以后稿费就很多,一天三顿吃饺子就没有问题。”

那个时候,我就开始产生一种文学的梦想。所以说我为什么写作呢?最主要是最早的时候我就想为过上一天三顿吃饺子的幸福生活而写作。这跟鲁迅为了救治中国人麻木的灵魂相比,差别是多么大。鲁迅也不可能产生我这种低俗的想法,也跟他的出身有关,我今天参观的时候,发现鲁迅家是一个大户人家,爷爷是进士,家里有那么多房子,曾经过过非常富贵的生活,他知道富人家生活的内容,不会像我们这样低俗。

\subsection{为写出跟别人不一样的小说而写作}

1984年我考到了解放军艺术学院文学系。那个时期的写作目的,已经不那么低俗了。“军艺”的环境彻底改变了我当初那种文学观念。1984、1985年的时候有很多非常红非常流行的小说。我不满足这些小说,觉得它们并不像大家说得那么好,起码不是我最喜欢的小说。那么什么是我最喜欢的小说?我心里也没有一个准确的想法,但总感觉我应该写一些跟当时很走红很受欢迎的小说不一样的作品。这就是当时我梦寐以求的事情。

后来果然做了一个很好的梦,梦到在秋天的原野上,有一大片萝卜地——我们老家有一种很大的红萝卜,萝卜皮就像我们这个大讲堂后面的标语一样鲜红。太阳刚刚升起——太阳也是鲜红的,太阳下走来一个身穿红衣的丰满的少女,手里拿着一个鱼叉,来到这片萝卜地里,用鱼叉叉起了萝卜,然后就迎着太阳走了。

梦醒以后就跟我同寝室的同学们讲:“我做了一个梦,一个非常美的梦。”有的同学说“你很弗洛伊德嘛”。我说我是不是可以把它写成小说。一个同学说你能写成当然很好。我的同学给了我很大的鼓励。我就在这个梦境的基础上,结合个人的一段经历,写了一篇小说叫《透明的红萝卜》,这就是我的成名作。今天在座的有我一个老同学,当我讲到这里,他一定会回忆起我们当时在一个寝室里学习的景象,以及我的小说发表前后他做出的一些贡献——他们当时为了抬举我,开讨论会一块儿为我的小说说好话。

《透明的红萝卜》这部小说的发表,对我来讲确实是一个转折,因为在这之前我写的很多小说实际上都是很“革命”的,是一种主题先行的小说。当时我认为小说能够配合我们的政策,能够配合我们某项运动是一件非常光荣、了不起的事情。解放军刊物编辑悄悄地跟我说:“我们马上要发一批配合整党运动的小说,假如你的小说能变成整党的读物的话,你一下子就可以成名了。”我也真的向这方面来努力,无非就是编一个“文化大革命”期间,怎样跟“四人帮”作斗争,怎样坚持毛主席的革命路线的小说。这些小说可以发表,在当时也有可能得到这样那样的奖项。但写完了《透明的红萝卜》,回头再来看这些小说,就感到这些小说根本性的缺陷就是虚假。上世纪80年代之前,“文革”前后,我们尽管高举革命现实主义的旗帜,实际上,我认为这个现实主义完全是一种虚构的、空虚的现实主义,不是一种真正的现实主义。当时明明大多数老百姓饥肠辘辘,但是我们自认为生活得很好;当时中国人的生活水平在全世界明明是很低的,但是我们还认为全世界有三分之二的人生活在比我们要艰苦得多的水深火热的生活中,我们要去解放他们,拯救他们,把他们从水深火热中拯救出来。这样就确定了我们这种现实主义本质上是虚假的,前提就是虚假的,所以这种小说肯定也是假的。

在《透明的红萝卜》的创作过程中,我认识到现实主义其实是非常宽泛的,并不是说像镜子一样地反映生活,并不是说我原封不动地把生活中发生的事件搬到作品中就是现实主义。现实主义实际上也允许大胆的虚构,也允许大胆的夸张,也允许搞魔幻。

上世纪80年代的时候正好是我们这一批人恶补西方文学的时代。在“文革”前后,或者说在上世纪70年代、60年代、50年代这30年之间,中国人的阅读面是非常狭窄的。除了读中国自己的作家写的红色经典之外,还可以读到苏联的小说,当然也可以读到东欧、越南的一些小说,总而言之是社会主义阵营的,当然还可以读一些经典的,像托尔斯泰的小说、法国的批判现实主义小说。但是在这几十年当中,西方的现代派的作品,像法国的新小说、美国的意识流,尤其是到了60年代拉丁美洲的爆炸文学、魔幻现实主义,我们基本上是不知道的。

上世纪80年代初期思想解放,30年来积累下来的西方作品一夜之间好像全部都到中国来了。那个时候,我们真的有点像饥饿的牛突然进了菜园子一样,大白菜也好,萝卜也好,不知道该吃哪一口,感到每一本书似乎都是非常好的。这样一种疯狂的阅读也就是一种恶补,它产生了一个非常积极的作用是让我们认识到小说的写法、技巧是无穷无尽的。小说的写法非常多,许多我们过去认为不可以写到小说中的素材,实际上都是上好的小说材料。

过去我觉得我最愁的是找不到可以写的故事,挖空心思地编造,去报纸里面找,去中央的文件里找,但是找来找去都不对。写完《透明的红萝卜》以后,我才知道我过去的生活经验里实际上有许许多多的小说素材。像村庄的左邻右舍,像我自己在某个地方的一段劳动经历,甚至河流里的几条鱼,我放牧过的几头牛羊,都可以堂而皇之地写到小说里去。而且在我几十年的农村生活中,自己家的爷爷奶奶、邻居家的大爷大娘讲述的各种各样的故事,都可以变成创作的宝贵资源。一些妖魔鬼怪的故事,一会儿黄鼠狼变成了女人,一会儿狐狸变成了英俊小生,一会儿一棵大树突然变得灵验,一会儿哪个地方出了一个吊死鬼。突然有一天讲到历史传奇,在某个桥头发生过一场战斗,战斗过程中有一支枪因为打得太多,枪膛发热,后来一看,枪筒长出两公分。这些东西都非常夸张、非常传奇,这时候全部都到我眼前来了。

我在解放军艺术学院两年的时间里,一边要听课——因为我们是军队,一边要跑操,还要参加各种各样的党团活动,即便在那么忙的情况下,还是写了七八十万字的小说。就是因为像《透明的红萝卜》这种小说一下子开了记忆的闸门,发现了一个宝库。过去是到处找小说素材,现在是感到小说像狗一样跟在自己屁股后边追着我。经常在我写一篇小说的时候,另外一篇小说突然又冒出来了,那时就感到许许多多的小说在排着队等着我去写。

当然这个过程也没有持续多长时间。写了两三年以后,突然有一段时间,感到没有东西写了,另外也感到这些东西写得有点厌烦了。这时候我又在寻求一种新的变化,因为我想《透明的红萝卜》还是一篇儿童小说。尽管当时我30多岁,但是还是以一种儿童的视角、儿童的感觉来写的,这部小说还带着很多童话色彩。小说里一个小孩子可以听到头发落地的声音,可以在三九寒天只穿一条短裤,光着背而且身上毫无寒冷的感受,可以用手抓着烧红的铁钻子非常坦然地走很远……这些东西都极端夸张。

\subsection{为证实自己而写作}

写完《透明的红萝卜》和后来的一系列作品之后,进入到1985年年底的时候,因为一个契机,我写了一部小说《红高粱家族》。

解放军艺术学院是属于总政治部管理的学校,解放军总政文化部开了一个军事文学创作研讨会。在会上军队的很多老一代作家忧心忡忡,拿苏联军事文学跟中国的军事文学相比较,说苏联的卫国战争只打了四年,但是有关卫国战争的小说层出不穷,而且好作品很多,写卫国战争的作家据说已经出了五代,一代又一代的作家都在写卫国战争;而我们中国共产党领导的革命战争长达28年——还不加上对越南的自卫还击战,为什么就产生不了像苏联那么多那么好的军事小说呢?最后的结论就是因为“文化大革命”把一批老作家给耽搁了。他们忧心忡忡的一个原因是这一批有过战争经验的老作家,有非常丰富的生活经验,有很多素材,但是因为“文革”的耽搁,他们想写却心有余而力不足;而我们这一批年轻作家,有才华有经历也有技巧,但就是没有战争经验。因此他们认为中国的军事文学前途非常令人忧虑,非常不光明。

当时我是跳出来发话的一个,初生牛犊不怕虎。我说苏联的五代写卫国战争的作家有很多并没参加过卫国战争。我们尽管没有像你们老一辈作家一样参加过抗日战争、解放战争,但是从你们的作品里也知道了很多战争的经验,从身边的老人嘴里也听到了很多关于战争的传说,完全可以用这种资料来弥补我们战争经验的不足,完全可以用想象力来弥补没有亲身实践的不足。举个例子,譬如说尽管我没有杀过人,没有像你们一样在战争场上跟敌人搏斗拼刺刀、亲手杀死过敌人,但是我小时候曾经在家里杀过好几只鸡,完全可以把杀鸡的经验移植到杀人上来。

对于我那种说法,当时很多老同志不以为然,还有一个人悄悄地问:“这个人叫什么名字?”我回去以后憋了一股劲——我一定要写出一部跟战争有关的小说来,这就是《红高梁家族》。

刚才我说过,我可以用我个人的经验来弥补没有战争经验的不足。《红高粱家族》小说里曾经有过这样的场面描写,描写游击队战士用大刀把敌人的头颅砍掉,敌人被砍掉头颅之后,脖子上的皮肤一下子就褪下去了。后来有一年我在西安临潼疗养院碰到一个老红军,他好像很热爱文学,还看过这部小说。他问我:“你的《红高梁家族》里写到鬼子被砍掉了头颅.脖子上的皮褪下去,你是怎么知道的呢?”我说:“我杀鸡的时候看到的就是这样。”他说这跟杀人是一样嘛。我说:“我也不知道是一样还是不一样,您既然说一样,那肯定是一样的。即便不一样也不要紧,因为我的读者里像您这样的老革命,像您这样有过杀人经验的人非常少。只要我写得逼真就好,写得每一个细节都非常生动,就像我亲眼所见一样。”我这个细节描写所产生足够的说服力就会让读者信服,让读者认为我是一个参加过很长期间的革命战争、立过很多战功的老军人。所以有人当时认为我已经60多岁了,见面以后发现我才30来岁,感到很吃惊。

也就是说,《红高梁家族》这部作品写作的目的是要证实自己,没有战争经验的人也完全可以写战争。这里还有一个歪理,很多事情未必要亲身体验:我们过去老是强调一个作家要体验生活,老是强调生活对艺术、小说的决定性作用,我觉得有点过头了。当然,从根本上来讲,没有生活确实也没有文学。一个作家生活经验的丰厚与否,决定了他创作的成就大小。但是我觉得这话如果过分强调的话,会走向反面。从某种意义上来讲,没有战争经验的人写出来的战争也许更有个性,因为这是属于他自己的,是他的个人经验,建立在他个人经验的基础上的一种延伸想象。就像没有谈过恋爱的人,写起爱情来也许会写得更加美好,是同一个道理。因为情场老手一般写不了爱情,他已经没有这种真正的感情了,已经知道所谓的男女恋爱本质是怎么回事,本质是什么。只有一直没有谈过恋爱的人才会把爱情想象得无限美好。

《红高粱家族》获得声誉之后,关于这部小说的解读也越来越多。本来我写的时候也没有想到,既然别人说了我也就顺水推舟。后来关于《红高粱家族》的写作目的就变得非常的复杂——不仅仅要证明自己能写战争小说,而是要为祖先树碑立传,要创造一种新的叙事视角,要打通历史跟现实之间的界限……这是我写的时候根本没有想到的,写的时候怎么样痛快,怎么样顺畅,就怎么样写。

《红高梁家族》一开始写“我爷爷”、“我奶奶”,后来一些评论家说这是莫言的发明创造。我当时实际上是逼出来的,我想如果用第一人称来写祖先的故事显得很不自然,肯定没法写,我不可能变成“我爷爷”、“我奶奶”;如果用第三人称来写祖先的故事显得很陈旧、很笨拙;用“我爷爷”、“我奶奶”这种写法,我觉得非常自由。想抒发我个人感受的时候,我就跳出来。我要写“我爷爷”、“我奶奶”的这一时刻,我仿佛变成了“我爷爷”、“我奶奶”他们本人,能够进入他们的内心世界。而且可以把我当下的生活跟我所描写的历史生活结合在一起,完全没有了历史跟现实之间的障碍,非常自由地出入于历史和现实之中。就像我们现在经常看的东北“二人转”,一方面在舞台上表演,一方面跟舞台下的观众打情骂俏,很自由。《红高梁家族》的叙事视角跟东北“二人转”叙事的角度实际上是一样的。跳进跳出,台上台下,历史现实,都融汇在一起。

\subsection{为农民和技巧试验而写作}

到1987年的时候,我的创作目的又发生了一个变化,这个时候我真的是要为农民说话,为农民写作。

1987年,山东南部的一个县发生了“蒜薹事件”,震动了全国。那个地方生产大蒜,农民收获了大量的蒜薹,但是由于官僚主义、官员腐败,政府部门办事不力,包括地方的封闭,不让外地客商进入,导致农民辛辛苦苦所种的几千万斤蒜薹全部腐烂变质。愤怒的农民就把他们的蒜薹推着、拉着运往县城,包围了县政府,用腐烂的蒜薹堵塞了道路,要求见县长。县长不敢见农民,跑到一个地方去躲起来,农民就冲进县政府,火烧了县政府办公大楼,砸了县长办公室的电话机,结果就变成了一个非常大的事件,因为建国以后还没有农民敢这样大胆地造反。

当时我正在故乡休假,从《大众日报》上读到了这条新闻。这个时候我就感觉到我心里这种农民的本性被唤醒了。尽管当时我已在北京工作,又是解放军的一个军官,已经脱离了农村,不吃庄户饭,但是我觉得我本质上、骨子里还是一个农民。这个事件也就发生在我的家乡,是村庄里的事情。在这个“蒜薹事件”中,后来很多领头闹事的农民被抓了起来,并被判了刑——当然有些官员也被撤职了。这个时候我就觉得我应该为农民说话。我就要以“蒜薹事件”作为素材写一部为农民鸣不平的小说,为农民呼吁。

由于当时心情非常激动,可以说是心潮澎湃,所以写这个小说用的时间非常短,用了一个月零三天。事后有很多读者,包括发生“蒜薹事件”的县里的一些读者也给我写信,说:“你是不是秘密地到我们县里来采访过?你写的那个‘四叔’就是我爸。”当时高密县有一个副县长和我是朋友,正好在这个县里面代职,他回来悄悄地向我传达了当地某些官员对我的看法:“莫言什么时候敢到我们县来,把他的腿给打断。”他叫我千万别去。我说我干嘛要去,根本不去,我又没写他们县,写的是一个天堂县——虚构的一个县名。

为什么这个小说会写得那么快,而且当地的农民觉得写的就是他们的心理?其实可以看到天下农民的遭遇和命运都是差不多的。我实际上是以我生活了20多年的村庄作为原型来写的:我家的房子,我家房子后面的一片槐树林,槐树林后面的一条河流,河流上的小石桥,村头小庙,村南的一片一眼望不到边的黄麻地……完全是以我这个村庄作为描写环境,而且小说里的主要人物写的都是我的亲属。小说的主人公实际上就是我的一个四叔,他当然不是去卖蒜薹,而是去卖甜菜。他拉了一车甜菜,非常幸福地想卖了甜菜换了钱给儿子娶媳妇,却被公社书记的一辆车给撞死了。把人撞死了,把拉车的牛也撞死了,那是一头怀孕的母牛。车也轧得粉碎。最后这一辆车、一头怀孕的母牛加上我四叔一条人命,赔偿了3300块钱。

在我对四叔的不幸去世感到痛心疾首的时候,我发现四叔的一个儿子竟然在公社大院里看电视。因为他们把死者的尸首放到公社大院里,说你们不给我们解决问题,我们就不火化、不下葬。在这么一个非常时刻,电视机里正在放电视剧《霍元甲》,四叔的这个儿子竟然把父亲的尸首扔到一边,跑到里边津津有味地看《霍元甲》。这让我心里感到很凉。我想争什么啊,无非是从3300争到13000,争到了13000,没准四叔的几个儿子还要打架。3300还好分,13000没准就眼红了。而且也没有办法,因为这个公社书记还是我的一个瓜蔓子亲戚,这个亲戚找到了我的父亲,最后就不了了之了。

但我总感觉心里面压着很大的一股气,所以在写的时候就把生活当中自己积累了很久很久、很沉痛的一些感情写到小说里去。这篇小说按说是一部主题先行的小说,而且是一篇完全以生活中发生的真实事件为原型的小说。它之所以没有变成一部简单的说教作品,我想在于我写的是自己非常熟悉的地方,塑造人物的时候写了自己的亲人。也就是说这部小说之所以还能够勉强站得住,最重要的就在于它塑造出了几个有性格的、能够站得住的人物,并没有被事件本身所限制。如果我仅仅是根据事件来写,而忘了小说的根本任务是塑造人物,那么这部小说也是写得不成功的。

我想这也是一种歪打正着。就是说这个阶段我是要为农民说话,这个阶段大概持续了两三年。写了好几部小说,就是为农民鸣不平的,反映当时农村的农民生活的种种不公平境遇,譬如各种各样的“苛捐杂税”,农民的卖粮难、卖棉花难这一类题材的小说。

写完《天堂蒜薹之歌》之后,我发现这确实也不是一个路子,小说归根结底还是不应该这样写。想用小说来解决某一社会问题的想法,像我刚才说的那样,是非常天真幼稚的。这个时候我就特别迷恋小说的技巧,我认为一个小说家应该在小说文体上做出贡献,也应该对小说的文学语言、结构、叙事等进行大大的探索。像马原这些作家,在这方面积累了很多成功的经验。

写完《天堂蒜薹之歌》之后,我就进入到一个技巧试验的时期。这个时候大概是1988年,我又写了一部很多同学都不知道的小说《十三步》,以一个中学为背景,写了中学里的一些老师和学生。主要还是对小说进行了许许多多叙事角度的试验。这部小说里我把汉语里所有的人称都试验了一遍——我、你、他、我们、你们、他们,各种叙事角度不断变换。我个人认为这是一部真正的试验小说。同时我也发现,当我把所有的汉语人称都试验过一遍之后,这个小说的结构自然就产生了。

当然这带来了一个巨大的问题,就是小说阅读起来非常困难。有一年我到法国去,碰到了一个法国的读者,他说:“我读这个小说的法文译本时,用五种颜色的笔做了记号,但还是没有读明白,能不能给我解释一下是怎么回事,说的是什么?”我说:“我去年为了出文集,把《十三步》读了一遍,用六种颜色的笔做了记号,也没读明白,都忘了自己怎么写的了。”

这种写作是以技巧作为写作的主要目的。我为什么要写这个小说?因为我要进行技巧试验。不过这好像也不是一条正确的道路。因为读者归根结底是要读故事的,所以还是要依靠小说的人物、人物的命运来感染读者,唤起读者感情方面的共鸣——也许有极少数的作家、极少数的文学读者要读那个技巧。那么这样的小说无疑是自绝生路——谁来买你这个小说,谁来看你这个小说?而且这样的技巧试验很快就会“黔驴技穷”,再怎么变,你一个人能变出什么花样来?

\subsection{沿着鲁迅开辟的道路向前探索}

到了21世纪,写了一部重要的小说,就是《檀香刑》。《檀香刑》就是一部说书的小说。这个时候我就是把上世纪90年代写短篇的那种感受发扬了。我想应该用这样的方式和这样的感觉来写,以说书人的身份写这么一部小说。写这个小说的时候还借助了一些民间的戏曲,因为我的故乡有一种小戏叫“茂腔”,当然在小说里我把它改造成了“猫腔”。这部小说应该说是一个小说化的戏曲,或是戏曲化的小说。里边的很多人物实际上都是很脸谱化的,有花脸、花旦、老生、小丑,是一个戏曲的结构,很多语言都是押韵的,都是戏文。这样的小说,因为是戏剧化的语言,就不可能像鲁迅小说那么考究语言。它里面有很多语病,因为一个说书人的语言有很多夸张和重复,这都是允许存在的。这是这部小说的一个写作目的。

为什么写《檀香刑》呢?想恢复作家的说书人的身份,另外一个还是要向鲁迅学习。我在童年时期读鲁迅的《药》、《阿Q正传》,知道鲁迅对这种看客非常痛恨。鲁迅最大的一个发现就是发现了这种看客心理。但是我觉得鲁迅还没有描写刽子手的心理。

我觉得中国漫长的历史实际上也是一台大戏,在这个舞台上有不断被杀的和杀人的。更多的人,没有被杀或杀人,而是围着看热闹。所以当时中国的任何一场死刑都是老百姓的一场狂欢的喜剧,围观这个杀人场面的都是些善良的老百姓,尽管他们在看的时候也会感到惊心动魄,但是有这样的场面他们还是要来看。

即便到了现在,这种心理还是存在的。“文革”期间经常举行公审大会、万人大会,我也去参加过。我们都围绕着看,官方的目的是要杀一儆百,警戒老百姓不要犯罪,但老百姓却将它当戏剧看。鲁迅的小说里有很多这样的描写,但我觉得如果仅仅写了看客和受刑者,这场戏是不完整的,三缺一,所以我在《檀香刑》这部小说里就塑造了一个刽子手的形象。

我觉得刽子手跟罪犯是合演的关系。他们俩是在表演,而观众是看客。罪犯表现得越勇敢,越视死如归,越慷慨激昂,喝一大碗酒,然后高呼“二十年后又是一条好汉”,观众才会叫好,才感到满足。这个时候罪犯到底犯了什么罪行并不重要,杀人犯还是抢劫犯都不重要。无论他犯下多么十恶不赦的罪行,只要他在临死这一刻表现得像个男人,视死如归,那么观众就认为这是一条了不起的汉子,就为他鼓掌,为他喝彩。即便是一个被冤死的人,但是在受刑这一刻,他瘫了,吓得双腿罗圈了,变得神智不清楚了,那么所有的看客都不满足,所有的人都会鄙视他。这个时候刽子手也感觉没有意思,碰到一个窝囊废。刽子手碰到“好汉”也感到非常精彩,就是说像棋逢对手,碰到一个不怕死的汉子,“哥们儿你手下的活儿利索一点”,这是我们过去经常会看到的一些场面。《聊斋志异》里描写人要被杀头的时候对那刽子手说:“你活儿利索点。”刽子手说:“没问题,当年你请我喝过一次酒,我欠你个情来着,我今天特意把这刀磨得特别快。”一刀砍下去,这个人头在空中飞行的时候还高喊:“好快刀。”也就是说,到了后来一切都变得病态化了。

我们分析了罪犯的心理、看客的心理,那么这个杀人者——刽子手,到底是一种什么样的心理?这样的人在社会上地位是很低的。当时北京菜市口附近据说有一家肉铺,后来很多人都不去那儿买肉,因为它离刑场太近。刽子手这个行当是非常低贱的。上世纪90年代的时候我们翻译过一本法国小说——《刽子手合理杀人家族》,那个家族的后代很多都愿意承认自己的出身,他们是怎么活下来的?是用怎样的方式来安慰自己把这个活干下来的?对于这个,我在《檀香刑》中作了很多分析。他认为:“不是我在杀人,而是皇上在杀人,是国家在杀人,是法律在杀人,我不过是一个执行者,我是在替皇上完成一件工作。”后来,他又说:“我是一个手艺人,我是在完成一件手艺。”

封建社会里,一个人犯了最严重的罪行、最十恶不赦的罪行,就要用最漫长的方法来折磨。把死亡的过程拖延得越长,他们才感觉惩罚的力度越够。一刀杀了、一枪崩了,那是便宜你了,只有让你慢慢地死,让你不得好死,才会产生巨大的震撼力,才会让老百姓感到更加恐惧。但是结果,老百姓却把它当作了最精彩的大戏。

小说《檀香刑》里就写到把一个人连续五天钉到一个木桩上,如果刽子手在五天之内就让他死了,就要砍刽子手的头;刽子手让他活的时间越长,那么得到的奖赏就越多。刽子手一边给他使着酷刑,一边给他灌着参汤以延续他的生命,让老百姓看到他是在忍受怎样的刑罚。“你不是要反皇上吗?你不是要反叛朝廷吗?下场就是这样的”,封建社会下这种刑罚的心理就是这样的,就是让这人不得好死。刽子手就是要把活做好,凌迟不是要割五百刀吗?刽子手割了三千刀这个人还没有死。很多野史里有很多关于凌迟、腰斩等残酷刑罚的描写。

当然,我的《檀香刑》写完之后确实有很多人提出了强烈抗议,说看了这个小说后吓得多少天没睡着觉。这样说的多半是男人,而有一些女读者反而给我来信说写得太过瘾了。所以,有时候我觉得女性的神经比男性的神经还要坚强,并不是只有女人才害怕。总而言之,我想我是沿着鲁迅开辟的一条道路往前作了一些探索。

《檀香刑》这部小说得到了一些好评,当然批评的声音也一直非常强烈。我想这也是非常正常的。我也为自己辩护过,认为对这些残酷场面的描写当然值得商榷,但是让我删除我也不舍得。我觉得如果没有这些残酷场面,这部小说也不成立。因为这部小说的第一主人公是一个刽子手,如果不这样写,这个人物就丰满不了,就立不起来。尽管这样写可能会吓跑一些读者,但我觉得为了塑造这个人物,为了文学,这是值得的。

假如将来再处理这样的题材,是不是还要这样淋漓尽致?我也要认真想一想。因为在《红高粱家族》里,我也曾经写过一个日本强盗剥中国人的人皮这样一些描写。那时候这样的批评也存在,当时叫自然主义的描写。后来到了《檀香刑》,这种批评就更加猛烈。我写的时候也没有意识到这个问题有多么严重,后来反映的人多了,也促使我对这个问题进行了很多反思,希望我将来写的时候找一下有什么别的方式可以替换,又能塑造生动的人物,又能避免这种过分激烈的场面描写。

写完《檀香刑》,我又写了《四十一炮》《生死疲劳》这些小说。写《四十一炮》的时候,我是想对我的儿童视角写作作一个总结。因为我写的很多小说,尤其是中短篇小说里,有大量的儿童视角。《四十一炮》就是我用儿童视角写的一个长篇小说。

写《生死疲劳》就是想进一步从我们的民族传统和民族文化里发掘和寻找小说的资源。我使用了章回体来写这个小说,当然这是雕虫小技,谁都会用章回体。当然,对我来讲这也是一个符号,我希望用这种章回体来唤起我们对中国历史上的长篇章回体小说的一种回忆或者一种致敬。当然,也有一些评论家认为这种章回体太简单了。这部小说如果不用章回体它依然是成立的,用了章回体也不能说是一种了不起的发明创造——这肯定不是发明创造,这就是我对中国古典小说的致敬。

\newpage

\section{梁鸿:写作与世界的关系}

\emph{本文内容来自2019年3月14日梁鸿在伦敦光华书店的演讲(有删减)。}
\vspace{2em}

写作与世界的关系,就像魔术师与真相的关系。

真相从来只有一个:魔术师的表演不可能是真的。但是,大家却仍然为魔术师的表演所着迷,因为那里面包含着人类的想象力,人类对自身及世界的渴望,它探求的是可能性。

文学也是一样,它的目的不是在告诉你,真相就是这样,而是告诉你,它还有另外的可能性。这一可能性既来源于人类已经创造的事实——社会形态、文明结构和人性状态,也来源于人类内部所包含的可能的方向。在此意义上,文学与世界是在一种悖反、错位和隐喻中彼此彰显的。

这也是我今天想要给大家分享的三个层面。

\subsection{悖反关系}
写作不是简单地解释世界,而是背向这个世界。作家要走到阴影处,走到时尚、华丽的巨型建筑背后,去看那里世界的形态和道路的走向。并且,这一巨型建筑——也可以说巨型话语——越具有确定性,越需要作家转过去看看。这两者是相互依存的关系,就像阳光和阴影,肯定与否定,巨大与微小。

在巨型话语中看到日常人生的众生相,在喧嚣中寻找沉默的、安静的那一部分,并且,试图发掘它们之间如何互为生长,这是作家非常重要的任务。

《百年孤独》中马孔多小镇的扩张是历史的正面,是正在发生着的社会现实。在政治层面,它甚至可能是必然的,拉美的城市化、现代性,和中国的城市化、现代性在现代文明发展的过程中都是一种必然,但是,作家的任务不是要写马孔多发展的必然性,而是写出这一必然性中所同时生长出来的百年孤独。乌苏娜、奥雷连诺上校,他们被飓风一样的发展挟裹着,无法决定自己的命运,但又顽固地保持自己的存在。这一顽固性为他们赢得了尊严,也无形中成为一种力量和“历史的正面”博弈。

\subsection{小等于大,或者,小大于大的关系}
就像我刚才讲的《百年孤独》,一个村庄可以是全世界,马孔多从一个几户人家的村庄到繁荣的小镇,再到衰败,在此过程中,作者所描述的众生相包含了人类很多的形象,马孔多小镇的命运也几乎是整个现代拉丁美洲的命运。

一个人的爱情可以是全人类的爱情。譬如托尔斯泰的《安娜·卡列尼娜》。书中有一个细节是这样的。安娜和渥伦斯基吵架后,渥伦斯基独自一人坐着马车去彼得堡参加宴会。这时,安娜站在楼上的窗口边,看着英俊、衣着整齐的渥伦斯基走向马车,内心非常嫉妒、失落和不甘。他们一起私奔,但是,彼得堡的上流社会可以接受渥伦斯基,却不接受她。她只能躲在家里,任凭绝望吞噬自己。我想,在那一刻,安娜的痛苦不只是一个上层社会女性的痛苦,而是,所有爱情在现实面前遭遇壁垒的悲伤,是所有爱情都可能经历的悲伤。

一个人的梦魇可以是全人类的梦魇。卡夫卡《变形记》的第一句话是“格里高尔清晨起来,变成一只甲虫。”我们都清楚,人不可能变成甲虫,但是,每个人都经历过这样的时刻,恐惧、焦虑、担忧,非常非常压抑,卡夫卡把这种无形的情绪变成有形的语言给表达了出来。

回到我们的主题,今天我来到伦敦,坐在这个书店,给大家讲梁庄及梁庄的人生。

梁庄是中国当代村庄,我的家乡,它与英国,与此时正在倾听的你又有什么关系呢?我给大家讲一个小故事。以《出梁庄记》的结尾小黑女儿的故事为例。

当时我正在老家做《出梁庄记》的调查。一个早晨,小黑女儿奶奶带着小黑女儿到诊所看病,一检查,发现病情非常严重,我就赶紧开着车拉她们去县城医院。在医生给小黑女儿看病时,我试图联系我认识的一些人,派出所的、法院的,等等,看怎么办。所有的人,不管是医生、警察还是法官,都说,报案肯定是对的,但不建议报案,根据以往的经验,不报警肯定比报警的要伤害小黑女儿少一些。

那两天是我极为痛苦、煎熬的两天。事情不断回到原点。奶奶一会儿说要去报警,一会儿又揪着头发说对不起自己的儿子儿媳,把头往墙上撞,不如死了算了,再或者,就是把头低到腿上,默默地哭。奶奶心里是怯懦的,她其实不敢报警,她怕和邻居撕破脸,她怕人家倒打一耙,怕事情被人知道孙女将来找不到婆家,怕在村里、亲戚那里丢人。

有一次,我无意打开电脑里的录像存档,翻到采访小黑女儿的那一段,我又一次听到录像里我抑制不住的哭声。当时,我正问小黑女儿,为什么那个邻居老人第一次对她那样做时她没告诉奶奶,都那么疼了。小黑女儿慢慢说,因为她怕她奶奶伤心,因为哥哥太调皮,她奶奶已经很累了,她不想让她为自己多操心。

隔了那么多年,我仍然忍不住流了泪。她不知道她遭受了什么,而导致她进一步受伤害的原因竟然是心疼奶奶。

听着小黑女儿的诉说,再次看到她奶奶的花白头发,我想,也许事情发生的原因及处理的方式可能是中国式的,但那一刻,奶奶的痛苦一定包含在全人类的痛苦之内,它是人类永恒的痛苦和永远面临的困境。与此同时,小黑女儿因天真所遭受的伤害也是人类所有天真所遭受的伤害,它是真实的个人遭遇,但同时,却也好像人类世界的内在形象之一,与你我都息息相关。
在这个意义上,梁庄就是世界的中心,它承载了全世界人类在当代文明中的命运。

也可以说,写作与世界的关系是小等于大的关系,甚至,更大于这个世界,是小大于大的关系。

\subsection{隐喻关系}
写作与世界不是反映与被反映的关系,而是隐喻和象征的关系。当作家起笔写一个人物或一个村庄、某个庄园、某个故事时,他并不是按照现实的模型来写的,相反,它要把模型——这里的模型指的是日常观念中的现实认知——打碎,打成一个个元素,然后,再重新捏合。即使我们说《傲慢与偏见》是19世纪英国中产阶级的风俗画,也不能说它和现实一模一样,如同日本的浮世绘一样,把一些重要的人物突显起来,让他们成为某种隐喻和象征的存在。这也是故事之所以有价值的原因。

这是文学创作的基本起点。隐喻不只是一种修辞或创作手法,而是创造文学世界的基本起点,尤其是对于虚构文学而言。

如何既具有人类的众生相,但同时又能通向隐喻性和普遍性,这是所有作家所追求的艺术方向。

如布尔加科夫\footnote{米哈伊尔·阿法纳西耶维奇·布尔加科夫(1891年5月15日—1940年3月10日),前苏联作家,1909—1916年就读于基辅大学医学系,毕业后在斯摩棱斯克省乡村医院当医生,1918年返回基辅。1920年,他弃医从文,开始写作生涯,次年来到莫斯科,在《汽笛报》工作。1930年,在斯大林的亲自干预下,他被莫斯科艺术剧院录用为助理导演,业余坚持文学创作,并重新开始写他最重要的长篇小说《大师和玛格丽特》(1966年发表)。1940年3月10日布尔加科夫因患肾硬化去世。}的《大师与玛格丽特》透过以化身为教授的魔鬼撒旦考察人的灵魂为起点,考察了当时俄罗斯的社会现实和精神状态,这样一种超现实的起点本身就有强烈的隐喻色彩。马尔克斯、卡夫卡的小说都是这样的起点。

但是,也有另外一种方式,就是通过对现实世界非常清晰的精雕细刻,最后达到一种强烈的隐喻风格。前天去伦敦国家美术馆,看小汉斯荷尔拜因的肖像画,我被他画作强烈清晰的风格所吸引。他的画非常现实主义,每一个细节,哪怕是手上的褶皱,都会画出来,但是,当你观察整个肖像时,它们又具有强烈的超现实风格。人物像浮雕一样在空间中凸显出来,凌厉、强大,又孤独万分。

回到文学作品上,譬如像《傲慢与偏见》《包法利夫人》这样的作品,小说本身非常现实主义化,但是,最终却有强烈的隐喻风格。我想,这与作家对世界的理解,对人的理解都有很大关系。

我自己也刚完成一个小说《四象》。小说写一个患精神分裂的大学生,返回家乡河边的墓地放羊。在这里,他和三个人聊天、说话、学习,经过一系列事件之后,重返城市,被尊为大师。最后,他的精神基本上恢复了正常,但是,他再也听不到那三个人说话了。这三个人其实是墓地里的亡灵。

表面看来,这个故事有一点点魔幻性质,但我的目的并不是要写作一个魔幻现实主义的小说。在看荷尔拜因\footnote{小汉斯·荷尔拜因(约1497年—1543年11月29日以前)是德国画家,最擅长油画和版画,是欧洲北方文艺复兴时代的艺术家。其代表作有:木版画《死神之舞》。}的画时,我突然想到,我追求的就是这样一种风格:强烈的、清晰的真实性,这一真实性甚至是粗暴的,以至于最终能达到某种隐喻。所以,在我心里面,我一直把这墓地的三个人作为真实存在的人物来写的,他们在那个患精神分裂症的年轻人那里,也的确是真实存在的。

这个小说并不长,十三、四万字,但我写了两年多,已经改到第四遍,目前还在改,我个人非常喜欢,其中一个原因就是,它与现实世界是这样一种变形的、但又密切的关联。再回到开头,文学世界是一个既不同于现实世界,但又一定诞生于现实世界的世界,它与现实世界之间是你中有我、我中有你的关系,是看似一个面相,但其实却是由无数面相组成的关系。

我想,这也是写作与世界之间最基本的关系。谢谢。
\newpage

\chapter{如何写作}

\section{老舍:给初学写作者的建议}

\emph{本文摘自《老舍谈写作》(这本书精选了老舍关于写作的论述,深入浅出的阐述了“如何写作”这个主题)书中的《别怕动笔》小节,原载一九六〇年五月《文艺新兵》。}
\vspace{2em}

有不少初学写作的人感到苦恼:写不出来!
我的看法是:加紧学习,先别苦恼。

怎么学习呢?我看哪,第一步顶好是心中有什么就写什么,有多少就写多少。永远不敢动笔,就永远摸不着门儿。不敢下水,还学得会游泳么?自己动了笔,再去读书,或看刊物上登载的作品,就会明白一些写作的方法了。只有自己动过笔,才会更深入地了解别人的作品,学会一些窍门。好吧,就再写吧,还是有什么写什么,有多少写多少。又写完了一篇或半篇,就再去阅读别人的作品,也就得到更大的好处。

千万别着急,别刚一拿笔就想发表不发表。先想发表,不是实事求是的办法。假若有个人告诉我们:他刚下过两次水,可是决定马上去参加国际游泳比赛,我们会相信他能得胜而归吗?不会!我们必定这么鼓舞他:你的志愿很好,可是要拚命练习,不成功不拉倒。这样,你会有朝一日去参加国际比赛的。我看,写作也是这样。谁肯下功夫学习,谁就会成功,可不能希望初次动笔就名扬天下。我说有什么写什么,有多少写多少,正是为了练习,假若我们忽略了这个练习过程,而想马上去发表,那就不好办了。是呀,只写了半篇,再也写不下去,可怎么去发表呢?先不要为发表不发表着急,这么着急会使我们灰心丧气,不肯再学习。若是由学习观点来看呢,写了半篇就很不错啊,在这以前,不是连半篇也写不上来吗?

不知道我说的对不对,我总以为初学写作不宜先决定要写五十万字的一本小说或一部多幕剧。也许有人那么干过,而且的确一箭成功。但这究竟不是常见的事,我们不便自视过高,看不起基本练习。那个一箭成功的人,想必是文字已经写得很通顺,生活经验也丰富,而且懂得一些小说或剧本的写法。他下过苦功,可是山沟里练把式,我们不知道。我们应当知道自己的底。我们的文字的基础若还不十分好,生活经验也还有限,又不晓得小说或剧本的技巧,我们顶好是有什么写什么,有多少写多少,为的是练习,给创作预备条件。

首先是要把文字写通顺了。我说的有什么写什么,有多少写多少,正是为逐渐充实我们的文字表达能力。还是那句话:不是为发表。想想看,我们若是有了想起什么、看见什么,和听见什么就写得下来的能力,那该是多么可喜的事啊!即使我们一辈子不写一篇小说或一部剧本,可是我们的书信、报告、杂感等等,都能写得简练而生动,难道不是值得高兴的事吗?

当然,到了我们的文字能够得心应手的时候,我们就可以试写小说或剧本了。文学的工具是语言文字呀。

这可不是说:文学创作专靠文字,用不着别的东西。不是这样!政治思想、生活经验、文学修养……都是要紧的。我们不应只管文字,不顾其他。我在前面说的有什么写什么,和有多少就写多少,是指文字学习而言。这样能够叫我们敢于拿起笔来,不怕困难。在与动笔杆的同时,我们应当努力于政治学习,热情地参加各种活动,丰富生活经验,还要看戏,看电影,看文学作品。这样双管齐下,既常动笔,又关心政治与生活,我们的文字与思想就会得到进步,生活经验也逐渐丰富起来。我们就会既有值得写的资料,又有会写的本事了。

要学习写作,须先摸摸自己的底。自己的文字若还很差,就请按照我的建议去试试—有什么写什么,有多少写多少。同时,连写封家信或记点日记,都郑重其事地去干,当作练习写作的一种日课。文字的学习应当是随时随地的,不专限于写文章的时候。一个会写小说的当然也会写信,而一封出色的信也是文学作品—好的日记也是!文字有了点根底,可还是写不出文章来,又怎么办呢?应当去看看,自己想写的是什么,是小说,还是剧本?假若是小说或剧本,那就难怪写不出来。首先是:我们往往觉得自己的某些生活经验足够写一篇小说或一部三幕剧的。事实上,那点经验并不够支持这么一篇作品的。我们的那些生活经验在我们心中的时候仿佛是好大一堆,可以用之不竭。及至把它写在纸上的时候就并不是那么一大堆了,因为写在纸上的必是最值得写下来的,无关重要的都用不上,就好象一个大笋,看起来很粗很长,及至把外边的吃不得的皮子都剥去,就只剩下不大的一块了。我们没法子用这点笋炒出一大盘子菜来!

这样,假若我们一下手就先把那点生活经验记下来,写一千字也好,二千字也好,我们倒能得到好处。一来是,我们会由此体会出来,原来值得写在纸上的并不像我们想象的那么多,我们的生活经验还并不丰富。假若我们要写长篇的东西,就必须去积累更多的经验,以便选择。对了,写下来的事情必是经过选择的;随便把鸡毛蒜皮都写下来,不能成为文学作品。即须经过选择,那么用不着说,我们的生活经验越多,才越便于选择。是呀,手里只有一个苹果,怎么去选择呢?

二来是,用所谓的一大堆生活经验而写成的一千或二千字,可能是很好的一篇文章。这就使我们有了信心,敢再去拿起笔来。反之,我们非用那所谓的一大堆生活经验去写长篇小说或剧本不可,我们就可能始终不能成篇交卷,因而灰心丧气,不敢再写。\textbf{不要贪大!能把小的写好,才有把大的写好的希望。}况且,文章的好坏,不决定于字数的多少。一首千锤百炼的民歌,虽然只有四句或八句,也可以传诵全国。

还有:即使我们的那一段生活经验的确结结实实,只要写下来便是好东西,也还会碰到困难—写得干巴巴的,没有味道。这是怎么一回事呢?我看大概是这样:我们只知道这几个人,这一些事,而不知道更多的人与事,所以没法子运用更多的人与事来丰富那几个人与那一些事。是呀,一本小说或一本戏剧就是一个小世界,只有我们知道的真多,我们才能随时地写人、写事、写景、写对话,都活泼生动,写晴天就使读者感到天朗气清,心情舒畅,写一棵花就使人闻到了香味!我们必须深入生活,不断动笔!我们不妨今天描写一棵花,明天又试验描写一个人,今天记述一段事,明天试写一首抒情诗,去充实表达能力。生活越丰富,心里越宽绰;写的越勤,就会有得心应手的那么一天。是的,得下些功夫,把根底打好。别着急,别先考虑发表不发表。谁肯用功,谁就会写文章。

这么说,不就很难作到写作的跃进吗?不是!写作的跃进也和别种工作的跃进一样,必须下工夫,勤学苦练。不能把勤学苦练放在一边,而去空谈跃进。看吧,原本不敢动笔,现在拿起笔来了,这还不是跃进的劲头吗?然后,写不出大的,就写小的;写不好诗,就写散文;这样高高兴兴地,不图名不图利地往下干,一定会有成功那一天。难道这还不是跃进么?好吧,让咱们都兴高采烈地干吧!放开胆子,先有什么写什么,有多少写多少,咱们就会逐渐提高,写出像样子的东西来。不怕动笔,笔就会听咱们的话,不是吗?

\newpage

\section{茅盾:向生活学习}

\emph{按照原文表述,《怎样练习写作》可能是本书,但是在平台检索都没有相关信息。检索到本篇最初印入1944年10月重庆文风书局出版的《新少年文库》第三集。所以猜测以下内容应该是茅盾关于写作的专栏性文章。}
\vspace{2em}

\subsection{第一、不要学舌}
这本小书的题目是《怎样练习写作》。

所谓“写作”,范围应当很广,凡是叙事、抒情、议论,乃至日常应用文件,都是可以包括进去的;不过通常人们提到“写作”两字,那范围就不是这么广泛了,通常这是指文艺作品的写作。

我们当然不主张每个小朋友将来都从事于文艺工作,但是我们也觉得没有理由去禁止他们在学习写作的时候也试写些文艺性质的东西,从前有过这样的办法:读了几年书,要开始学习作文了,第一步便是“对对子”。这玩意儿,可以说是文艺性的。但学“对对子”的目的,却不是准备将来做文学家。现在这一个办法也不通行了。以前又有过另一个惯例,读书的小朋友到了能够写这么百来个字一篇的时候,先生出的作文题目便往往是《秦始皇汉武帝合论》、《性善性恶论》、《说忠》、《说孝》、《论富强张兵之道》……诸如此类的一套;这一类的大题目,放在小学生手上,居然也能用些陈腔滥调七拼八凑,完卷了事。

这样的“大题目”,也许现在也不大时行了,可是,类似的情形,却还存在,不过面目不同,现在我们的小朋友也会把一些标语口号拼凑起来,应付另一些大题目,例如《祝湘西大捷》、《论日寇必败》、《新生活运动十周纪念感想》之类。像这样的办法不是练习写作,而是练习“学舌”,练习“拼凑文字的七巧板”,练习“套用公式和教条”。这不是教导万物之灵的人类的幼小者的方式,这是调弄鹦鹉的方式。即使教者主观上并不希望得到这样的结果,但到头来这样的结果还是无可避免,为什么呢?因为小朋友们拿到了这些大题目除了“学舌”以外,实在很少办法。

当然,每个小朋友都应该知道我们在湘西打了怎样一个胜仗,都应该知道为什么日寇必败,都应该知道新生活运动是怎么一回事;但是,应当有这些知识,是一件事,而用这些知识作为题目叫他们作文,却是又一件事,测验他们有没有这些知识,并不一定是作文范围以内的事,并不一定要用作文这一个方式,因为作文的目的是练习写作,并不是默写他所已知的知识。而要练习写作,要增进写作的技巧,则最好是让他去抒写一些以生活经验得来的东西,不是要他记写耳朵听来的话。如果作文的题目限制着他们,使得他们只能把听来的话“学舌”一番,那就失却了“练习写作”的意义了。

一个天分高的小朋友拿到了《新生活运动十周纪念感想》的题目,也许会想起他的父亲或哥哥如何在参加了纪念大会以后回家来就打了一夜牌,―这是从他生活经验来的真切的感想,但是他一定不敢把这样的感想写出来,因为他觉得先生说的,报上登的,全要好看得多,冠冕得多,他如果这样写,就不合格,于是结果他只有抛弃了自身的真切的感想,而把听来的话“学舌”一番了。

在“学舌”的时候,思想不会焕发,情绪也不会热烈,甚至字句也不用自造,换言之,从头到底,只是默写,不是练习写作。然而假使先生发下来的题目不那样大,而是《我的妈妈》、《我的姊姊》、《我怎样过暑假》、《我最喜欢的事》……那么,我们的小朋友即使想贪懒学舌也有所不能了,他得动动脑筋,从他的生活经验中选择材料,他的感情也浓郁起来了,并且也不得不用心血来选字造句了。这时候,他是真正在练习写作了。

《我的妈妈》一类的题目,可说是带点文艺性的;特别是小朋友们拿到这一类题目总把它们写成一篇文艺性的东西,这是无可奈何的,因为小朋友们的观察力分析力总比成年人薄弱,小朋友们爱写的,总不外是一些给他印象最深,激动他的情感最强,而且适合他的发展中的想象力的物事;换言之,小朋友写作的兴趣是偏在于文艺性的题材,亦即是文艺性的题材比较地更能发挥他们的写作能力,增进他们的写作技巧。

但是,文艺性的题目也未可一概而论,有一些文艺性的题目也会诱起了小朋友们“学舌”的倾向,例如《美丽的春天》或者《伟大的长江》这一类的题目,便能使得大多数的小朋友又拿出应付《论日寇必败》一类题目的方法来了。春天,当然是他们经验中的东西,长江,或许他们也有若千印象,然而春天的美丽应当从哪一些地方去写,长江的伟大又应当从哪一些地方去写,这在一般小朋友便感到困惑,难以下手了;于是方便的诀门又从此生出,他们把平时在书上看来的一些陈腔滥调,什么黄莺儿在歌唱,蝴蝶在翱翔,太阳烘暖了你的心,……诸如此类的字句,都搬过来,又玩起文字的七巧板来了。这也是一种的“学舌”,这还不能算是模仿,因为模仿虽然是照了人家的样式去做,至少你是做了。

总结起来说,练习写作的首要的原则是,不要“学舌”,要说自己的话;要从生活经验中拣取对自己印象最深,激励感情最热烈而真挚的事物,用自己认为最合适的字句表达出来。把握住了这首要的原则,然后我们可以进一步谈怎样练习的具体看法。

这一章算是这本小书的引子。这是对小朋友们说的,但同时也是对小朋友们的先生们―特别是对于现在还有的好出“大题目”的风气,提出了我的看法。

\subsection{第二、美的几个条件}
怎样才算是写得好呢?练习着写作的小朋友大都会这样发问。或者,把问题归结到文艺性作品,而提出了怎样才能使一篇东西美妙。

这样的问题,本来不是几句话可以回答得了的,但在这里,又不能不试作一简单的解答:凡是文章(不一定是文艺性的,而文艺性的也包括在内),总有内容和形式这两方面,属于思想情绪者,谓之内容;属于字句篇章的构造安排者,谓之形式。打个比方,形式好似一个人的外相,内容则好似一个人的学问和品格。

一个人外貌生得漂亮而胸无点墨,俗语称之为“绣花枕头”,但即使外貌漂亮又加颇有知识,而品格卑劣,亦不为人们重视。文章也是如此,内容形式都好的才是好文章。非文艺性的文章,例如历史的、哲理的、政治的,除了内容好而外,也需要有好的形式:大历史家、大哲人和大政论家的文章都是在形式上也很完美的,这并不是他们写作的时候也曾特别注意形式的完美,而是因为他们的思想既极高超,学问又极渊博,感情又极真挚而热烈,结果他们的文字自然也就不同平凡了,只有属于应用科学的文章才不必讲究形式的美不美。

至于写作文艺性的东西,便须有意地来讲究形式上的完美了;但这不是说,有了形式上的完美便什么都好了。一个品格卑劣,未尝学问的人,即使拼命讲究外表的威仪,即使如何善于伪饰,终于不能欺骗有识者的眼睛,徒有形式的完美而内容贫乏或竟糟糕的文章,便等于是这样的人。

美也有种种不同的型:柔媚,幽雅,是美;但雄壮,豪放,也是美;匀整是美,而错综也能够是美;闲适和飘逸可以是美,但紧张热烈也可以是美;绚烂愉快和阴沉悲壮,同样能成其为美。这种种的美,都不能和思想情绪脱离关系,这样看来,内容贫乏或竟糟糕的东西,就是形式上的美也不能真正具有的,通常所谓“内容不行”,但“形式尚美”的东西,其实形式亦何尝能真美,臂如丑女子涂脂抹粉,只能欺骗近视眼,或者借灯光为掩护而已。

美既不能单从形式上求索,所以也就不能说那些字眼是美的。

美的,而另外一些字眼则不美,不能说怎样的句子的构造或篇章的布置才能够美,而别的就不成其为美。我们只能说:有几个条件是必要的,具备了这几个条件,文章就美了,否则就不美,或者不够美。

这几个条件是怎样的呢?

第一是明白通畅。把你的意思表达得清清楚楚,不折不扣,不会引起人的误解,这就是“明白”。把你的意思用浅显而平易的方式表达出来,特别是用大众所爱好、所习惯的方式表达出来,不故意卖弄才情,不弄玄虚,不搔首弄姿,这就是“通畅”。要明白,就不可以从书上去找现成的字句来配你的意思,而必须自己想出那切合你的意思的字句。要“通畅”,就不可以一味主观,坚信自己的表现方式,而应当留心观察学习大众的表现方式。

第二是感情要真挚热烈。怎样才算得是真挚热烈呢?成语有所谓“骨鲤在喉,不吐不快”,就是这句话的注脚。千万不要作干呕,干呕就等于“无病呻吟”,要不得。

第三,心地要坦白,思想要纯洁。

第四,不为写作而写作。这就是说,一不为分数,二不为出风头,三不为自己娱乐,四不为供别人消遣,五不作任何的“敲门砖”。为什么要写?因为有所思,有所感,有所见,因为我这所思、所感、所见,不仅是我个人的东西,而是和我以外的大多数人有关联的。

以上四个条件都具备了,你的文章就会叫人读了怪窼心,仿佛句句是代他说的,就能激动了读者的情绪,和你一同笑。

一同哭、一同愤怒、一同激昂,最后,跟你一同走。凡是能够激动人们的情绪到这样地步的,我们通常称之为“有力”,但“力”实在是表现在外面的现象,“力”之所从生的东西,即是“美”。“美”的感召力和激动力,是伟大到不可思议的,“美”应当这样的去解释,文章之美不美,也应当这样的去衡量。

\subsection{第三、材料和描写}
上面两章的内容都是议论居多,为什么我们要发这些议论?这本小书不是要讲怎样“练习”写作么?要回答这疑问,我们只好再发几句议论。

首先我们不要忘记,这里所谓“写作”,是指文艺性的东西,既不是历史和政论,也不是什么学术论文,尤其不是任何应用文件,如果我们是在这里讲究如何练习作“报告”,写应酬的“八行信”,草拟什么上行下行的“公文”,或者推盘受盘的“广告”,买田置产的“契约”,那自然完全不同了,那就根本用不到一点议论,一开始我们就搬出“程式”来练习就行了。然而我们这里妥讲的,是怎样练习写作文艺性的东西,因此就不能不先来一点议论(理论)。这一点议论,虽则看起来好像和实际的练习没有多大关系,可是写作时缺少了它就像行在大海里的船缺少了指南针。

不过光有指南针也不能行船,所以现在我们就要在实际练习这方面提出几点意见来。

初学者多半是性急的,他恨不能一口就吞下了他那学习对象的全部技巧,他希望有这么一套整整齐齐的规程让他记住了就万事大吉,对于文艺性的东西,抱着这样希望的也就很多。举个例:在怎样写什么什么的题目之下,总有不少热心而性急的人希望揭开书来就看见无数规程一条一条排在那里,像数学的公式似的,可是,文艺性的东西实在不能按照什么规程去写作,即使曾经有人给它们订下了若干规程,恐怕也不能像木匠那样按照着图样就能造出家具来,何况这样的规程实际上是不能有的。

因此,我们在这里要讲的,不是什么规程,更不是什么秘诀,只是几个步骤,几条原则。

先从原则方面来谈谈罢。
材料是写作者碰到的第一个问题。常常听得说:“还没找到材料”,或者“材料已经写完了”,这话其实只是随便说说,实际上,这不是有或没有的问题,而是“成熟”或“未成熟”的问题。写作的题材是一点一滴在平时累积起来的,并不是一下找就可以得到而且又很合适,而一个人的脑子也绝对不同于一间仓库,仓库有时会挤满,再也容纳不下方尺大小的东西,有时会搬空,若说人的脑子也会挤满也会搬空,那是不可思议的。一个人只要神经正常,他的脑子随时在接受外来的东西,就是随时在进材料,所以不是有或没有的问题,而是“成熟”或“未成熟”的问题,当你觉得还不够时,其实是未成熟罢了。

那么,成熟的征候是怎样的呢?征候是文思汹涌,兴奋即于不能自持,闭目默想的刹那间全篇就寂然突现于脑海,好像已经写成而且记熟了似的。

如果你落笔时文思滞涩,觉得左不是右不是,那就是没有成熟,这时最好干脆搁笔。这一搁即使是永久告别了你这一篇未完了的东西,也不可惜,在这里,也许有人问道:既然尚未成熟,过些时等它成熟起来,行不行呢?当然行的,但是也不要误解,等候材料成熟是一个简单的时间问题。这需要更多的生活经验的积累,而且需要各方面的生活经验,不单是你那未成熟的一方面的东西。

或者,又有人问道:既然未曾成熟,想来是其中缺少了什么,就按这所缺的去找去,行不行呢?我以为是不行的。

千万不要误会,一篇文艺性作品的材料只是物质性的人、物(包括自然)、事。材料感到缺少了什么的时候并不是多添了一二人物,多写一二自然风景,或多加些故事情节的问题。如果间题只这么一点,那是容易解决的。一篇文艺作品的材料,也还(而且主要的)包有思想问题:你对所写的材料的立场,你对于其中所有的问题的看法,你对于这些材料的了解的程度等等……所以,材料之成熟过程不是(比方说)物理学的而是化学的,不是单纯的量的增加,而是由量的增加达到质的变化的。

不过这一番话在小朋友看来也许太深了点,那么,请你们先记住:写作的材料是平时一点一滴累积起来的,积的时候不知不觉;如果没有积蓄,临时去找,那就不会有好的成绩。初学者又一担心的问题便是描写的方法。常常听得说:我只能直直落落叙述,却不知道如何描写,请告诉我,如何描写?曾有人投合这种要求,编出了所谓《文学描写辞典》。也居然有人希望这样的“辞典”会教给他如何描写。其实是此路不通。

说老实话,描写并无方法。而且不可能有方法。所谓“描写”,用文艺的术语来说,只是一个“形象化”的问题。何谓“形象化”?浅言之,这就是作品中的人、物、事都能直接呈现于读者眼前,而不依赖着抽象的说明。

《水浒》写李逵的性格和鲁智深的性格是同中有异而异中又有同,两个人都是鲁莽的,而又各有不同。《水浒》从这两人的言语举止、应付问题等等方面来写,结果是两个各有个性的活人呈现在我们眼前了。我们通常称这为“描写得好”,其实这是作者在形象化方面有办法。

为什么他有办法?是不是因为他懂得如何如何的方法?不是的。他之所以有办法,无非因为他观察得太深刻太周到,落笔以前他先有这么两个人物活在他心中了。如果不求自己所写的人物先活在自己心中,而痴心妄想去追求什么描写的方法,那一定是徒劳。所以千万不要担心你懂不懂描写方法,不要担心你会不会描写;应该担心的,是你心中先有了活生生的人、物、事没有?如果有了,则描写只是一个很单纯的技术问题。正像上文说过的一句话:“从人物的言语举止、应付问题等等方面来写”,再简单也没有,别无奥妙。

有两种错误的见解常常引人走入迷途。第一种是以为“描写”有赖于词藻。以为形容词用得愈多,便是描写得愈出色,往往用了一大串的形容词,弄得累赘不堪,而这些形容词又大多是前人用滥了的东西。实际上你如果有新鲜的感觉,能道前人之所未道,一个形容词也够了。第二种错误的见解是把“描写”看成装饰,以为“故事”是一篇作品的骨架,而背景描写和人物描写则仿佛是涂饰上去的油彩。这是把“描写”和整个技巧问题分离开来了,事实上描写却是整个技巧中有机的一部分。

总结起来,简单一句话:\textbf{通常被认为单纯技巧间题的描写,实在也包括在材料成熟与否的问题内;作品的材料到了成熟的境界时,描写问题是不会发生的,而成熟也者,主要又是学力修养生活经验的间题,换言之,即思想问题。}

\subsection{第四、不怕再三修改}
现在试谈写作实践时的步骤。
假如你觉得文思汹涌,下笔如有神助了,那么,我主张你就放胆一气写出来罢。此时千万不用徘徊迟疑,要相信这刹那间奔赴你笔下的字句就是最好的最妥当的。直到发泄完了为止。“推敲”二字,此时是用不到的。字句的斟酌,此时也暂且不管,想到什么,写下来再说。此时唯一目的在一口气把你脑中汹涌着的东西都移在纸上。

可是完篇以后,你就得改变你的态度。现在你要十分苛刻地来审查你自己刚才所写的东西。这时你对于自己心血的产物千万莫存丝毫姑息的意思。每段、每节、每句、每字,都不放松。毫不可惜地删掉那些不好的东西(凭你这时的判断)。不厌烦地修改了又修改。要像兑换商人辨别银币的真度一般“推敲”着每一个单字、每一个词、每一句。

在这时候,如果你发见初稿上太多要不得的段、节、句,字,你也千万莫灰心;能从自己的初稿上找出毛病来,这就证明了你的判断力是强的是健康的。倒是找不出毛病来的时候你该反省。为什么找不出毛病呢?是不是你这篇东西已经十全十美?当然不是的,自欺自满的心理最足以阻碍进步。你之所以找不出自己的毛病,正表示了你的判断力不见得高明。你得设法增进你的判断力。

但如果找出了毛病却又无力修改,那你也没有理由灰心。写作能力不是一下就能跳进几步的。一般说来,对于初稿的修改,好比是在一件制品的粗坯上加工,线条不很正直的把它修为正直,表面的粗糙加以磨光,小小的斑疤赘优加以刮剔而已;这时候,你不能因此认为这是你的写作能力有了进步,制粗坯的你和加工修光时的你,相差并不多。所以,当你找出了毛病而又修来修去总不惬意的时候,你一方面固然不得不自认已使尽了全力仍不过如此,另一方面也没有理由对自己失望。你不必忿忿然将原稿撕掉。你不如将原稿藏起来,抹一把额头的汗,暂时将这捞什子忘掉。你照常读书玩耍―或者,用一句通常话,照常生活。你如果有平素喜爱的名家作品,那就搬出来再温读也好。

这样过了一个时期(当然也不会太久),你这才再找出那篇原稿来再试试修改;这时候,也许你能够修改得惬意些了,也许原来待修改的地方仍不得惬意的改正,可是倒又发见新的需要修改的处所了,不过无论如何,这都是你的能力增进、你的批判力增强的表征;如果依然觉得无可增减的话,那倒应该自己警惕的。

不要羡慕“文不加点”的屁话。初稿写成后如果当真觉得无可改动,那要不是他的自负自满心理在作怪,便是他的批判力和写作力都停滞在已有的阶段再没有寸进了。在初学者,练习的唯一法门是多修改,一而再,再而三,永不倦怠。今天修改过了,隔几天再修改,又隔儿天再作第三次的修改。修改之后去读书,研究名著,读了书欣然有新得的时候再来修改。在原稿上一次一次新发见毛病,就表示你的能力在一天一天增长,练习写作的秘诀是不怕修改。而且当然也不要害羞,不肯将自己的初稿拿给人家看。人家的批评要用很客观的态度去听取。无论说好说坏,都不轻信,都要放在自己的理智的天平土仔细衡量。都要记下来,隔一些时再拿出来考虑。

或许有人要问道:“照这样说来,修改是无止境的了,那不是一篇东西没有完成的时候了么?

这也不然。从一个人的写作的全程上看来,修改是无止境的,亦即是学习无止境,进步无止境;这真是所谓“死而后已”的事。但就某一篇东西而言,经过多次修改以后,总会到一个应当告结束的时期。数百万字的《战争与和平》经过多年的修改,终于也脱稿了,何况薄物小篇呢!到了应当告结束的时期,不是拿出来“问世”,便是搁在书桌抽屉角,如果两者都不取,而啃住了这一篇无止境地修改下去,那不是在跟自己开玩笑,便是神经有点毛病。

那末,到了怎样的地步才算是应当告一结束呢?一是根据你那时的健全的批判力认为相当满意,二是根据你所敬佩的师友对这东西所给的意见。

\subsection{第五、自由探索,向多方面探索}
练习写作时又常常会碰到这样的事:一篇既成,修改又修改,总弄不好,此时人也累了,自信力也动摇了,但是过了些时,换个题目,换一种体裁,却居然写成了大别于昨日的作品。如果说这是一两天之内就大有进步,那显然是不合理的,于是就有别的解释,最通常的是:“这题目好写,那一个不好写”,最像有理然而实际并没说明什么的是:“那时你的灵感不来,这时它来了。”

我们最好不要把责任推给“灵感”,也不要把荣誉归于“灵感”,同时我们也不能首肯于题目有好写不好写的理论。“灵感”这玩意儿,太渺茫了,不可捉摸;如果真有所谓“灵感”的话,那么,按照“灵感”来时的精神状态来解释,这也不过是一定时间内注意力的非常集中与创作力的特别活跃而已。但如果你脑中本来空无所有,那么,所谓“灵感”者,也不能使你有所成就。因此,问题的关键仍在你的题材有没有成熟。当你写成一篇屡次修改而仍不惬意的时候,多半还是因为你的这个题材尚未成熟,而你对于此篇的体裁的一些技巧上的关节也还没有摸熟。

既然是这么一回事,因此写作的练习就不宜拘泥于一定范围的材料以及一定的体裁了。你应当把练习的范围扩展到最大的限度。因为材料有没有成熟的征候虽然大略如上文第三章所说,但有时也会被主观作用所左右而骗了你的,尽管拿笔时觉得文思汹涌,写出来的可以满不是这么一回事。因此你不必拘泥,练习的箭头要转向着四面八方。

但凡你的经验所有的,都是写作练习的对象。

但凡你所摸到过的体裁,也都是写作练习的对象。必先自由探索,向多方面探索,然后能得精熟一道的结果;必先自由发展,向多方面发展,然后能达到那适合于你的道路。

是故初学者应当活力充沛,眼光四射,在文艺的旷野纵横斥候。应当尝试写一切接触到的生活动态,尝试一切的文学体制。今天试写诗,明天试短篇小说、报告文学、散文,后天试剧本,都不要紧:只须写时严肃,不是开玩笑胡闹。不用怕人说你没有恒心;事实上你将来是否终身从事于文学工作此时也还未便决定呢!何况此时的习作,不恒又何妨?不要被“一鸣惊人”的观念所催眠,一开始就给自己划定了圈子,自认为终生园地,而死守在里边兜圈子,打算一下子完成了杰作。这不能算是认定方向专攻,这是作茧自缚。太早的自己指定一个方向去专攻,实际这是溢杀自我发展的代名词。

也许你是在每天写日记的?那么,你就把日记作为你的自由探索、自由发展的地盘。不要理那一套‘旧记”写法的方式,什么流水账似的月、日、星期几、阴或晴、冷或热、收发了什么信、会见了什么人、做了什么事―这些都是不必要的。你还不是一个大人物,这些流水账不会被人当作史料,你犯不着空费工夫。倒不如把你觉得可记的东西用各种体式来试一试。今天你送你大哥从军去了,你在日记本上可以写一首诗。明夭也许你的邻家欠了租被地主捉了去,那你不妨写篇报告文学。后天如果你的父亲和母亲为了某一件事情吵嘴了,这就给你试试写作剧本。练习写作须要多写,须要经常写,所以写日记也是一个法门;然而公式的流水账似的日记不会对你有好处,要利用写日记的精力使合于练习写作的目的,这里所举示的办法也还实际。

\subsection{第六、技巧不是神秘的东西}
我们假定阅读这本小书的人已经有了使用文字的相当技巧。请不要见了“技巧”两字,就觉得高不可攀,十分害怕。也请不要见了“技巧”两字,就联想到一长串的形容词,一些古怪的不常见的字眼,乃至一些拗口的似白话非白话的句子。所谓“技巧”,并无神秘性。你不用害羞,说:“我哪里够得上技巧。”事实上,能够把自己的意思明白说出来,就是技巧。连自己心里的意思都说不明白的,不是也常常可以遇到的么?要是又能够把自己的意思按照自己那时的情绪说的或委婉,或坚决,或洋洋然满是乐观,或低沉而悲愤,那就是技巧的程度又进一步了。只要你不上当,不迷信写在纸上的定要是书上的字眼和句法,只要你大胆把口里怎样说的写到纸上来,那你就没有理由不相信自己也相当的把握到技巧。

一定要先排除通常的对于“技巧”的神秘的看法,然后能够来谈怎样学取技巧。

凡借文字构成的文艺作品,最基本的单位是“字”。从前人讲究作文的方法,开头便讲“炼字”。这就是:为你所要表达的意思,或所要发泄的情绪,所要告人的物与事,找到那最适当最新鲜最响亮的单字。我们现在也主张“炼字”,也主张要那样去找去。不过从前人(现在也还有人)以为应当在书本上去找,我们却主张在活人的说话中去找―至少是要把这一个办法作为主要的基本的工作。在这一点上,我不反对“摆龙门阵”。

若干单字联缀起来,成为句子;所以句子的组织方法是要研究的第二步。这也是应当在人们的谈话中去找寻而研究的。你可以准备一本杂记簿,把听到的巧妙而特别的单字或句子,都随时记录下来。不过句子的组织法也还可以从语体的文学作品中去探寻。在那里,句子的组织法是经过作者加工的,因此就比通常人们谈话时更严密,更多变化。严密而多变化,这是造句技术的要点。

到此为止,“写”和“说”是一致的,“写”的技巧也就立根在“说”的技巧。再进一步,“写”就要求它特有的技巧了,然而也并不神秘。

我们试从一个实践的例子来说明这一问题。

茶馆里有人在讲故事。讲者富有口才,所以故事很动听,你把故事记录下来了,你研究,你会觉得它的精采地方,例如语言的生动而巧妙,有非写作所能及,然而比起一些好的写作的故事来,它的结构是松懈些,而情节的发展也平板些。这小小的研究,告诉我们一件事:茶馆里讲故事的那个人对于故事的技巧的一部分未尝有过研究,因为他不是有意要作一个说故事的人。

而另一方面,那些作为文艺作品写出来的故事却因作者有意地讲究这方面的技巧,所以就见得优胜。结构上的技巧是必要的。所谓结构,不仅指人与事的安排配合,还须顾到全篇的节奏―这就是从故事的发展中所产生的起伏抑扬的情调。一篇作品(除了若干例外),不能一个调子从头到底;要是这样,就成为平板,就不美。

因此须要有“波澜”,譬如一首曲子,拍子有快有慢的,音有高有低。一篇作品中的人、物、事,必须有现实的基础,然而到底是作者虚构的。正因为是虚构的,所以这些人、物、事的发生、发展和结局,必须一方面入情入理,有百分之百的真实性,而同时一方面又必须紧凑,各部分成为有机体,而且具有抑扬起伏的节奏,这样才可以增加它的色彩、律动和韵味,而强烈地感动了读者。

这些技巧,我们要到世界和本国的名著中去学习。这些技巧是经过了数百年乃至数千年的无数才人创造研究成功的。这些技巧,还在发展,绝对不会有止境。

然而这一类的技巧也不是只有从名著中方可找到,方能学习。我们也要从社会生活中去撷取创造新技巧的动力。社会是在变动的,新的社会生活会产生新的文艺上的技巧。这只要研究自古至今新的文艺形式之所以产生在特定的历史时期,就可以明白的。不过这一层说起来不大简单,这一本小书里是容纳不下的,我们只能在这里略提一笔,要详细研究,须得阅读专书。

以上所述,倘用一句常常听见的话来总结,就是:“向生活学习。”

\subsection{第七、写了再读,读了再写}
上文已就怎样练习写作一问题提出了若干意见,现在我们再加一补充,就是“多读”。“多写”,不怕修改,此外还须“多读”。

“读”与“写”应该并行,应该联系起来。懂得怎样去“读”,也就会懂得怎样去“写”。“读”的范围要广阔,要多方面。

“读”的目的,不在剽窃词句,不在摹拟作风,而在学习技巧!透彻地研究了名著的写作方法,把握住这方法,使为我用:这就尽了学习的能事。

因此,“读”须有方法。摘抄佳句不是正确的方法。如果读书只为了摘抄佳句,那就不如买一本《文艺描写辞典》。专就一章一段作繁琐的研究,也不是正确的方法。这结果是见木不见林。从作文法的立场去讲什么起承开合,不是说这一“转”如何巧妙,就是说那一“伏笔”如何关系非凡,―这并不是正确的方法。这好比变戏法,尽管别人看了惊异而五体投地,你自己除了那一点“手法”其实一无所有,要是名著的作者死而后生,听了这样的说法,大概不会承认他写的时候是有这许多小巧的计较的。

当然还有不少成问题的读法,现在不暇一一举述。

我们读一本名著,要提问题;我们要求这名著来回答我们的问题。而如何使名著能够回答,则全在我们读的方法对不对。
我们的第一个问题:作者写这部书的目的何在?他希望读者弄明白的,是什么?
第二个问题是:作者用什么方法完成池的任务?全部完成了呢,还是不全?
第三个问题是:作者所用的方法有什么特点:是不是这是唯一好的方法(在这里,我们要和其他的名著比较一番)?
第四个问题是:书中的主要人物是怎样创造成功的?第五个问题是:作者有没有创造出新的风格?

这五个问题的答案可繁可简。每一个问题的答案可以长到写成一本小册子,乃至一本书,也可以短到几十百来字。但不论长短,总之是非把全书读过几遍,完全消化了以后则是无从置答的。

善于读书和不善于读书,大有分别。不善于读书者,只在书本子外边绕圈子,结果是浮光掠影,仅得皮色。或者是一头钻进了书里,而不能出来。他迷失在字里行间,就像迷途在森林中一样。他对于一枝一叶也许辨析得系入微芒,然而他甚至连整株的树也没看清,更不必说整个的林了。善于读书者却是先钻进书里然后又出来高高地站在书之上,俯瞰着。

但是为了学习写作方法而读名著,则先钻进去而后又出来站在书之上,固然是必要的,可也还觉不足。我们又须能够把它拆开然后又装配起来,像技工们拆散了机器又装配好一样。能拆又能装,然后对于机器的微妙曲折之处算是摸熟了,我们读名著也必如此,然后真能从它学到了什么值得学的东西。

最后,我们再总结一句:练习写作,并没有什么秘奥法门,只在多写多读,不怕修改,写了以后去读,读了以后再写。

\newpage

\section{莫言:如何寻找写作灵感}

\emph{文章摘自《文艺报》2015年6月17日期。}
\vspace{2em}

三十多年前,我初学写作时,为了寻找灵感,曾经多次深夜出门,沿着河堤,迎着月光,一直往前走,一直到金鸡报晓时才回家。

少年时我胆子很小,夜晚不敢出门,白天也不敢一个人往庄稼地里钻。别的孩子能割回家很多草,我却永远割不满筐子。母亲知道我胆小,曾经多次质问我:你到底怕什么?我说我也不知道怕什么,但我就是怕。我一个人走路时总是感到后边有什么东西在跟踪我。我一个人到了庄稼地边上,总是感觉到随时都会有东西窜出来。我路过大树时,总感觉到大树上会突然跳下来什么东西。我路过坟墓时,总感觉到会有东西从里边跳出来。我看到河中的漩涡,总感觉到漩涡里隐藏着奇怪的东西……我对母亲说我的确不知道怕什么东西,但就是怕。母亲说:世界上,所有的东西都怕人!毒蛇猛兽怕人,妖魔鬼怪也怕人。因此人就没有什么好怕的了。我相信母亲说的话是对的,但我还是怕。后来我当了兵,夜里站岗时,怀里抱着一支冲锋枪,弹夹里有三十发子弹,但我还是感到怕。我一个人站在哨位上,总感到脖子后边凉飕飕的,似乎有人对着我的脖子吹气。我猛地转回身,但什么也没有。

\textbf{因为文学,我的胆子终于大了起来。}有一年在家休假时,我睡到半夜,看到月光从窗棂射进来。我穿好衣服,悄悄地出了家门,沿着胡同,爬上河堤。明月当头,村子里一片宁静,河水银光闪闪,万籁俱寂。我走出村子,进入田野。左边是河水,右边是一片片的玉米和高粱。所有的人都在睡觉,只有我一个人醒着。我突然感到占了很大的便宜。我感到这辽阔的田野,这茂盛的庄稼,包括这浩瀚的天空和灿烂的月亮都是为我准备的。我感到我很伟大。我知道我的月夜孤行是为了文学,我知道一个文学家应该是一个不同寻常的人,我知道许多文学家都曾经干过常人不敢干或者不愿意干的事,我感到我的月夜孤行已经使我与凡夫俗子拉开了距离,当然,在常人的眼里,这很荒诞也很可笑。

我抬头望月亮,低头看小草,侧耳听河水。我钻进高粱地里听高粱生长的声音。我趴在地上,感受大地的颤动,嗅泥土的气味。我感到收获很大,但也不知道到底收获了什么。

我连续几次半夜外出,拂晓回家,父母和妻子当然知道,但他们从来没有问过我什么。只是有一次,我听到母亲对我妻子说,他从小胆小,天一黑就不敢出门,现在胆子大了。

我回答过很多次文学有什么作用的问题,但一直没想起我母亲的话,现在突然忆起来,那就赶快说:如果再有人问我文学有什么功能的问题,我就会回答他:文学使人胆大。

真正的胆大,其实也不是杀人不眨眼,其实也不是视死如归,其实也不是盗窃国库时面不改色心不跳,\textbf{而是一种坚持独立思考、不随大流、不被舆论左右、敢于在良心的指引下说话、做事的精神。}

在那些个月夜里,我自然没有找到什么灵感,但我体会了找灵感的感受。当然,那些月夜里我所感受到的一切,后来都成为了我的灵感的基础。

我第一次感受到灵感的袭来,是1984年冬天我写作《透明的红萝卜》的时候。那时候我正在解放军艺术学院学习。一天早晨,在起床号没有吹响之前,我看到一片很大的萝卜地,萝卜地中间有一个草棚。红日初升,天地间一片辉煌。从太阳升起的地方,有一个身穿红衣的丰满女子走过来,她手里举着一柄鱼叉,鱼叉上叉着一个闪闪发光的、似乎还透着明的红萝卜……

这个梦境让我感到很激动。我坐下来奋笔疾书,只用了一个星期就写出了初稿。当然,仅仅一个梦境还构不成一部小说。当然,这样的梦境也不是凭空产生的。它跟我过去的生活有关,也跟我当时的生活有关。这个梦境,唤醒了我的记忆,我想起了少年时期在桥梁工地上给铁匠师傅当学徒的经历,我想起了因为拔了生产队一个红萝卜而被抓住在群众面前被批斗的沉痛往事。

写完《透明的红萝卜》不久,我从川端康成的小说《雪国》里面读到一段话:“一只壮硕的黑色秋田狗蹲在潭边的一块踏石上,久久地舔着热水。”我的眼前立即出现了一幅生动的图画:街道上白雪皑皑,路边的水潭里,热气蒸腾,黑色的大狗伸出红色的舌头,“呱唧呱唧”地舔着热水。这段话不仅仅是一幅画面,也是一个旋律,是一个调门,是一个叙事的角度,是一部小说的开头。我马上就联想到了我的高密东北乡的故事,于是就写出了:“高密东北乡原产白色、温驯的大狗,绵延数代之后,很难再见一匹纯种。”这样一段话,这就是我最有名的短篇小说《白狗秋千架》的开篇。开篇几句话,确定了整部小说的调门,接下来的写作如水流淌,仿佛一切早就写好了,只需我记录下来就可以了。

实际上,高密东北乡从来也没有什么“白色温驯的大狗”,它是川端康成的黑狗引发出的灵感的产物。

在那段时间里,我经常去书店买书。有的书写得很差,但我还是买下。我的想法是,写得再差的书里,总是能找到一个好句子的,而一个好句子,很可能就会引发灵感,由此产生一部小说。

我也曾从报纸的新闻上获得过灵感,譬如:长篇小说《天堂蒜薹之歌》,就得益于山东某县发生的真实事件;而中篇小说《红蝗》的最初灵感,则是我的一个朋友所写的一条不实新闻。

我也从偶遇的事件中获得过灵感,譬如我在地铁站看到了一个妇女为双胞胎哺乳,由此而产生了长篇小说《丰乳肥臀》的构思。我在庙宇里看到壁画上的六道轮回图,由此产生了长篇小说《生死疲劳》的主题架构。

获得灵感的方式千奇百怪,因人而异,而且是可遇而不可求。像我当年那样夜半起身到田野里去寻找灵感,基本上是傻瓜行为——此事在我的故乡至今还被人笑谈。据说有一位立志写作的小伙子学我的样子,夜半起身去寻找灵感,险些被巡夜的人当小偷抓起来——这事本身也构成一篇小说了。

灵感这东西确实存在,但无论用什么方式获得的灵感,要成为一部作品,还需要大量的工作和大量的材料。

灵感也不仅仅出现在作品的构思阶段,同样出现在写作的过程中,而这写作过程中的灵感,甚至更为重要。一个漂亮的句子,一句生动的对话,一个含意深长的细节,无不需要灵感光辉的照耀。

一部好的作品,必是被灵感之光笼罩着的作品。而一部平庸的作品,是缺少灵感的作品。我们祈求灵感来袭,就必须深入到生活里去。我们希望灵感频频降临,就要多读书多看报。我们希望灵感不断,就要像预防肥胖那样:“管住嘴,迈开腿”,从这个意义上说,夜半三更到田野里去奔跑也是不错的方法。
\newpage

\section{叶圣陶:最要紧的是锻炼语言习惯}

\emph{本篇为叶圣陶《怎样学写作》书中的第一章节《怎样写作》。}
\vspace{2em}

这一次讲的题目是《怎样写作》。怎样写作,现在有好些作文法一类的书,讲得很详细。不过写作的时候,如果要临时翻查这些书,一一按照书里说的做去,那就像一手拿着烹饪讲义一手做菜一样,未免是个笑话了。

这些书大半从现成文章里归纳出一些法则来,告诉人家怎样怎样写作是合乎法则的,也附带说明怎样怎样写作是不合乎法则的。我们有了这些知识,去看一般文章就有了一把量尺,不但知道某一篇文章好,还说得出好在什么地方,不但知道某一篇文章不好,还说得出不好在什么地方。

自然,这些知识也能影响到我们的写作习惯,可是这种影响只在有意无意之间。写文章,往往会在某些地方写得不合法则,有了作文法的知识,就会觉察到那些不合法则的地方。于是特地留心,要把它改变过来。这特地留心未必马上就有成效,或许在三次里头,两次是改变过来了,一次却依然犯了老毛病。必须从特地留心成为不待经意的习惯,才能每一次都合乎法则。所以作文法一类书对于增强我们看文章的眼力有些直接的帮助,对于增强我们写文章的腕力只有间接的帮助。所以光看看这一类书未必就能把文章写好。如果临到作文而去翻查这些书,那更是毫无实益的傻事。

诸位现在都写语体文。语体文的最高的境界就是文章同说话一样。写在纸上的一句句的文章,念起来就是口头的一句句的语言,叫人家念了听了,不但完全明白文章的意思,还能够领会到那种声调和神气,仿佛当面听那作文的人亲口说话一般。要达到这个境界,不能专在文字方面做功夫,最要紧的还在锻炼语言习惯。因为语言好比物体的本身,文章好比给物体留下一个影像的照片,物体本身完整而有式样,拍成的照片当然完整而有式样。语言周妥而没有毛病,按照语言写下来的文章当然也周妥而没有毛病了。所以锻炼语言习惯是寻到根源去的办法。

不过有一句应当声明,语言习惯是本来要锻炼的。一个人生活在人群中间,随时随地都有说话的必要,如果语言习惯上有了缺点,也就是生活技能上有了缺点,那是非常吃亏的。把语言习惯锻炼得良好,至少就有了一种重要的生活技能。对于作文,这又是一种最可靠的根源。我们怎能不努力锻炼呢?
现在小学里有说话的科目,又有演讲会、辩论会等的组织,中学里,演讲会和辩论会也常常举行。这些都是锻炼语言习惯的。参加这种集会,仔细听人家说的话,往往会发现以下的几种情形。

说了半句话,缩住了,另外换一句来说,和刚才的半句话并没有关系,这是一种。“然而”“然而”一连串,“那么”“那么”一大堆,照理用一个就够了,因为要延长时间,等待着想下面的话,才说了那么许多,这是一种。应当“然而”的地方不“然而”,应当“那么”的地方不“那么”,只因为这些地方似乎需要一个词,可是想不好该用什么词,无可奈何,就随便拉一个来凑数,这是一种。

有一些话听去很不顺耳,仔细辨辨,原来里头有几个词用得不妥当,不然就是多用了或者少用了几个词,这又是一种。这样说话的人,他平时的语言习惯一定不很好,而且极不留心去锻炼,所以在演讲会、辩论会里就把弱点表露出来了。若叫他写文章,他自然按照自己的语言习惯写,那就一定比他的口头语言更难使人明白。因为说话有面部的表情和身体的姿势作为帮助,语言虽然差一点,还可以使人家大体明白。写成文章,面部的表情和身体的姿势是写不进去的,让人家看见的只是支离破碎前不搭后的一些文句,岂不叫人糊涂?

我由于职务上的关系,有机会读到许多中学生的文章,其中有非常出色的,也有不通的,所谓不通,就是除了材料不健全不妥当以外,还犯了前面说的几种毛病,语言习惯上的毛病。这些同学如果平时留心锻炼语言习惯,写起文章来就可以减少一些不通。加上经验方面的洗练,使写作材料健全而妥当,那就完全通了。所谓“通”原来不是什么高不可攀的境界。

锻炼语言习惯要有恒心,随时随地当一件事做,正像矫正坐立的姿势一样,要随时随地坐得正立得正才可以养成坐得正立得正的习惯。我们要要求自己,无论何时不说一句不完整的话,说一句话一定要表达出一个意思,使人家听了都能够明白;无论何时不把一个不很了解的词硬用在语言里,也不把一个不很适当的词强凑在语言里。

我们还要要求自己,无论何时不乱用一个连词,不多用或者少用一个助词。说一句话,一定要在应当“然而”的地方才“然而”,应当“那么”的地方才“那么”,需要“吗”的地方不缺少“吗”,不需要“了”的地方不无谓地“了”。这样锻炼好像很浅近、很可笑,实在是基本的,不可少的。家长对于孩子,小学教师对于小学生,就应该教他们,督促他们,做这样的锻炼。可惜有些家长和小学教师没有留意到这一层,或者留意到而没有收到相当的成效。

我们要养成语言这个重要的生活技能,就只得自己来留意。留意了相当时间之后,就能取得锻炼的成效。不过要测验成效怎样,从极简短的像“我正在看书”“他吃过饭了”这些单句上是看不出来的。我们不妨试说五分钟连续的话,看这一番话里能够不能够每句都符合自己提出的要求。如果能够了,锻炼就已经收了成效。到这地步,作起文来就不觉得费事了,口头该怎样说的笔下就怎样写,把无形的语言写下来成为有形的文章,只要是会写字的人,谁又不会做呢?依据的是没有毛病的语言,文章也就不会不通了。

听人家的语言,读人家的文章,对于锻炼语言习惯也有帮助。只是要特地留意,如果只大概了解了人家的意思就算数,对于锻炼我们的语言就不会有什么帮助了。必须特地留意人家怎样用词,怎样表达意思,留意考察怎样把一篇长长的语言顺次地说下去。这样,就能得到有用的资料,人家的长处我们可以汲取,人家的短处我们可以避免。

写语体文只是十几年来的事。好些文章,哪怕是有名的文章家写的,都还不纯粹是口头的语言。写语体文的技术还没有练到极纯熟的地步。不少人为了省事起见,往往凑进一些文言的调子和语汇去,成为一种不尴不尬的文体。刚才说过,语体文的最高境界就是文章同说话一样。所以这种不尴不尬的文体只能认为过渡时期的产物,不能认为十分完善的标准范本。这一点认清楚了,才可以不受现在文章的坏影响。但是这些文章也有长处,当然应该模仿;至于不很纯粹的短处,就努力避免。如果全国中学生都向这方面用功夫,不但自己的语言习惯可以锻炼得非常好,还可以把语体文的文体加速地推进到纯粹的境界。

从前的入学做文章都注重诵读,往往说,只要把几十篇文章读得烂熟,自然而然就能够下笔成文了。这个话好像含有神秘性,说穿了道理也很平常,原来这就是锻炼语言习惯的意思。文言不同于口头语言,非但好多词不同,一部分语句组织也不同。要学不同于口头语言的文言,除了学这种特殊的语言习惯以外,没有别的方法。而诵读就是学这种特殊的语言习惯的一种锻炼。所以前人从诵读学做文章的方法是不错的。

诸位若要作文言,也应该从熟读文言入手。不过我以为诸位实在没有作文言的必要。说语体浅文言深,先习语体,后习文言,正是由浅入深,这种说法也没有道理。文章的浅深该从内容和技术来决定,不在乎文体的是语体还是文言。况且我们既是现代人,要表达我们的思想情感,在口头既然用现代的语言,在笔下当然用按照口头语言写下来的语体。能写语体,已经有了最便利的工具,为什么还要去学一种不切实用的文言?

若说升学考试或者其他考试,出的国文题目往往有限用文言的,不得不事前预备,这实在由于主持考试的人太不明白。希望他们通达起来,再不要做这种故意同学生为难而毫没有实际意义的事。而在这种事还没有绝迹以前,诸位为升学计,为通过其他考试计,就只得分出一部分工夫来,勉力去学作文言。

以上说了许多话,无非说明要写通顺的文章,最要紧的是锻炼语言习惯。因为文章就是语言的记录,二者本是同一的东西。可是还得进一步,还不能不知道文章和语言两样的地方。

前面说过,说话有面部的表情和身体的姿势作为帮助,但是文章没有这样的帮助,这就是两样的地方。写文章得特别留意,怎样适当地写才可以不靠这种帮助而同样可以使人家明白。两样的地方还有一些。如两个人闲谈,往往天南地北,结尾和开头竟可以毫不相关。就是正式讨论一个问题,商量一件事情,有时也会在中间加入一段插话,像藤蔓一样爬开去,完全离开了本题。直到一个人省悟了,说:“我们还是谈正经话吧。”这才一刀截断,重又回到本题。做文章不能这样。文章大部分是预备给人家看的,小部分是留给自己将来查考的,每一篇都有一个中心,没有中心就没有写作的必要。所以写作只该把有关中心的话写进去,而且要配列得周妥,使中心显露出来。

那些漫无限制的随意话,像藤蔓一样爬开去的枝节话,都该剔除得干干净净,不让它浪费我们的笔墨。又如用语言讲述一件事情,往往啰啰唆唆,细大不捐;传述一场对话,更是照样述说,甲说什么,乙说什么,甲又说什么,乙又说什么。做文章不能这样。文章为求写作和阅读双方的省事,最要讲究经济。一篇文章,把紧要的话都漏掉,没有显露出什么中心来,这算不得经济。必须把紧要的话都写进去,此外再没有一句啰唆的话。正像善于用钱的人一样,不该省钱的地方绝不妄省一个钱,不该费钱的地方绝不枉费一个钱,这才够得上称为经济。

叙述一件事情,得注意详略。对于事情的经过不做同等分量的叙述,必须叫人家详细明白的部分不惜费许多笔墨,不必叫人家详细明白的部分就一笔带过。如果记人家的对话,就得注意选择。对于人家的语言不做照单全收的记载,足以显示其人的思想、识见、性情等的才入选,否则无妨丢开。又如说话往往用本土的方言以及本土语言的特殊调子。做文章不能这样。文章得让大家懂,得预备给各地的人看,应当用各地通行的语汇和语调。本土的语汇和语调必须淘汰,才可以不发生隔阂的弊病。以上说的是文章和语言两样的地方。知道了这几层,也就知道作文技术的大概。由知识渐渐成为习惯,作起文来就有记录语言的便利而没有死板地记录语言的缺点了。

现在来一个结束。怎样写作呢?最要紧的是锻炼我们的语言习惯。语言习惯好,写的文章就通顺了。其次要辨明白文章和语言两样的地方,辨得明白,能知能行,写的文章就不但通顺,而且是完整而无可指摘的了。

\newpage

\section{莫言:写作源自模仿}

\emph{本文为莫言在北京十一中学演讲文稿。}
\vspace{2em}

现在,我讲讲阅读和创作之间的关系,我觉得,创作最好的老师就是阅读。如果说文学创作或者小说创作有什么诀窍的话,那就是阅读——然后就拥有了建立在阅读基础上的“魔法”。

刚才在休息室里,一名同学问了我有关读书的问题。我认为,对年轻人来讲,对任何人来讲,应该掌握两种阅读方法。

一种是精读,就像我读我大哥的语文课本一样,翻来覆去地读,读到能够把其中的主要内容背下来。

另一种是广泛浏览。世界上的读物浩如烟海,一个人即便是从有阅读能力时开始读,一直读到白发苍苍,也读不完其中的百万分之一,你只能读非常少的一部分。在这种情况下,把阅读分成精读和广泛浏览就非常重要了,你不能总是把一本书很认真地从头读到尾。

经典的书要认真地读,要精读;对于一般的读物,尤其是现在网络上出现的很多东西,一目十行地浏览一下,了解一下大概的文风,知道说了什么,也就可以了。

有了精读和广泛浏览的基础,假如你要从事文学创作的话,就应该从模仿开始。

当然,模仿,对于一个成熟的作家来讲,是个不光彩的词。如果现在还有人问:“莫言,你最近的作品模仿了谁的小说?”我认为这是我的一个巨大耻辱,别人也会对我嗤之以鼻,瞧不起我——都写了20多年小说了,新作竟然还在模仿别人!我觉得这是一个作家最大的耻辱,说明你没有什么创造历程,你一直在靠模仿生存。

但对于一个初学写作的人来讲,模仿不是耻辱,而是一个捷径,或者说是一个窍门。我当年在学校里给大学生讲课的时候,也曾经反复说过,大家不要以为模仿就是见不得人的事情,刚开始写作时谁都在模仿,包括鲁迅,他的早期作品也都有模仿的痕迹,《狂人日记》就是模仿果戈理\footnote{尼古莱·瓦西里耶维奇·果戈理·亚诺夫斯基(英语:Nikolai Vasilievich Gogol-Anovskii,1809年4月1日—1852年3月4日),笔名果戈理(英译Gogol),俄国批判主义作家,代表作有《死魂灵》和《钦差大臣》。尽管果戈理对社会程序进行了严厉的批判,但他还是认为国家体系是不能毁坏的。人们经常把果戈理的作品看作是对社会秩序的完全否定和对国家制度的深刻批判。}的同名小说。

鲁迅的很多作品,研究者都可以找出模仿的原作来,但这并不妨碍鲁迅成为伟大的文学家,因为他很快就超越了模仿的阶段,慢慢形成了自己的文风,形成了独特的鲁迅文体。

我为什么要反复强调刚开始应该模仿,就在于我觉得模仿是培养语感的最重要的方法。我们也经常批评一些作家,说“你这么大年纪了,怎么还有一股学生腔调”。“学生腔调”就是指一个人还没有形成自己的语言风格。

一个人的语言风格实际上跟一个人对语言的感受力密不可分。当年李希贵校长在山东省高密县第一中学担任校长的时候,我们讨论过这个问题,认为初中阶段对培养一个人一生的语感至关重要。如果你在初中阶段没有培养起对语言的感受力,那么你以后的努力很可能事倍功半。

假如我们在初中阶段就掌握了很好的语感,这就像一个从事音乐工作的人掌握了很好的乐感一样——培养出了懂音乐的耳朵,会为你今后的音乐工作打下最坚实的基础。假如我们在小学、初中阶段就培养出了非常敏锐的语言感受力,那么它对你将来无论是否从事文字方面的工作,都会非常有用。

培养语感最重要的方法,我觉得就是在反复阅读基础之上的模仿。

举一个简单的例子。如果我们连续让学生读10篇鲁迅的著名杂文,然后让学生写一篇类似题目的杂文,我们就会发现,几乎每一个孩子的杂文里边都出现了一种鲁迅的笔调,出现了一种鲁迅的腔调。也就是说,鲁迅的文风潜移默化地影响了每一个学生的写作风格。

这种模仿实际上是不自觉的,是建立在认真、大量阅读同一作家的作品基础之上的。假如我们连续精读古今中外10位作家的作品,然后有意识地模仿这10位作家,那么在这个过程中,你自己的东西就会慢慢出来。就像一个有志于学习书法的人,临摹了颜真卿、王羲之,又临摹了柳公权,他临摹了很多碑帖,但他最后写出的作品与他先前的作品相比,还是有细微的差别。
写作确实有点类似于书法,你模仿得多了,自己的风格也就慢慢确定了。更准确地说,是你得到了一种语感。得到一种语感对于文学创作非常重要。

如果我要写一个人内心非常痛苦,这个内心非常痛苦的人走到长安街上,用他的眼睛来看周围的事物,用他的各种感官——他的嗅觉、他的视觉、他的听觉来感受长安街。因为他内心痛苦,这个时候他写出的文字或者说作家写出的文字,必然带着一种痛苦、低沉的调子。

反过来,如果这个人是兴高采烈的,他还走在这条街道上,因为作家现在想努力表现一个人兴高采烈的状态,他的全部感觉都用在表现这种情绪上。有了这种语言的感觉之后,写出来的文字自然也就带上了一种兴高采烈的感觉,我想这就是语感。

音乐也类似。我们当年在农村生活的时候,很多农村的二胡演奏者,他们并不懂简谱,更不懂五线谱,他们是文盲,一个字都不认识,但是他们照样可以拿起琴来演奏一首非常婉转动听的乐曲。这就是说,他们在长期的模仿过程中使自己的耳朵、手指与乐器之间产生了感觉。

这种感觉我是亲身体验过的。在我十几岁的时候,我父亲说我什么本事都没有,“你看看街上那个瞎子,拉着琴可以去讨饭,可以要来很多粮食,我们家挂着一个二胡,你练练吧!”当时我只会拉一些很简单的革命歌曲,像《东方红》《大海航行靠舵手》,刚开始就是闭着眼睛瞎拉。

刚拉两下,我母亲就说:“不要拉了,明天的小米已经够喝了!”农村有一种石碾,推碾子碾米的时候,碾子会发出吱吱的声音,所以我母亲讽刺我拉二胡的声音就像推着石碾在转圈碾米一样。

“碾小米”大概“碾”了有两三个月,我就能拉出《东方红》了。那时,我脑子里一直想着《东方红》的旋律,同时手在弦上摸来摸去,摸了两三个月,我的手和耳朵以及《东方红》的旋律之间就建立起一种联系。也就是说,我的手已经有乐感了。

后来,我听到什么曲子,只要记住那个旋律,就可以拉出来了。所以,我就明白我们的民间音乐家为什么可以一个字不识,根本不懂任何乐曲,也可以拉出他心里的旋律。

有很多民间的天才音乐家,像阿炳\footnote{阿炳(1893年8月17日—1950年12月4日),原名华彦钧,出生于江苏省无锡市,中国内地民间男音乐家、正一派道士,人称“瞎子阿炳”。}那样的人,他是个盲人而且是个文盲,为什么能创作出像《二泉映月》这样经典的民族音乐?因为他已经超越了模仿别人旋律的阶段。他心中巨大的痛苦无法表述,在心里自然生成了一种悲苦的旋律,表现在他的手和琴弦上,就成了经典乐曲。

我想,文学创作的过程,文学创作过程当中的语言、语感,与音乐家、音乐演奏家们的创作和演奏过程当中的乐感是一个道理,即建立在这种多读和进行模仿写作的基础之上。

我本来想详细地讲一下我的创作,但没有多少时间了,所以,我只简单地讲一下。我的创作也是分了几个阶段。20世纪70年代末,我在部队的时候就开始学习创作了。这个时期主要是模仿,而且模仿得很拙劣。

到了20世纪80年代初期,我开始发表作品,这个阶段也还停留在模仿的阶段。刚才有同学提到的《春夜雨霏霏》,就是模仿茨威格\footnote{斯蒂芬·茨威格(Stefan Zweig,1881年11月28日—1942年2月22日),奥地利小说家、诗人、剧作家、传记作家。茨威格出身富裕犹太家庭,青年时代在维也纳和柏林攻读哲学和文学,日后周游世界,结交罗曼·罗兰和弗洛伊德等人并深受影响。创作诗、小说、戏剧、文论、传记,以传记和小说成就最为著称。第一次世界大战期间从事反战工作,1934年遭纳粹驱逐,流亡英国和巴西。1942年2月22日在巴西自杀。}的《一个陌生女人的来信》。后来,我还写过一篇叫《售棉大道》的小说,模仿了阿根廷作家科塔萨尔的作品《南方高速公路》。还有一篇叫《民间音乐》的小说,模仿了美国的一位女作家麦卡勒斯的作品《伤心咖啡馆之歌》。

那个阶段我还是在模仿,而且很多编辑也一眼就看出来了,问:“这个书是模仿了谁的小说吧?”我说,确实是,但他们还是决定发表,因为这里边已经出现了我自己的东西。第一,表现的是中国的内容;第二,语言有自己的特色,有很多高密的乡言土语,而且融合得很好。当然,模仿的痕迹还是存在的。

我真正走出模仿的阶段,逐渐形成了自己的文风是在1984年,那是我到解放军艺术学院进修以后。

我的成名小说应该是《透明的红萝卜》。这部小说所描述的内容跟我个人的经验有很大的关系。我曾经在一个桥梁工地上为一个铁匠师傅做过小工,所以我对打铁的生活非常熟悉。我描写深更半夜的时候,在秋风萧瑟的桥洞里面,一个铁匠,一个赤着上身只穿一条短裤的孩子,拉着风箱,看着熊熊燃烧的炉火……那种想象,那种很奇妙的感受,都跟我的个人经验分不开。


那个时候,我的小说不仅内容上中国化了,而且小说所使用的语言也个性化了。所以,我觉得一个作家成熟的最重要的标志,就是形成了他自己的文风,让人一眼就可以看出来。就像我们读鲁迅的文章,即便把鲁迅的名字盖住,我们依然可以读出来,这是鲁迅的语言风格。张爱玲的小说,也有她自己独特的语言风格。沈从文的小说也有他自己独特的语言风格。

也就是说,只有当一个作家形成了自己的语言风格,当他的语言为丰富作品做出贡献的时候,我们才认为他已经超越了一个小说家或者说小说匠的阶段,他可以说是一个文学家了。

因此,文学家跟小说家是有区别的,小说家成群结队,文学家寥寥无几。所以,在现当代小说家的队伍当中,能够称得上文学家的也就是鲁迅、沈从文、张爱玲等屈指可数的几个人。直到现在,莫言依然是一个小说家,只是一个说书人而已。也许再奋斗20年,终我这一生也很难上升到文学家这个阶段。今后的岁月尽管前途渺茫,但我还要努力奋斗!

\newpage

\section{王安忆:写作与个人经历的距离}

\emph{本文摘自王安忆新作《小说六讲》,在讲稿中,王安忆结合自身经验,以讲故事的方式带出阅读与写作的机要。}
\vspace{2em}

\subsection{文学创作的开始}
80年代,我到中国作家协会举办的第五届文学讲习所学习,参加学习的学员都是已经非常著名的作家,包括张抗抗、贾平凹等。当时贾平凹已是成熟的作家,就没有来,名额给了另一个也是写作经验成熟的作家,他也没来,于是文学讲习所多了一个名额。宿舍是四人一间房,但只有三名女生,所以这个名额就指定是女生。讲习所最后决定把这个名额优惠给上海,因上海这个大城市只有一名学员,就是竹林,当时已经写了长篇《生活的路》,影响很大。

这个名额落到上海少年儿童出版社,说明当时年轻作者都是儿童文学出身。出版社推荐了三个女孩子,我是其中之一。文学讲习所特别强调是给写作者提供文学补习,所以不建议高校学生参加讲习所,这是一个补救的方法,给没有机会受教育的青年补一课。上海推荐的那两个女孩子其时都在读大学,所以这名额就给了我。我只写了《谁是未来的中队长》,还有几篇谁都没看过的散文,可是机会落到我头上,至今想起来还是觉得幸运。尤其是后来又多出一个名额,就近落在北京,来的是一名女生,我们又搬进一间五人宿舍。老师们都说悬得很,要是她比我先到,就没有我的事了。

我在文学讲习所学习期间,发表了我的第一篇成人小说,名叫《雨,沙沙沙》。《雨,沙沙沙》以现在的文学归类概念,可算是青春小说,故事讲述一个名叫雯雯的女孩子,经历了插队落户回到城市,和我经历非常接近。她面临爱情问题,选择怎样的爱人和生活,这是很普遍的青春问题。她向往爱情和未来,不知道要什么,只知道不要什么。然后,在一个雨天遭遇一个偶然的邂逅,于是模糊的向往呈现出轮廓,就是“雨,沙沙沙”。开始的时候,人们很容易觉得我是因为母亲的关系才得到学习的名额——我母亲是60年代崛起的作家,她的名字叫茹志鹃,代表作《百合花》几十年都收在中学语文课本——所以对我别有看法。《雨,沙沙沙》这篇小说出来,大家都感到耳目一新。

80年代的时候,写作还延续着长期形成的一种公式,题材和母题,都是在公认的价值体系中。以此观念看,《雨,沙沙沙》就显得暧昧了,这个女孩的问题似乎游离于整个社会思潮之外,非常有个人性,所以大家都觉得新奇。那个时代社会刚从封闭中走出来。现在许多理所当然的常识,当时却要经过怀疑、思考、理论和实践才能得到,叫作“突破禁区”。今天的常识,就是那些年突破一个又一个禁区得到的。当时有个同学说《雨,沙沙沙》像日本的私小说。我们那时候根本不懂得什么是私小说,后来才知道是类型小说的一种,写个人私密生活。我非常欢迎同学给我的小说这么命名,对当时以公共思想为主题的意识形态来说,“私”这个字的出现,是带有革命性的。

《雨,沙沙沙》是我走上写作道路的标志,主角雯雯就像是我的化身,一个怀着青春困惑的女性,面临各种各样的生活难题和挑战。她对社会没有太大的承担,对时代也不发一言,她只面向内在的自我。这小说刚出来时引起大家的关注,因为那时的小说潮流是以《乔厂长上任记》《在小河那边》为主体,承担着历史现实批判、未来中国想象的任务,有着宏大的叙事风气。我这个带有私小说色彩的小人物出现,一方面大家觉得她很可爱,另一方面又觉得她和中国主流文化、话语系统不一样,也有点生疑。总之,引起了关注。

就这样,我虚构的这一个在文学主流之外的女孩子“雯雯”,忽然受到众多评论家的注意。有一个著名的评论家叫曾镇南,当时谁能够得到他的评论都是不得了的。他写了一篇评论,并发表在重要的评论杂志《读书》上,题目叫《秀出于林》。后来又有上海的年轻评论者程德培,写了第二篇,这篇评论文章的题目直接就叫《雯雯的情绪天地》。我觉得他这篇文章的命名有两点很重要,一个是“雯雯”这个人物,一个是“情绪”两个字,意味着一种内向型的写作。事情的开端很引人注目,可是接下去就不好办了,因为我的生活经验很简单,不够用于我这样积极大量的写作。外部的经验比较单薄,我就走向内部,就是评论家程德培所说的“雯雯的情绪天地”,我就写情绪,可没有经验的支持,内部生活也会变得贫乏。

我的生活经验在我们那一代人之中是最浅最平凡的。像莫言,他经历过剧烈的人生跌宕起伏,从乡村到军队再到城市,生活面很广。而我基本上是并行线的:没有完整的校园生活;有短暂的农村插队落户经历,作为知青,又难以真正认识农村;在一个地区级歌舞团,总共六年,未及积累起人生经验又回到上海城市;再到《儿童时代》做编辑,编辑的工作多少有些悬浮于实体性的生活;再接着写作,就只能够消费经验,而不能收获。有时候我听同辈那些作家,尤其来自农村的,他们讲自己的故事时,我都羡慕得不得了,怎么会那么有色彩,那么传奇,那么有故事?城市的生活是很没有色彩的,空间和时间都是间离的。我虽然有过两年的农村生活,可是因为苦闷和怨愤,农村的生活在我看来是非常灰暗的,毫无意趣可言。回想起来,其实我是糟蹋了自己的经验。

记得我在农村时,母亲写信给我,说我应该写日记,好好注意周围的人和事,可以使生活变得有乐趣,可我只顾沉浸在自己的情绪里,都没有心思去理会其他。这是一个大损失,我忽略了生活,仅只这一点可怜的社会经验,也被屏蔽了,这时候,便发现写作材料严重匮乏。等到把雯雯的故事写完,我好像把自己的小情小绪都掏尽了,就面临着不知道写什么好的感觉,可写作的欲望已经被鼓舞起来,特别强烈,写什么呢?就试图写一些离自己人生有距离的故事。

\subsection{写作与个人经历的距离}
开始写与自己人生经历有点距离的故事,我的文学创作似乎又继续顺利地滑行,取得了一些奖项、好评和注意。其中有获得全国奖的短篇小说《本次列车终点》。《本次列车终点》讲述青年陈信终于完成夙愿,从乡下回到上海建立新生活,却发现上海并非想象中那么完美,在上海生活并不容易。他努力争取回到一直想念的上海,以为可以将断裂的生活接续上来,可是那个断裂处横亘在他的人生里,使他失去归宿感。表面上看,这好像是一个和我有距离的故事,因为我写的是一个男性,他的生活状态和我也不太一样,但回头再看,这故事还是有我的个人经验。我离开八年再回到上海,以为一切皆好,事实上却感到失落。你以为你还能在这城市找回原来失去的东西,但时间流走了,失去的依然失去,你再也找不到,就像刻舟求剑,你再也找不到你的剑了。就是这么一个心情,还是和我个人有关系。

当时我确实在努力寻找一些和我有距离的故事,企图扩大自己的题材面,但从某个角度来说,我还是在自己的经验范围里。小说里的男主角“陈信”不是我,又是我,他一定是和我靠得最近的人,如果我不理解他,不同情他,那为什么要去写他呢?同时,他又和我存在着距离,这距离可让我看得清楚。

举个也许不恰当的例子:中国著名京剧大师、男旦梅兰芳,他是一个男性,身在其外,懂得女性要怎样才有吸引力,所以演得比女人还像女人。可能有时候作者必须与小说里的人物保持一些距离,如果没有距离,就看不清楚他,或者会过于同情和沉醉,那就变成一种自赏自恋。所以说作者与小说人物的关系是非常复杂的,一方面你要和他痛痒相关,另一方面又要对他有清醒的认识。

当我写《本次列车终点》的时候,题材上已经落后了。我写的是知青生活,可是从时间上来说,我已经错过了知青文学这班车,知青文学浪潮已经过去。80年代真是不得了,时间急骤地进行,先是伤痕文学,然后是知青文学、“右派”文学,然后又是反思文学,波涛迭起,后浪推前浪。知青文学早已经遥遥领先,壮烈激情,感天动地。时间上说,已经是在尾声。内容上,且不在批判的大趋势里,而是好像有点反动,写一个知青终于回到城市,面对新生活的困顿,怀念起旧生活,而这恰巧是知青文学所控诉的对象,于是又不能纳入知青文学思潮的主流。所以评论者给我定位时也感觉蛮犹豫的,他们把我定到知青文学里,因为我是知青的身份,但最安全是把我定在女性作者,这是肯定不会有误的。

另一篇得到全国奖的是中篇小说《流逝》。《流逝》写的不是我个人的经验,是我邻居家的故事。从这点来说,就和我也有关系。故事写一个资产者家庭的女性,在“文革”时经历了非常艰苦的生活,由昔日的少奶奶变成持家的主妇。“文革”结束,拨乱反正,财产失而复得,家庭秩序回复常态,但她在艰困生活中的主动性和价值感却消失殆尽,又回到传统中的附属地位。这故事虽不是我个人的经验,但也包含了我的一些心情:我们都经历了艰苦的岁月,如果那些岁月不给你留下一点遗产的话,你的人生不是白费了吗?写这小说时,我以为那是我经验以外的故事,等到成熟以后回过头看,故事的情绪还是和自己的经验有点关系。

如果你们将来要写小说,要注意一个事实,新人一定会得到好多好评的,大部分人对新人是很宽容的,会对你说很多好听的话。但当过了新人阶段后,你会得到不同的评价,这段时间一定要冷静。我的作品得到更多人注意后,对我的批评也开始多起来,这些批评可能更客观,标准也更高。无论你能接受还是不能接受,它都是在帮助你,帮助你形成你的认识论和方法论。批评说我好的地方是从主观世界走进了客观世界,说不好的也是这个,认为我放弃了自我。当时确实也很苦恼,你真的不晓得应该怎么做才好,但可以写作的欲望是这样强烈,无论多么茫然,还是要写下去。

\subsection{生活经验:重要的是内心}
现在的年轻人没有经历过匮乏的年代,尤其是大陆的孩子,多是独生子女,没有经历过争夺的日子。生活在富裕的社会,他们的生活确实是比较顺利。顺利的生活带来的却是平淡,缺乏丰富性。但是真正来衡量一个人的生活是不是丰富,恐怕更取决于心理经验。

普鲁斯特\footnote{马塞尔·普鲁斯特(Marcel Proust,1871—1922)是20世纪法国最伟大的小说家之一,意识流文学的先驱与大师,也是20世纪世界文学史上最伟大的小说家之一。}一个人躺在床上,生活是优渥的,不需要谋求衣食,一天到晚沉浸于冥想。他向我们证明,冥想的能量同样足够促成伟大的小说家。当然,普鲁斯特肯定写不出像莫言的作品,莫言拥有着极其丰富的外部生活,多姿多彩。莫言的乡村里有无数农人,经历着同样的人和事,可是能成为莫言的就只有他一个。因此,莫言之所以为莫言不只是取决于他的外部生活,更取决于内部,也就是冥想。我们身边存在着很多事物,每个人都有反应,反应的差异决定你是什么样的人。世间的生活,大体上差不多,在彼此相像的经验底下一定是存在着差异,这就要看个人体察的能力,如何发现事物,又如何表现事物。

哲学家和作家是相反的。哲学家可以在很多不同的东西里发现相同的东西,但是作家,则是在看似相同的东西里发现个别性。一般人眼睛里彼此很相似的事情,在作家眼里,却会不相像。还有的情形是,很多事情在当时当地来看并不觉得怎么样。过去以后,当你经历过更多的人生,有了认识,再回过头看,才发现独特性。所以,作家又是一种总是在回顾里生活的人。因为,我们写的任何东西其实都已经发生过了,都是过去式。

我曾在澳门大学郑裕彤书院做工作坊,同学们交上来的作业差不多都是写校园生活的,大家笔下的校园生活所见略同,不外乎爱情、友谊、师生关系、宿舍起居。这些人和事近在当下,来不及拉开距离审视。我说能否谈谈你们从小生活的地方,当他们谈到从小生活的地方,谈到他们的父母,气氛就变得活跃起来了,特殊性开始呈现起来。这些遥远的事物,其实是带有起源性质的,它潜藏在表面相似的经历里,使共同的生活分化。当然,今天的生活确是越来越走向同质化,这是一个困扰全世界作家的问题,所以作家笔下最常出现的就是主流以外的人。流浪汉、精神病患者,或者是性取向异常的人,为什么这类人会成为作者笔下的故事呢,因为主流生活已经格式化,唯有往主流外面的边缘地带去寻找艺术的对象。但这只是一个策略,本质性的还在于内心。

让经验释放更大的价值
几十年的写作实践,我可自称是一个职业作家,“我为什么写作”这个问题常常浮现在我心里,最初的答案已经不够解释了,新的又是什么?经过很长的时间,终于有一句话来回答媒体,回答评论者,也回答我自己,就是“我要创造,我渴望创造”。我渴望创造的是我在现实里无法实现的一种生活,无法兑现、仿佛是乌有之存在,但在某种程度上又和我的生活有关系,如果没有在现实生活中积累起的情感的容量,我不可能产生创造另一种存在的欲望。

好比我现在是一个木匠,我造一张桌子,我用的材质是木头,但这木头不是块死木头,它是由一棵树长成的,这棵树是有生命的,由巨大的生命力促成的一个占位。我这个木匠为了某一种自私的需要,很残酷地把这棵树给斫下来,造成一张桌子,这张桌子则在空间里形成一个全新形态的占位,另一个从无到有。

在某种意义上来说,我们这些写作者也有残忍的一面。我们可以把自己活生生的经验割裂下来送出去,有时候割裂下来送出去的东西还不见得有价值,它的价值还抵不上那经验本身,可是渴望创造的欲望很强,就顾不上这些了,哪怕把我活生生的经验变成一段死木头,样子难看极了,我也得去做。我曾经在浙江乌镇参观一间床博物馆,里面陈列着很多木床,造得非常华丽,有的就像一间小房子,有几进,第一进是起居,第二进是盥洗,第三进最里面,才是卧床。床架帐屏顶棚布满雕镂,有花卉、虫鱼、鸟兽,还有各种仙俗故事,最壮阔的是一整部三国。最使我感到有意思的不仅是它工艺的繁复,而是木匠造床的规矩,不收工钱,收红包,一个大红包。原因是木匠给人造床是要折寿的,造棺材则是积德,这是个奇怪的理论,一定有着现代人不能解的伦理。造床的木匠是要留名的,木匠会做一块精致的木牌,刻上自己的名字,非常具有仪式感。

我想这确是一件需要隆重对待的事情,一棵活生生的树造成一张床,让它在新造型里复活。要造一张好好的床才对得起这棵树,就如我们要写一部好好的小说才对得起我们经验的生活和感情。因此,我在不断地认识我的经验,寻找更好的方式表达,使我阅历过的时间在另一种时间里释放出更大的价值。


\newpage

\section{余华:我叙述中的障碍物}

\emph{本篇摘自余华杂文集《我只知道人是什么》。}
\vspace{2em}

这次的演讲题目很明确,我想把自己创作中的经验告诉大家,可能对你们没有用,因为每个人都不一样,对我有用的经验可能对你们没有用,我选择这个题目就是要把自己写作过程中遇到过一个个障碍物告诉你们。

第一个障碍物是如何坐下来写作,这个好像很简单,其实不容易。我去过的一些地方,这些年少了,过去多一些,总会有一些学生或者年轻人问我怎样才能成为一个作家,我说只有一个字—写,除此以外没有别的方法。写就像是人生里的经历,没有经历就构不成你的人生,不去写的话不会拥有你的作品。

我记得写第一篇小说的时候,是短篇小说,我都不知道分行怎么分,标点符号该怎么点,因为我小学一年级到高中毕业刚好是文革,所以刚写小说的时候我根本不知道该怎么写,就拿起一本文学杂志,打开来随便找了一个短篇小说研究,什么时候分行,什么地方用什么标点符号,我第一次学习的短篇小说分行很多,语言也比较简洁,我就这样学下来,刚开始很艰难,坐在书桌前的时候,脑子里什么都没有,逼着自己写下来,必须往下写,这对任何一个想成为作家的人是第一个障碍。我要写一万字,还要写的更长,而且要写的有内容。好在写作的过程对写作者会有酬谢,我记得第一篇小说写的乌七八糟,不知道写什么,但是自我感觉里面有几句话写的特别好,我竟然能写出这么牛的句子来,很得意,对自己有信心了,这就是写作对我的酬谢,这篇小说没有发表,手稿也不知道去哪里了。

然后写第二篇,里面好像有故事了。再写第三篇,不仅有故事,还有人物了,很幸运这第三篇发表了。

我胃口很大,首先是寄《人民文学》和《收获》,退回来以后把他们的信封翻一个面,用胶水粘一下,剪掉一个角,寄给《北京文学》和《上海文学》,又退回来后,就寄到省一级的文学杂志,再退回来,再寄到地区级文学杂志,我当时手稿走过的城市比我后来去过的还要多。

当时我们家有一个院子,邮递员骑车过来把退稿从围墙外面扔进来,只要听到很响的声音就知道退稿来了,连我父亲都知道。有时候如果飞进来像雪花一样飘扬的薄薄的信,我父亲就说这次有希望。我1983年发表小说,两年以后,1985年再去几家文学杂志的编辑部时,发现已经没有这样的机会了,自由投稿拆都不拆就塞进麻袋让收废品的拉走,成名作家或者已经发表过作品的作家黑压压一大片,光这些作家的新作已让文学杂志的版面不够用了,这时候编辑们不需要寻找自由来稿,编发一下自己联系的作家的作品就够了。所以我很幸运,假如我晚两年写小说,现在我还在拔牙,这就是命运。

对我来说,坐下来写作很重要,这是第一个障碍物,越过去了就是一条新的道路,没越过去只能原地踏步。总是有人问我怎样才能成为作家,我说首先要让你的屁股和椅子建立起友谊来,你要坐下来,能够长时间坐在那里。我的这个友谊费了很大劲才建立起来,那时候我还年轻,窗外阳光明媚,鸟儿在飞翔,外面说笑声从窗外飘进来,引诱我出去,当时空气也好,不像现在。我很难长时间坐在那里,还是要坚持坐下去,这是我写作遇到的第一个障碍。

第二个障碍是在我作品不断发表以后,那时候小有名气了,发表作品没问题了,可是写作还在继续,写作中的问题还在继续出现,比较突出的问题就是如何写好对话。写好对话可以说是衡量作家是否成熟的一个标准,当然只是很多标准中的一个,但是很重要。比如我们读一些小说,有时会发现,某个作家描写一个老农民,老农民神态,老农民生活的环境都很准确,可是老农民一开口说话,不是老农民的腔调,是大学教授的腔调,这就是问题,什么人说什么话是写小说的基本要素。

当我还不能像现在这样驾驭对话的时候,采取的办法让应该是对话部分的用叙述的方式去完成,有一些对话自己觉得很好,胸有成竹,再用引号标出来,大部分应该是对话来完成的都让叙述去完成。我那时发现苏童处理对话很有技巧,他的不少小说通篇是用叙述完成的,人物对话时没有引号,将对话和叙述混为一谈,既是叙述也是对话,读起来很舒服,这是他的风格,我学不会,我要找到自己的方法。

我是在写长篇小说时解决了这个问题,自然而然就解决了。可能是篇幅长的原因,写作时间也长,笔下的人物与我相处也久,开始感到人物有他们自己的声音,这是写作对我的又一次酬谢,我就在他们的声音指引下去写对话,然后发现自己跨过对话的门槛了,先是《在细雨中呼喊》,人物开始出现自己的声音,我有些惊奇,我尊重他们的声音,结果证明我做对了。接着是《活着》,一个没有什么文化的老农民讲述自己的故事,这个写作过程让我跨过了更高的门槛。然后是通篇对话的《许三观卖血记》了。
我年轻时读过詹姆斯·乔伊斯的《一个青年艺术家的画像》,通篇用对话完成的一部小说,当时就有一个愿望,将来要是有机会,我也要写一部通篇用对话完成的长篇小说,用对话来完成一个短篇小说不算困难,但是完成一部长篇小说就不容易了,如果能够做到,我觉得是一个很大的成就。我开始写小说的时候,对不同风格的小说都有兴趣,都想去尝试一下,有的当时就尝试了,有的作为一个愿望留在心里,将来有机会时再去尝试,这是我年轻时的抱负。

1995年我开始写《许三观卖血记》,写了一万多字后,突然发现这个小说开头是由对话组成的,机会来了,我可以用对话的方式来完成这部小说了,当然中间会有一些叙述的部分,我可以很简洁很短地去处理。写作《许三观卖血记》的时候,我意识到通篇对话的长篇小说的障碍在什么地方,这是当年我读《一个青年艺术家的画像》的时候感受不到的困难,詹姆斯·乔伊斯的困难。当一部长篇小说是以对话来完成时,这样的对话和其他以叙述为主的小说的对话是不一样的,区别在于这样的对话有双重功能,一个是人物在发言,另一个是叙述在推进。所以写对话的时候一定要有叙述中的节奏感和旋律感,如何让对话部分和叙述部分融为一体,简单的说如何让对话成为叙述,又让叙述成为对话。

所以我在写对话时经常会写得长一点,经常会多加几个字,让人物说话时呈现出节奏和旋律来,这样就能保持阅读的流畅感,一方面是人物的对话,另一方面是叙述在推进。

写完《许三观卖血记》以后,对于写对话我不再担心了,想写就写,不想写可以不写,不再像过去那样小心翼翼用叙述的方式去完成大部分的对话,留下一两句话用引号标出来,不再是这样的方式,我想写就写,而且我知道对话怎么写,什么人说什么话,这个在写完《在细雨中呼喊》和《活着》之后就没问题了,写完《许三观卖血记》后更自信了。写作会不断遇到障碍,同时写作又是水到渠成,这话什么意思呢,就是说障碍在前面的时候你会觉得它很强大,当你不是躲开而是迎上去,一步跨过去之后,突然发现障碍并不强大,只是纸老虎,充满勇气的作者总是向前面障碍物前进,常常是不知不觉就跨过去了,跨过去以后才意识到,还会惊讶这么轻松就过去了。

接下去说说我叙述里的第三个障碍物,这个很重要,对于在座的以后从事写作的人也许会有帮助。我说的是心理描写,对我来说这是最大的障碍。当我写了一些短篇小说,又写了一些中篇小说,开始写长篇小说的时候,也就是我小说越写越长,所写的内容越来越丰富复杂的时候,我发现心理描写是横在前面的一道鸿沟,很难跨越过去。为什么?当一个人物的内心是平静的话,这样的内心是可以描写的,可是没有必要去描写的,没有价值。当一个人物的内心兵荒马乱的时候,是很值得去描写,可是又不知道如何去描写,用再多的语言也无法把那种百感交集表达出来。当一个人物狂喜或者极度悲伤极度惊恐之时,或者遇到什么重大事件的时候,他的心理是什么状态,必须要表现出来,这是不能回避的。当然很多作家在回避,所以为什么有些作家的作品让我们觉得叙述没有问题,语言也很美,可是总在绕来绕去,一到应该冲过去的地方就绕开,很多作家遇到障碍物就绕开,这样的作家大概占了90\%以上,只有极少数的作家迎着障碍物上,还有的作家给自己制造障碍物,跨过了障碍以后往往会出现了不起的篇章。当时心理描写对我来说是很大的障碍,我不知道该怎么写,每次写到那个地方的时候就停下笔,不知道怎么办,那时候还年轻,如果不解决心理描写这个难题,人物也好,故事也好,都达不到我想要的那种叙述的强度。

这时候我读到了威廉·福克纳\footnote{威廉·福克纳(William Faulkner 1897年9月25日—1962年7月6日),美国文学史上最具影响力的作家之一,意识流文学在美国的代表人物,1949年诺贝尔文学奖得主,获奖原因为“因为他对当代美国小说做出了强有力的和艺术上无与伦比的贡献”。福克纳一生共写了19部长篇小说与120多篇短篇小说,最有代表性的作品是《喧哗与骚动》。}的一个短篇小说叫《沃许》,威廉·福克纳是继川端康成和卡夫卡\footnote{弗兰兹·卡夫卡(Franz Kafka,1883年7月3日—1924年6月3日),奥匈帝国捷克德语小说家,本职为保险业职员。主要作品有小说《审判》《城堡》《变形记》等。卡夫卡1883年出生于犹太商人家庭,他生活在奥匈帝国即将崩溃的时代,又深受尼采、柏格森哲学影响,对政治事件也一直抱旁观态度,故其作品大都用变形荒诞的形象和象征直觉的手法,表现被充满敌意的社会环境所包围的孤立、绝望的个人。卡夫卡与法国作家马赛尔·普鲁斯特,爱尔兰作家詹姆斯·乔伊斯并称为西方现代主义文学的先驱和大师。}之后,我的第三个老师。《沃许》写一个穷白人如何把一个富白人杀了,一个杀人者杀了人以后,他的内心应该是很激烈的,好在这是短篇小说,长篇小说你没法去研究,看了前面忘了后面,看了后面忘了前面,短篇小说还是可以去研究,去分析的。

我惊讶地读到福克纳用了近一页纸来描写刚刚杀完人的杀人者的心理,我当时就明白了,威廉·福克纳的方式很简单,当心理描写应该出现的时候,他所做的是让人物的心脏停止跳动,让人物的眼睛睁开,全部是视觉,杀人者麻木地看着躺在地上的尸体,还有血在阳光下的泥土里流淌,他刚刚生完孩子的女儿对他感到厌恶,以及外面的马又是怎么样,他用非常麻木的方式通过杀人者的眼睛呈现出来,当时我感到威廉·福克纳把杀人者的内心状态表现得极其到位。但是我还不敢确定心理描写是不是应该就是这样,我再去读记忆里的一部心理描写的巨著,陀思妥耶夫斯基的《罪与罚》,重新读了一遍,有些部分读了几遍。拉斯科尔尼科夫把老太太杀死以后内心的惊恐,陀思妥耶夫斯基大概写了好几页,我忘了多少页,没有一句是心理描写,全是人物的各种动作来表达他的惊恐,刚刚躺下,立刻跳起来,感觉自己的袖管上可能有血迹,一看没有,再躺下,接着又跳起来,又感觉到什么地方出了问题。

他杀人以后害怕被人发现的恐惧,一个一个的细节罗列出来,没有一句称得上是心理描写。还有司汤达的《红与黑》,当时我觉得这也是一部心理描写的巨著,于连和德-瑞纳夫人,还有他们之间的那种情感,重读以后发现没有那种所谓的心理描写。然后我知道了,心理描写是知识分子虚构出来的,来吓唬我们这些写小说的,害得我走了很长一段弯路。

这是我在80年代写作时遇到最大的障碍,也是最后的障碍。这个障碍跨过去以后,写作对我来说就变得不是那么困难,我感觉到任何障碍都不可能再阻挡我了,剩下的就是一步一步往前走,就是如何去寻找叙述上更加准确、更加传神的表达方式,把想要表达的充分表现出来。

当然叙述中的障碍物还有很多,在我过去的写作中不断出现过,在我将来的写作中还会出现,以后要是有时间的话可以写一本书,那是比较具体的例子,今天就不再多说。

最后我再说一下,就是障碍物对一个小说家叙述的重要性,伟大的作家永远不会绕开障碍物,甚至给自己制造障碍物,我们过去有一句话“有条件要上,没有条件创造条件也要上。”伟大作家经常是有障碍要上,没有障碍创造障碍也要上……司汤达把一场勾引写得跟一场战争一样激烈,这是一个伟大的作家,别的作家不会这样去处理,但是伟大的作家都是这样处理。所以我们读到过的伟大的文学篇章,都是作家跨过了很大的障碍以后出来的。托尔斯泰对安娜·卡列尼娜最后自杀时候的描写,可以说是文学史上激动人心的篇章,托尔斯泰即使简单地写下安娜·卡列尼娜的自杀情景也可以,因为叙述已经来到了结尾,前面的几百页已经无与伦比,最后弱一些也可以接受,但是托尔斯泰不会那么做,如果他那么做了,也不会写出前面几百页的精彩,所以他在结尾的时候把安娜·卡列尼娜人生最后时刻的点点滴滴都描写出来了,绝不回避任何一个细节,而且每一个句子每一个段落都是极其精确有力。

20世纪也有不少这样的作家,比如前不久去世的马尔克斯\footnote{加夫列尔·加西亚·马尔克斯(Gabriel José de la Concordia García Márquez,1927年3月6日—2014年4月17日),又名贾西亚·马奎斯,出生于哥伦比亚,毕业于波哥大大学,哥伦比亚作家、记者和社会活动家,拉丁美洲魔幻现实主义文学的代表人物。1982年诺贝尔文学奖得主。其小说在表现手法上是魔幻现实主义的,“将真事隐去”,用魔幻的、离奇的、现实生活中不存在的事物和现象反映、体现、暗示现实生活。代表作有《百年孤独》《霍乱时期的爱情》。},你在他在叙述里读不到任何回避的迹象。《百年孤独》显示了他对时间处理的卓越能力,你感觉有时候一生就是一天,一百年用20多万字就解决掉,这是非常了不起的。马尔克斯去世时,有记者问我,他与巴尔扎克、托尔斯泰有什么区别,我告诉那位记者,托尔斯泰从容不迫的叙述看似宁静实质气势磅礴而且深入人心,这是别人不能跟他比的。我听了巴赫的《马太受难曲》以后,一直在寻找,文学作品中是不是也有这样的作品,那么的宁静,那么的无边无际,同时又那么的深入人心。后来我重读《安娜·卡列尼娜》,感觉这是文学里的《马太受难曲》,虽然题材不一样,音乐和小说也不一样,但是叙述的力量,那种用宁静又广阔无边的方式表现出来的力量是一样的,所以我说这是托尔斯泰的唯一。巴尔扎克有一些荒诞的小说,也有现实主义的小说,你看他对人物的刻划丝丝入扣,感觉他对笔下人物的刻划像雕刻一样,是一刀一刀刻出来的,极其精确,而且栩栩如生。我对那个记者说,从这个意义来说,所有伟大作家都是唯一的,马尔克斯对时间的处理是唯一的,我还没有读到哪部作品对时间的处理能够和《百年孤独》比肩,所以他们都有自己的唯一,才能成为一代又一代读者不断去阅读的经典作家。

当然唯一的作家很多,仅仅俄罗斯文学就可以列出不少名字,托尔斯泰、陀思陀耶夫斯基、果戈理、契可夫,就是苏联时期还有帕斯捷尔纳克、布尔加科夫,肖洛霍夫,肖洛霍夫的《静静的顿河》我读了两遍,四卷本的书读了两遍,这是什么样的吸引力。当年这本书在美国出版时因为太厚,兰登书屋先出了第一和第二卷的合集,叫《顿河在静静流》,出版后很成功,又出版了第三和第四卷的合集,叫《顿河还在静静流》。虽然这部小说里有不少缺陷,尽管如此,仍然无法抵销这部作品的伟大,那些都是小毛病,可以忽略的小毛病。这部小说结束时故事还没有结束,我觉得他在没有结束的地方结尾了很了不起,我读完后难过了很多天,一直在想以后怎么样了?真是顿河还在静静流。

\rightline{2014年5月6日 北京}

\newpage

\section{贾樟柯:写作是一个爬坡的过程}

\emph{贾樟柯是著名导演,这也是他最被大众熟悉的一个身份,但阅读过他文字的朋友应该可以感受到那种独特的纪实表达的魅力。这也是把这篇选进来的原因。可以看看他是怎么理解写作的。}
\vspace{2em}


\emph{学生提问(上海温哥华电影学院的院长信箱栏目):}

\emph{总是会觉得自己写的故事不够好,‍‍在这样的心态下容易失去每天写作的动力,请问如何在日常生活中持续保持创作的动力?}

\vspace{2em}

不一定要每天写作。我接触很多好的作家,他们都跟上班一样,特别是写长篇小说,他们就要求上午9点开始,每天是6小时还是8小时写作。

但是很多作家也跟我说,他每天坐在那,有时候一天写几千字,有时候一个字也写不出来。但是重要的是要坐在那,‍‍重要的是连续的。并不是单纯说连续地坐在那,每天都能出来让自己兴奋的作品,而是说工作不能三天打鱼两天晒网。

一方面要对写作规律有一个了解:它确实不是每天‍‍都像打石油一样不停地往外喷涌,能量那么足,‍‍有时候确实就卡壳了。但是如果离开你的书桌或者离开你的思考,一旦放下,可能‍‍一个月两个月就过去了。

如果你就像爬坡一样,每天去爬,‍‍可能第2天没有什么进展,第3天微微前进了一步,但到了第4天很可能就越过困难了。‍‍所以我觉得,写作它确实是情感波动很大的‍‍一个工作。顺利时不知疲倦,不顺利时,甚至会怀疑自己的能力。这都是非常正常的,写作就是这个样子。

所以当你写作‍‍不顺利的时候,‍‍要告诉自己,这其实是一个爬坡的过程。我给大家讲一个例子,是我的一个感受:在现场拍电影‍‍,有时候你灵感特别多,觉得拍摄异常顺利,但那可能只是你的惯性处理。因为你拍的感觉很舒服很爽,所以一直往下拍,总以为拍得好像很好,而最终的呈现效果反而是不好的。

‍‍但是那些拍摄不顺利,就是你觉得怎么处理都处理不好,不是这出问题就是那出问题的拍摄,‍‍反而这一场戏或者片段,最终的呈现效果一定是最好的。
因为,那个过程是你在寻求新的方法。

‍‍所以写作也一样。当你不顺利的时候,一定是你找到了新的东西,即将看到新的风景。这是‍‍一个转折,要有耐心来渡过这样一个很短暂的时间‍。甚至写剧本,‍‍比如说你写到哪卡住了,觉得怎么也写不好,你可以试试再写另外一个故事。因为你会发现有一天,突然就明白那个地方应该怎么写了。


\newpage

\section{迟子建:关于写作的十二则体会}

\emph{本文曾在人民文学出版社公众号上发布。}
\vspace{2em}

1.作家要善于取材,更要善于掌握“火候”,这个火候,需要作家有全面素养,比如看待历史的广度、看待现实的深度、对美的追求等。当然,更重要的是一个作家精神上的孤寂,他们对待艺术独立的姿态,身上有一股不怕被潮流忽略和遗忘的勇气,这样能使每一次的出发都是独特的。

2.没有描写苦难,诗意怎会呈现?温暖也是一样,没有冷作为底衬,没有用笔化解寒凉,它从何而来?如果作品一味地展览苦难,却没有希望的微光闪烁,这样的苦难就是真的苦难了,而如果苦难里有柔软的光影浮动,苦难就不是深渊,它会散发着湿漉漉的动人的光泽。所以我很喜欢弘一法师临终手书的“悲欣交集”,它道出了人生的真相,也道出了艺术的真谛。

3.一个作家除了尊重史实,还要建构你自己的精神世界,也就是描摹鼠疫中平常百姓的悲欢离合。因为资料上的人都是死的,你要用想象把这些人复活。而没有坚实的史料做依托,没有一颗沧桑而温暖的心去揣摩和贴近人物,他们又怎么复活得起来呢。

4.如果说诗意是艺术的话,那么小说家当然不能放弃对诗意的追求。在这里我要特别强调,我从来没有,将来也不会在作品中回避苦难;我也从来没有,将来也不会在作品中放弃诗意。苦难中的诗意,在我眼里是文学的王冠。

5.一部小说的好坏,很大程度取决于语言的成色。小说语言如果没有个性,缺乏表现力,就成了“说明文”,不管故事多么新奇,小说的魅力将大打折扣。如今有些小说尽管故事不错,但是语言粗糙平淡,缺乏光彩,你就喜欢不起来。

6.我热爱世俗生活,安然过着自己的小日子。平淡,朴素,但不潦草。如果因为写作就活得潦草,那是傻瓜。写小说是蛮有情调的事,如果一个作家的生活过得无滋无味,那么写出来的东西也必然是乏味的。

7.一个作家能否走到底,拼的不是拥有什么样的生活,占有什么样的素材,而是精神世界的韧性、广度和深度。

8.一个作家写人性永远是不错的,因为人性是最复杂的。

9.我的体悟,就是我们不要把个人的痛苦放大。一定要想到众生的这种苦难,那么你的作品会获得一种升华、一种沉淀。从文学意义上、艺术意义上,这种沉淀就是一种艺术上的飞翔,这是特别重要的。一定要经过长时间的历练,很多东西在一个瞬间把你唤醒,才能和艺术融合。

10.写作的人不会孤独,你周围有那么多笔下的人物陪着你呢。我特别能理解费雯丽演电影演得精神失常,演员很容易把角色的身世遭遇放到自己身上。我从一开始就没有想到写作会是万众瞩目的事业,它不过,是给自己找了一个默默的伴侣。

11.我想一个作家的感情是质朴的,他的写作才会浮现质朴的风貌。不是由于你写了土壤,你就质朴了;也不会因为你写了旧上海,你就不是质朴的。说到底,一个作家的气质,决定了他作品的气质。

12.如果碰到瓶颈期,也没什么不好。瓶颈是妖娆的障碍啊,能从它颈下爬出来,必定会脱胎换骨的。作家假如有勇气面对有难度的写作的话,就不要怕遭受瓶颈。

\newpage

\section{刘震云:写作是有近路可抄的}

\emph{本文摘自刘震云相关访谈、文章等。}
\vspace{2em}

1.作品人物的一点看法和认知,跟作者的认识认知是两回事,并不一定作品人物的认识就是作者的认识。最好的作者其实是离作品人物越远越好,我说的远是作者的影子退得越远越好。当然有很多作品作者很强势,他的作品就是作品人物的思想,而且他要教导这个人物,包括读者应该怎么做,当然还有一种作者,他跟作品里的人物只是朋友,他们在交谈。

2.文学特别重要的作用,是唯一能够把生活中不同层面的乱象码放清楚的工具。

3.文学能够用情节、细节、对话,更重要的是情节、细节、对话之外的那些弦外之音,言外之意,那些趣味、那些氛围能够一点点给它丝丝缕缕地码放清楚。它确实存在跟现实生活不一样的东西,但是它更接近现实。

4.写作是有近路可抄的,是可以“投机”的:什么叫投机性写作?第一,你选择的体裁要极端。第二,写法要极端。凡是极端,别人没用过的,你可以轻易区别、超过别人。第三,更大的投机,你可以写得谁都看不懂。

5.写作是一个极古老的职业,就像钉鞋的、做杂碎汤的职业一样古老。选择写作为生,是因为我喜欢,它给我带来很多乐趣。每一个人对世界都是懂得少,不懂的多。我用这样的方式来探索这个世界上不懂的东西。

6.与现实太像的文学作品没有存在的价值,人们直接到现实中看就好了。而特别好的文学作品是,你要写的是这个阶层,但它一定跟这个阶层的公众看法是不一样的。

7.我不是把蚂蚁写成蚂蚁,而是写蚂蚁怎么翻了几个跟头变成了大象,我写的是中国人思维演变的过程,能看到社会、政治架构的演变,但并不是要揭露黑暗、反腐倡廉。别人写的官场一定是“黑暗”,我写的一定是“温情”“温暖”,贪污腐败的背后是多么的温暖。如果你愤怒,你就相信它了。

8.用男性或女性的视角来写你的作品都非常狭隘,因为世界本就是有两个方面。一定有一个比男性或女性更高的角度,民族的差异等可能会在某些方面被放大,但人性方面是共通的。

9.情节与细节的荒诞是以一种严肃的状态、表情在运作,作家的想象力就是把这些情节和细节组成一个波澜壮阔、震撼人心的长篇故事,作家结构出来的这个虚构故事应该比生活更接近真实和本质。这是作家的任务。更重要的是,小说中的认识一定跟生活中的认识是不一样的,甚至是完全相反的,这是小说存在的价值。

10.一个作家真正的功力不在有形的小说,而是后面无形的东西,这是一个层面。另一个层面,具体到一个作品里面,作家真正的功力包括呈现出的力量,不管是荒诞还是什么,它不在你的文字表面。功夫在诗外。这就是结构的力量,这个结构力量特别考验作家的胸怀,这个胸怀就是你能看多长看多宽,你对生活的认识、对人性的认识、对文学的认识以及对自己的认识。

11.写作,不是要写自己懂的那部分,懂得的就不用写了,写的是自己不懂的那一部分,试图通过写作能够接近那个不懂。不懂,是我写作最大的动力。

12.什么叫写作?生活停止的地方,写作开始了。我们在生活中并没有那么深入的、情感的表达,特别是对那些日常生活中被忽略的人。李雪莲想说一句话,全世界有谁知道呢?当全世界的人不听她说话时,我是一头牛,是李雪莲在牛棚里说话的第二头牛,我来听她说话。写书的根本意义,是我替“李雪莲们”把话说出来了。

13.在“不同”里找出“不同”那是很正常的;但是在“相同”里面找出“不同”,是很不容易的。你写一个人和一朵花好,就比写一个人和另一个人好容易。这是《聊斋志异》和《红楼梦》的区别。

14.一个人对生活的态度、对朋友的态度、对每一个细节的态度,能反映出他的胸怀和见识,而这些一定会带到他的作品里。如果在生活中不懂得尊重朋友、尊重生活、尊重每一个细节,他在作品中就很难尊重他作品中的人物。所以我觉得,一个人在生活中的态度和见识,是衡量他是不是一个好作者、好导演特别重要的标志。

15.大多数中国作家还在与历史和灵魂对话时,我愿意坚持为现实中的小民呐喊。

16.写作是个世界上最容易的事儿,如果你喜欢它。

\newpage

\section{汪曾祺:认识到的和没有认识的自己}

\emph{本文写于1988年8月16日,刊载于1989年第一期《北京文学》,收录于汪曾祺《汪曾祺的写作课》(第25章)。}
\vspace{2em}

作家需要评论家。作家需要认识自己。“文章千古事,得失寸心知。”但是一个作家对自己为什么写,写了什么,怎么写的,往往不是那么自觉的。经过评论家的点破,才会更清楚。作家认识自己,有几宗好处。一是可以增加自信,我还是写了一点东西的。二是可以比较清醒,知道自己吃几碗干饭,可以心平气和,安分守己,不去和人抢行情,争座位。更重要的,认识自己是为了超越自己,开拓自己,突破自己。我应该还能搞出一点新东西,不能就是这样,磨道里的驴,老围着一个圈子转。认识自己,是为了寻找还没有认识的自己。

我大概算是一个现实主义的作家。现实主义,本来是简单明了的,就是真实地写自己所看到的生活。后来不知道怎么搞得复杂起来了。大概是苏联提出了社会主义现实主义。而将以前的现实主义的前面加了一个“批判的”。“批判的现实主义”总是不那样好就是了。什么是“社会主义的现实主义”呢?越说越糊涂。本来“社会主义”是一个政治的概念,“现实主义”是文学的概念,怎么能搅在一起呢?

什么样的作品是“社会主义现实主义”的呢?标准的作品大概是《金星英雄》。中国也曾经提过社会主义现实主义,后来又修改成革命的现实主义和革命的浪漫主义相结合,叫做“两结合”。怎么结合?我在当了右派分子下放劳动期间,忽然悟通了。有一位老作家说了一句话:有没有浪漫主义是个立场问题。我琢磨了一下,是这么一个理儿。你不能写你看到的那样的生活,不能照那样写,你得“浪漫主义”起来,就是写得比实际生活更美一些,更理想一些。我是真诚地相信这条真理的,而且很高兴地认为这是我下乡劳动、思想改造的收获。我在结束劳动后所写的几篇小说:《羊舍一夕》、《看水》、《王全》,以及后来写的《寂寞和温暖》,都有这种“浪漫主义”的痕迹。

什么是“革命的现实主义和革命的浪漫主义相结合”?咋“结合”?典型的作品,就是“样板戏”。理论则是“主题先行”、“三突出”。从“两结合”到“主题先行”、“三突出”是历史发展的必然。“主题先行”、“三突出”不是有样板戏之后才有的。“十七年”的不少作品就有这个东西,而其滥觞实为“社会主义现实主义”。我是在样板团工作过的,比较知道一点什么叫“两结合”,什么是某些人所说的“浪漫主义”,那就是不说真话,专说假话,甚至无中生有,胡编乱造。我们曾按江青的要求写一个内蒙草原的戏,四下内蒙,作了调查访问,结果是“老虎闻鼻烟,没有那八宗事”。我们回来向于会泳作了汇报,说没有那样的生活,于会泳答复说:“没有那样的生活更好,你们可以海阔天空。”

物极必反。我干了十年样板戏,实在干不下去了。不是有了什么觉悟,而是无米之炊,巧妇难为。没有生活,写不出来,这是最简单不过的事。样板戏实在是把中国文学带上了一条绝径。从某一方面说,这也是好事。十年浩劫,使很多人对一系列问题不得不进行比较彻底的反思,包括四十多年来文学的得失。“四人帮”倒台后,我真是松了一口气。我可以按照自己的方法写作了。我可以不说假话,我怎么想的,就怎么写。《异秉》、《受戒》、《大淖记事》等几篇东西就是在摆脱长期的捆绑的情况下写出来的。从这几篇小说里可以感觉出我的鸢飞鱼跃似的快乐。

我写的小说的人和事大都是有一点影子的。有的小说,熟人看了,知道这写的是谁。当然不会一点不走样,总得有些想象和虚构。没有想象和虚构,不成其为文学。纪晓岚是反对小说中加入想象和虚构的。他以为小说里所写的必须是亲眼所见,亲耳所闻:小说既述见闻,即属叙事,不比戏场关目,随意装点。

他很不赞成蒲松龄,他说:今燕昵之词,狎之态,细微曲折,摹绘如生。使出自言,似无此理,使出作者代言,则何从而闻见之。

蒲松龄的确喜欢写狎之态,而且写得很细微曲折,写多了,令人生厌。但是把这些燕昵之词、狎之态都去了,《聊斋》就剩不下多少东西了。这位纪老先生真是一个迂夫子,那样的忠于见闻,还有什么小说呢?因此他的《阅微草堂笔记》实在没有多大看头。不知道鲁迅为什么对此书评价甚高,以为“叙述复雍容淡雅,天趣盎然”。

想象和虚构的来源,还是生活。一是生活的积累,二是长时期的对生活的思考。接触生活,具有偶然性。我写作的题材几乎都是可遇而不可求的。一个作家发现生活里的某种现象,有所触动,感到其中的某种意义,便会储存在记忆里,可以作为想象的种子。我很同意一位法国心理学家的话:所谓想象,其实不过是记忆的重现与复合。完全没有见过的东西,是无从凭空想象的。其次,更重要的是对生活的思索,长期的,断断续续的思索。井淘三遍吃好水。生活的意义不是一次淘得清的。我有些作品在记忆里存放三四十年。好几篇作品都是一再重写过的。《求雨》的孩子是我在昆明街头亲见的,当时就很感动。他们敲着小锣小鼓所唱的求雨歌:

\centerline{\emph{小小儿童哭哀哀,}}
\centerline{\emph{撒下秧苗不得栽。}}
\centerline{\emph{巴望老天下大雨,}}
\centerline{\emph{乌风暴雨一起来。}}

这不是任何一个作家所能编造得出来的。我曾经写过一篇很短的东西,一篇散文诗,记录了我的感受。前几年我把它改写成一篇小说,加了一个人物,望儿。这样就更具体地表现了中国农村的孩子从小就知道稼穑的艰难,他们用小小的心参与了农田作务,休戚相关。中国的农民从小就是农民,小农民。《职业》原来只写了一个卖椒盐饼子西洋糕的,这个孩子我是非常熟悉的。我改写了几次,始终不满意。到第四次,我才想起先写了文林街上六七种叫卖声音,把“椒盐饼子西洋糕”放在这样背景前面,这样就更苍凉地使人感到人世多苦辛,而对这个孩子过早的失去自由,被职业所固定,感到更大的不平。思索,不是抽象的思索,而是带着对生活的全部感悟,对生活的一角隅、一片段反复审视,从而发现更深邃,更广阔的意义。思索,始终离不开生活。

我是一个极其平常的人。我没有什么深奥独特的思想。年轻时读书很杂。大学时读过尼采、叔本华。我比较喜欢叔本华。后来读过一点萨特,赶时髦而已。我读过一点子部书,有一阵对庄子很迷。但是我感兴趣的是其文章,不是他的思想。我读书总是这样,随意浏览,对于文章,较易吸收;对于内容,不大理会。我大概受儒家思想影响比较大。一个中国人或多或少,总会接受一点儒家的影响。我觉得孔子是个很有人情的人,从《论语》里可以看到一个很有性格的活生生的人。孔子编选了一部《诗经》(删诗),究竟是为了什么?我不认为“国风”和治国平天下有什么关系。编选了这样一部民歌总集,为后代留下这样多的优美的抒情诗,是非常值得感谢的。“国风”到现在依然存在很大的影响,包括它的真纯的感情和回环往复,一唱三叹的形式。《诗经》对许多中国人的性格,产生很广泛的、潜在的作用。“温柔敦厚,诗之教也。”我就是在这样的诗教里长大的。我很奇怪,为什么论孔子的学者从来不把孔子和《诗经》联系起来。

我的小说写的都是普通人,平常事。因为我对这些人事熟悉。顿觉眼前生意满,须知世上苦人多。

我对笔下的人物是充满同情的。我的小说有一些是写市民层的,我从小生活在一条街道上,接触的便是这些小人物。但是我并不鄙薄他们,我从他们身上发现一些美好的、善良的品行。于是我写了淡泊一生的钓鱼的医生,“涸辙之鲋,相濡以沫”的岁寒三友。我写的人物,有一些是可笑的,但是连这些可笑处也是值得同情的,我对他们的嘲笑不能过于尖刻。我的小说大都带有一点抒情色彩,因此,我曾自称是一个通俗抒情诗人,称我的现实主义为抒情现实主义。我的小说有一些优美的东西,可以使人得到安慰,得到温暖,但是我的小说没有什么深刻的东西。

现实主义在历史上是和浪漫主义相对峙而言的。现代的现实主义的对立面是现代主义。在中国,所谓现代主义,没有自己的东西,只是摹仿西方的现代主义。这没有什么不好。

我年轻时受过西方现代主义的影响,也可以说是摹仿。后来不再摹仿了,因为摹仿不了。文化可以互相影响,互相渗透,但是一种文化就是一种文化,没有办法使一种文化和另一种文化完全一样。我在美国几个博物馆看了非洲雕塑,惊奇得不得了。都很怪,可是没有一座不精美。我这才明白为什么有人说法国现代艺术受了非洲艺术很大的影响。我又发现非洲人搞的那些奇怪的雕塑,在他们看来一点也不奇怪。他们以为雕塑本来就应该是这样,只能是这样,他们对世界的认识就是这样。他们并没有先有一个对事物的理智的、现实的认识,然后再去“变形”、扭曲、夸大、压扁、拉长……

他们从对事物的认识到对事物的表现是一次完成的。他们表现的,就是他们所认识的。因此,我觉得法国的一些摹仿非洲的现代派艺术也是“假”的。法国人不是非洲人。我在几个博物馆看了一些西洋名画的原作,也看了芝加哥、波士顿艺术馆一些中国名画,比如相传宋徽宗摹张萱的捣练图。我深深感到东方的——主要是中国的文化和西方文化绝对不是一回事。

中国画和西洋画的审美意识完全不同。中国人插花有许多讲究,瓶与花要配称,横斜欹侧,得花之态。有时只有一截干枝,开一朵铁骨红梅。这种趣味,西方人完全不懂。他们只是用一个玻璃瓶,乱哄哄地插了一大把颜色鲜丽的花。中国画里的折枝花卉,西方是没有的。更不用说墨绘的兰竹。毕加索认为中国的书法是伟大的艺术,但是要叫他分别一下王羲之和王献之,他一定说不出所以然。中国文学要全盘西化,搞出“真”现代派,是不可能的。因为你是中国人,你生活在中国文化的传统里,而这种传统是那样的悠久,那样的无往而不在。你要摆脱它,是办不到的。而且,为什么要摆脱呢?

最最无法摆脱的是语言。一个民族文化的最基本的东西是语言。汉字和汉语不是一回事。中国的识字的人,与其说是用汉语思维,不如说用汉字思维。汉字是象形字。形声字的形还是起很大作用。从木的和从水的字会产生不同的图像。汉字又有平上去入,这是西方文字所没有的。中国作家便是用这种古怪的文字写作的,中国作家对于文字的感觉和西方作家很不相同。中国文字有一些十分独特的东西,比如对仗、声调。对仗,是随时会遇到的。有人说某人用这个字,不用另一个意义相同的字,是“为声俊耳”。声“俊”不“俊”,外国人很难体会,但是作为一个中国作家是不能不注意的。

上来就说:“首先我要问你一个你自己很难回答的问题:你认为你在中国文学里的位置是什么?”我想了一想,说:“我大概是一个文体家。”“文体家”原本不是一个褒词。伟大的作家都不是文体家。这个概念近些年有些变化。现代小说多半很注重文体。过去把文体和内容是分开的,现在很多人认为是一回事。我是较早地意识到二者的一致性的。文体的基础是语言。一个作家应该对语言充满兴趣,对语言很敏感,喜欢听人说话。

苏州有个老道士,在人家做道场,斜眼看见桌子下面有一双钉靴,他不动声色,在诵念的经文中加了几句,念给小道士听:

\centerline{\emph{台子底下,}}
\centerline{\emph{有双钉靴。}}
\centerline{\emph{拿俚转去,}}
\centerline{\emph{落雨着着,}}
\centerline{\emph{也是好格。}}

这种有板有眼,整整齐齐的语言,听起来非常好笑。如果用平常的散文说出来,就毫无意思。我们应该留意:一句话这样说就很有意思,那样说就没有意思。其次要读一点古文。“熟读唐诗三百首”,还是学诗的好办法。我们作文(写小说式散文)的时候,在写法上常常会受古人的某一篇或某几篇的影响,自觉或不自觉。老舍的《火车》写火车着火后的火势,写得那样铺张,没有若干篇古文烂熟胸中,是办不到的。我写了一篇散文《天山行色》,开头第一句:

\centerline{\emph{所谓南山者,是一片塔松林。}}

我自己知道,这样的突兀的句法是从龚定庵的《说居庸关》那里来的。《说居庸关》的第一句是:

\centerline{\emph{居庸关者,古之谈守者之言也。}}

这样的开头,就决定这篇长达一万七千字的散文,处处有点龚定庵\footnote{龚自珍(1792年8月22日—1841年9月26日),号定庵。汉族,浙江仁和(今杭州)人。清代思想家、诗人、文学家和改良主义的先驱者。他主张革除弊政,抵制外国侵略,曾全力支持林则徐禁除鸦片。}的影子,这篇散文可以说是龚定庵体。文体的形成和一个作家的文化修养是有关系的。文学和其他文化现象是相通的。作家应该读一点画,懂得书法。中国的书法是纯粹抽象的艺术,但绝对是艺术。书法有各种书体,有很多家,这些又是非常具体的,可以感觉的。中国古代文人的字大都是写得很好的。李白的字不一定可靠。杜牧的字写得很好。苏轼、秦观、陆游、范成大的字都写得很好。宋人文人里字写得差一点的只有司马光,不过他写的方方正正的楷书也另有一种味道,不俗气。现代作家不一定要能写好毛笔字,但是要能欣赏书法。我虽不善书,“知书莫若我”,经常看看书法,尤其是行草,对于行文的内在气韵,是很有好处的。我是主张“回到民族传统”的,但是并不拒绝外来的影响。我多少读了一点翻译作品,不能不受影响,包括思维、语言、文体。我的这篇发言的题目,是用汉字写的,但实在不大像一句中国话。我找不到更恰当的语言表达我要说的意思。

我是沈从文先生的学生,有人问我究竟从沈先生那里继承了什么。很难说是继承,只能说我愿意向沈先生学习什么。沈先生逝世后,在他的告别读者和亲友的仪式上,有一位新华社记者问我对沈先生的看法。在那种场合下,不遑深思,我只说了两点。一、沈先生是一个真诚的爱国主义者;二、他是我见到的真正淡泊的作家,这种淡泊不仅是一种“人”的品德,而且是一种“人”的境界。沈先生是爱中国的,爱得很深。我也是爱我们这个国的。“儿不嫌母丑,狗不厌家贫。”中国尽管有这样那样的问题,这样那样的缺点,但它是我的国家。正如沈先生所说,在任何情况下,都不应丧失信心。

我没有荒谬感、失落感、孤独感。我并不反对荒谬感、失落感、孤独感,但是我觉得我们这样的社会,不具备产生这样多的感的条件。如果为了赢得读者,故意去表现本来没有,或者有也不多的荒谬感、失落感和孤独感,我以为不仅是不负责任,而且是不道德的。文学,应该使人获得生活的信心。淡泊,是人品,也是文品。一个甘于淡泊的作家,才能不去抢行情,争座位;才能真诚地写出自己所感受到的那点生活,不耍花招,不欺骗读者。至于文学上我从沈先生继承了什么,还是让评论家去论说吧。我自己不好说,也说不好。

\newpage

\chapter{其他}

\section{讲故事的人}

\emph{北京时间2012年12月8日凌晨,2012年诺贝尔文学奖获得者莫言身着胸前刺绣着“莫言”两字红色篆刻图案的深色中山装,面对着200多名中外听众,在瑞典学院发表文学演讲,主题为“讲故事的人”(Storyteller)。}
\vspace{2em}

\leftline{尊敬的瑞典学院各位院士,女士们、先生们:}

通过电视或者网络,我想在座的各位,对遥远的高密东北乡,已经有了或多或少的了解。你们也许看到了我的九十岁的老父亲,看到了我的哥哥姐姐我的妻子女儿和我的一岁零四个月的外孙女。但有一个我此刻最想念的人,我的母亲,你们永远无法看到了。我获奖后,很多人分享了我的光荣,但我的母亲却无法分享了。

我母亲生于1922 年,卒于1994 年。她的骨灰,埋葬在村庄东边的桃园里。去年,一条铁路要从那儿穿过,我们不得不将她的坟墓迁移到距离村子更远的地方。掘开坟墓后,我们看到,棺木已经腐朽,母亲的骨殖,已经与泥土混为一体。我们只好象征性地挖起一些泥土,移到新的墓穴里。也就是从那一时刻起,我感到,我的母亲是大地的一部分,我站在大地上的诉说,就是对母亲的诉说。

我是我母亲最小的孩子。我记忆中最早的一件事,是提着家里唯一的一把热水瓶去公共食堂打开水。因为饥饿无力,失手将热水瓶打碎,我吓得要命,钻进草垛,一天没敢出来。傍晚的时候,我听到母亲呼唤我的乳名。我从草垛里钻出来,以为会受到打骂,但母亲没有打我也没有骂我,只是抚摸着我的头,口中发出长长的叹息。我记忆中最痛苦的一件事,就是跟随着母亲去集体的地里捡麦穗,看守麦田的人来了,捡麦穗的人纷纷逃跑,我母亲是小脚,跑不快,被捉住,那个身材高大的看守人搧了她一个耳光。她摇晃着身体跌倒在地。看守人没收了我们捡到的麦穗,吹着口哨扬长而去。我母亲嘴角流血,坐在地上,脸上那种绝望的神情让我终生难忘。多年之后,当那个看守麦田的人成为一个白发苍苍的老人,在集市上与我相逢,我冲上去想找他报仇,母亲拉住了我,平静地对我说:“儿子,那个打我的人,与这个老人,并不是一个人。”

我记得最深刻的一件事是一个中秋节的中午,我们家难得地包了一顿饺子,每人只有一碗。正当我们吃饺子时,一个乞讨的老人,来到了我们家门口。我端起半碗红薯干打发他,他却愤愤不平地说:“我是一个老人,你们吃饺子,却让我吃红薯干,你们的心是怎么长的?”我气急败坏地说:“我们一年也吃不了几次饺子,一人一小碗,连半饱都吃不了!给你红薯干就不错了,你要就要,不要就滚!”母亲训斥了我,然后端起她那半碗饺子,倒进老人碗里。

我最后悔的一件事,就是跟着母亲去卖白菜,有意无意地多算了一位买白菜的老人一毛钱。算完钱我就去了学校。当我放学回家时,看到很少流泪的母亲泪流满面。母亲并没有骂我,只是轻轻地说:“儿子,你让娘丢了脸。”

我十几岁时,母亲患了严重的肺病,饥饿,病痛,劳累,使我们这个家庭陷入困境,看不到光明和希望。我产生了一种强烈的不祥之感,以为母亲随时都会自寻短见。每当我劳动归来,一进大门,就高喊母亲,听到她的回应,心中才感到一块石头落了地,如果一时听不到她的回应,我就心惊胆颤,跑到厢房和磨坊里寻找。有一次,找遍了所有的房间也没有见到母亲的身影。我便坐在院子里大哭。这时,母亲背着一捆柴草从外边走进来。她对我的哭很不满,但我又不能对她说出我的担忧。母亲看透我的心思,她说:“孩子,你放心,尽管我活着没有一点乐趣,但只要阎王爷不叫我,我是不会去的。”我生来相貌丑陋,村子里很多人当面嘲笑我,学校里有几个性格霸蛮的同学甚至为此打我。我回家痛哭,母亲对我说:“儿子,你不丑。你不缺鼻子不缺眼,四肢健全,丑在哪里?而且,只要你心存善良,多做好事,即便是丑,也能变美。”后来我进入城市,有一些很有文化的人依然在背后甚至当面嘲弄我的相貌,我想起了母亲的话,便心平气和地向他们道歉。

我母亲不识字,但对识字的人十分敬重。我们家生活困难,经常吃了上顿没下顿,但只要我对她提出买书买文具的要求,她总是会满足我。她是个勤劳的人,讨厌懒惰的孩子,但只要是我因为看书耽误了干活,她从来没批评过我。有一段时间,集市上来了一个说书人。我偷偷地跑去听书,忘记了她分配给我的活儿。为此,母亲批评了我。晚上,当她就着一盏小油灯为家人赶制棉衣时,我忍不住地将白天从说书人那里听来的故事复述给她听,起初她有些不耐烦,因为在她心目中,说书人都是油嘴滑舌、不务正业的人,从他们嘴里,冒不出什么好话来。但我复述的故事,渐渐地吸引了她。以后每逢集日,她便不再给我排活儿,默许我去集上听书。为了报答母亲的恩情,也为了向她炫耀我的记忆力,我会把白天听到的故事,绘声绘色地讲给她听。

很快的,我就不满足复述说书人讲的故事了,我在复述的过程中,不断地添油加醋。我会投我母亲所好,编造一些情节,有时候甚至改变故事的结局。我的听众,也不仅仅是我的母亲,连我的姐姐,我的婶婶,我的奶奶,都成为我的听众。我母亲在听完我的故事后,有时会忧心忡忡地,像是对我说,又像是自言自语:“儿啊,你长大后会成为一个什么人呢?难道要靠耍贫嘴吃饭吗?”我理解母亲的担忧,因为在村子里,一个贫嘴的孩子,是招人厌烦的,有时候还会给自己和家庭带来麻烦。我在小说《牛》里所写的那个因为话多被村里人厌恶的孩子,就有我童年时的影子。我母亲经常提醒我少说话,她希望我能做一个沉默寡言、安稳大方的孩子。但在我身上,却显露出极强的说话能力和极大的说话欲望,这无疑是极大的危险,但我的说故事的能力,又带给了她愉悦,这使她陷入深深的矛盾之中。

俗话说“江山易改,本性难移”,尽管有我父母亲的谆谆教导,但我并没改掉我喜欢说话的天性,这使得我的名字“莫言”,很像对自己的讽刺。我小学未毕业即辍学,因为年幼体弱,干不了重活,只好到荒草滩上去放牧牛羊。当我牵着牛羊从学校门前路过,看到昔日的同学在校园里打打闹闹,我心中充满悲凉,深深地体会到一个人——哪怕是一个孩子——离开群体后的痛苦。到了荒滩上,我把牛羊放开,让它们自己吃草。蓝天如海,草地一望无际,周围看不到一个人影,没有人的声音,只有鸟儿在天上鸣叫。我感到很孤独,很寂寞,心里空空荡荡。有时候,我躺在草地上,望着天上懒洋洋地飘动着的白云,脑海里便浮现出许多莫名其妙的幻像。我们那地方流传着许多狐狸变成美女的故事。我幻想着能有一个狐狸变成美女与我来做伴放牛,但她始终没有出现。但有一次,一只火红色的狐狸从我面前的草丛中跳出来时,我被吓得一屁股蹲在地上。狐狸跑没了踪影,我还在那里颤抖。有时候我会蹲在牛的身旁,看着湛蓝的牛眼和牛眼中的我的倒影。有时候我会模仿着鸟儿的叫声试图与天上的鸟儿对话,有时候我会对一棵树诉说心声。但鸟儿不理我,树也不理我。——许多年后,当我成为一个小说家,当年的许多幻想,都被我写进了小说。很多人夸我想象力丰富,有一些文学爱好者,希望我能告诉他们培养想象力的秘诀,对此,我只能报以苦笑。就像中国的先贤老子所说的那样:“福兮祸所伏,祸兮福所倚”,我童年辍学,饱受饥饿、孤独、无书可读之苦,但我因此也像我们的前辈作家沈从文那样,及早地开始阅读社会人生这本大书。前面所提到的到集市上去听说书人说书,仅仅是这本大书中的一页。

辍学之后,我混迹于成人之中,开始了“用耳朵阅读”的漫长生涯。二百多年前,我的故乡曾出了一个讲故事的伟大天才——蒲松龄,我们村里的许多人,包括我,都是他的传人。我在集体劳动的田间地头,在生产队的牛棚马厩,在我爷爷奶奶的热炕头上,甚至在摇摇晃晃地行进着的牛车上,聆听了许许多多神鬼故事,历史传奇,逸闻趣事,这些故事都与当地的自然环境、家族历史紧密联系在一起,使我产生了强烈的现实感。

我做梦也想不到有朝一日这些东西会成为我的写作素材,我当时只是一个迷恋故事的孩子,醉心地聆听着人们的讲述。那时我是一个绝对的有神论者,我相信万物都有灵性,我见到一棵大树会肃然起敬。我看到一只鸟会感到它随时会变化成人,我遇到一个陌生人,也会怀疑他是一个动物变化而成。每当夜晚我从生产队的记工房回家时,无边的恐惧便包围了我,为了壮胆,我一边奔跑一边大声歌唱。那时我正处在变声期,嗓音嘶哑,声调难听,我的歌唱,是对我的乡亲们的一种折磨。

我在故乡生活了二十一年,期间离家最远的是乘火车去了一次青岛,还差点迷失在木材厂的巨大木材之间,以至于我母亲问我去青岛看到了什么风景时,我沮丧地告诉她:什么都没看到,只看到了一堆堆的木头。但也就是这次青岛之行,使我产生了想离开故乡到外边去看世界的强烈愿望。

1976 年2 月,我应征入伍,背着我母亲卖掉结婚时的首饰帮我购买的四本《中国通史简编》,走出了高密东北乡这个既让我爱又让我恨的地方,开始了我人生的重要时期。我必须承认,如果没有多年来中国社会的巨大发展与进步,如果没有改革开放,也不会有我这样一个作家。

在军营的枯燥生活中,我迎来了八十年代的思想解放和文学热潮,我从一个用耳朵聆听故事,用嘴巴讲述故事的孩子,开始尝试用笔来讲述故事。起初的道路并不平坦,我那时并没有意识到我二十多年的农村生活经验是文学的富矿,那时我以为文学就是写好人好事,就是写英雄模范,所以,尽管也发表了几篇作品,但文学价值很低。

1984 年秋,我考入解放军艺术学院文学系。在我的恩师著名作家徐怀中\footnote{徐怀中,原名怀忠,河北邯郸人,作家。1945年参加八路军,次年加入中国共产党,曾任晋冀鲁豫军区政治部文工团团员、第二野战军政治部文工团美术组组长。徐怀中的作品《底色》荣获2014年第六届鲁迅文学奖报告文学奖。2019年8月16日,凭借作品《牵风记》获得第十届茅盾文学奖。}的启发指导下,我写出了《秋水》、《枯河》、《透明的红萝卜》、《红高粱》等一批中短篇小说。在《秋水》这篇小说里,第一次出现了“高密东北乡”这个字眼,从此,就如同一个四处游荡的农民有了一片土地,我这样一个文学的流浪汉,终于有了一个可以安身立命的场所。我必须承认,在创建我的文学领地“高密东北乡”的过程中,美国的威廉·福克纳和哥伦比亚的加西亚·马尔克斯给了我重要启发。我对他们的阅读并不认真,但他们开天辟地的豪迈精神激励了我,使我明白了一个作家必须要有一块属于自己的地方。一个人在日常生活中应该谦卑退让,但在文学创作中,必须颐指气使,独断专行。

我追随在这两位大师身后两年,即意识到,必须尽快地逃离他们,我在一篇文章中写道:他们是两座灼热的火炉,而我是冰块,如果离他们太近,会被他们蒸发掉。根据我的体会,一个作家之所以会受到某一位作家的影响,其根本是因为影响者和被影响者灵魂深处的相似之处。正所谓“心有灵犀一点通”。所以,尽管我没有很好地去读他们的书,但只读过几页,我就明白了他们干了什么,也明白了他们是怎样干的,随即我也就明白了我该干什么和我该怎样干。我该干的事情其实很简单,那就是用自己的方式,讲自己的故事。我的方式,就是我所熟知的集市说书人的方式,就是我的爷爷奶奶、村里的老人们讲故事的方式。坦率地说,讲述的时候,我没有想到谁会是我的听众,也许我的听众就是那些如我母亲一样的人,也许我的听众就是我自己,我自己的故事,起初就是我的亲身经历,譬如《枯河》中那个遭受痛打的孩子,譬如《透明的红萝卜》中那个自始至终一言不发的孩子。我的确曾因为干过一件错事而受到过父亲的痛打,

我也的确曾在桥梁工地上为铁匠师傅拉过风箱。当然,个人的经历无论多么奇特也不可能原封不动地写进小说,小说必须虚构,必须想象。很多朋友说《透明的红萝卜》是我最好的小说,对此我不反驳,也不认同,但我认为《透明的红萝卜》是我的作品中最有象征性、最意味深长的一部。那个浑身漆黑、具有超人的忍受痛苦的能力和超人的感受能力的孩子,是我全部小说的灵魂,尽管在后来的小说里,我写了很多的人物,但没有一个人物,比他更贴近我的灵魂。或者可以说,一个作家所塑造的若干人物中,总有一个领头的,这个沉默的孩子就是一个领头的,他一言不发,但却有力地领导着形形色色的人物,在高密东北乡这个舞台上,尽情地表演。自己的故事总是有限的,讲完了自己的故事,就必须讲他人的故事。于是,我的亲人们的故事,我的村人们的故事,以及我从老人们口中听到过的祖先们的故事,就像听到集合令的士兵一样,从我的记忆深处涌出来。他们用期盼的目光看着我,等待着我去写他们。我的爷爷、奶奶、父亲、母亲、哥哥、姐姐、姑姑、叔叔、妻子、女儿,都在我的作品里出现过,还有很多的我们高密东北乡的乡亲,也都在我的小说里露过面。当然,我对他们,都进行了文学化的处理,使他们超越了他们自身,成为文学中的人物。

我最新的小说《蛙》中,就出现了我姑姑的形象。因为我获得诺贝尔奖,许多记者到她家采访,起初她还很耐心地回答提问,但很快便不胜其烦,跑到县城里她儿子家躲起来了。姑姑确实是我写《蛙》时的模特,但小说中的姑姑,与现实生活中的姑姑有着天壤之别。小说中的姑姑专横跋扈,有时简直像个女匪,现实中的姑姑和善开朗,是一个标准的贤妻良母。现实中的姑姑晚年生活幸福美满,小说中的姑姑到了晚年却因为心灵的巨大痛苦患上了失眠症,身披黑袍,像个幽灵一样在暗夜中游荡。我感谢姑姑的宽容,她没有因为我在小说中把她写成那样而生气;我也十分敬佩我姑姑的明智,她正确地理解了小说中人物与现实中人物的复杂关系。母亲去世后,我悲痛万分,决定写一部书献给她。这就是那本《丰乳肥臀》。因为胸有成竹,因为情感充盈,仅用了83 天,我便写出了这部长达50 万字的小说的初稿。

在《丰乳肥臀》这本书里,我肆无忌惮地使用了与我母亲的亲身经历有关的素材,但书中的母亲情感方面的经历,则是虚构或取材于高密东北乡诸多母亲的经历。在这本书的卷前语上,我写下了“献给母亲在天之灵”的话,但这本书,实际上是献给天下母亲的,这是我狂妄的野心,就像我希望把小小的“高密东北乡”写成中国乃至世界的缩影一样。

作家的创作过程各有特色,我每本书的构思与灵感触发也都不尽相同。有的小说起源于梦境,譬如《透明的红萝卜》,有的小说则发端于现实生活中发生的事件——譬如《天堂蒜薹之歌》。但无论是起源于梦境还是发端于现实,最后都必须和个人的经验相结合,才有可能变成一部具有鲜明个性的,用无数生动细节塑造出了典型人物的、语言丰富多彩、结构匠心独运的文学作品。有必要特别提及的是,在《天堂蒜薹之歌》中,我让一个真正的说书人登场,并在书中扮演了十分重要的角色。我十分抱歉地使用了这个说书人真实姓名,当然,他在书中的所有行为都是虚构。在我的写作中,出现过多次这样的现象,写作之初,我使用他们的真实姓名,希望能借此获得一种亲近感,

但作品完成之后,我想为他们改换姓名时却感到已经不可能了,因此也发生过与我小说中人物同名者找到我父亲发泄不满的事情,我父亲替我向他们道歉,但同时又开导他们不要当真。我父亲说:“他在《红高粱》中,第一句就说‘我父亲这个土匪种’,我都不在意你们还在意什么?”

我在写作《天堂蒜薹之歌》这类逼近社会现实的小说时,面对着的最大问题,其实不是我敢不敢对社会上的黑暗现象进行批评,而是这燃烧的激情和愤怒会让政治压倒文学,使这部小说变成一个社会事件的纪实报告。小说家是社会中人,他自然有自己的立场和观点,但小说家在写作时,必须站在人的立场上,把所有的人都当做人来写。

只有这样,文学才能发端事件但超越事件,关心政治但大于政治。可能是因为我经历过长期的艰难生活,使我对人性有较为深刻的了解。我知道真正的勇敢是什么,也明白真正的悲悯是什么。我知道,每个人心中都有一片难用是非善恶准确定性的朦胧地带,而这片地带,正是文学家施展才华的广阔天地。只要是准确地、生动地描写了这个充满矛盾的朦胧地带的作品,也就必然地超越了政治并具备了优秀文学的品质。

喋喋不休地讲述自己的作品是令人厌烦的,但我的人生是与我的作品紧密相连的,不讲作品,我感到无从下嘴,所以还得请各位原谅。在我的早期作品中,我作为一个现代的说书人,是隐藏在文本背后的,但从《檀香刑》这部小说开始,我终于从后台跳到了前台。如果说我早期的作品是自言自语,目无读者,从这本书开始,我感觉到自己是站在一个广场上,面对着许多听众,绘声绘色地讲述。这是世界小说的传统,更是中国小说的传统。我也曾积极地向西方的现代派小说学习,也曾经玩弄过形形色色的叙事花样,但我最终回归了传统。

当然,这种回归,不是一成不变的回归,《檀香刑》和之后的小说,是继承了中国古典小说传统又借鉴了西方小说技术的混合文本。小说领域的所谓创新,基本上都是这种混合的产物。不仅仅是本国文学传统与外国小说技巧的混合,也是小说与其他的艺术门类的混合,就像《檀香刑》是与民间戏曲的混合,就像我早期的一些小说从美术、音乐、甚至杂技中汲取了营养一样。

最后,请允许我再讲一下我的《生死疲劳》。这个书名来自佛教经典,据我所知,为翻译这个书名,各国的翻译家都很头痛。我对佛教经典并没有深入研究,对佛教的理解自然十分肤浅,之所以以此为题,是因为我觉得佛教的许多基本思想,是真正的宇宙意识,人世中许多纷争,在佛家的眼里,是毫无意义的。这样一种至高眼界下的人世,显得十分可悲。当然,我没有把这本书写成布道词,我写的还是人的命运与人的情感,人的局限与人的宽容,以及人为追求幸福、坚持自己的信念所做出的努力与牺牲。小说中那位以一己之身与时代潮流对抗的蓝脸,在我心目中是一位真正的英雄。这个人物的原型,是我们邻村的一位农民,我童年时,经常看到他推着一辆吱吱作响的木轮车,从我家门前的道路上通过。给他拉车的,是一头瘸腿的毛驴,为他牵驴的,是他小脚的妻子。这个奇怪的劳动组合,在当时的集体化社会里,显得那么古怪和不合时宜,在我们这些孩子的眼里,也把他们看成是逆历史潮流而动的小丑,以至于当他们从街上经过时,我们会充满义愤地朝他们投掷石块。事过多年,当我拿起笔来写作时,这个人物,这个画面,便浮现在我的脑海中。我知道,我总有一天会为他写一本书,我迟早要把他的故事讲给天下人听,但一直到了2005年,当我在一座庙宇里看到“六道轮回”的壁画时,才明白了讲述这个故事的正确方法。

我获得诺贝尔文学奖后,引发了一些争议。起初,我还以为大家争议的对象是我,渐渐的,我感到这个被争议的对象,是一个与我毫不相关的人。我如同一个看戏人,看着众人的表演。我看到那个得奖人身上落满了花朵,也被掷上了石块、泼上了污水。我生怕他被打垮,但他微笑着从花朵和石块中钻出来,擦干净身上的脏水,坦然地站在一边,对着众人说:对一个作家来说,最好的说话方式是写作。我该说的话都写进了我的作品里。用嘴说出的话随风而散,用笔写出的话永不磨灭。我希望你们能耐心地读一下我的书,当然,我没有资格强迫你们读我的书。

即便你们读了我的书,我也不期望你们能改变对我的看法,世界上还没有一个作家,能让所有的读者都喜欢他。在当今这样的时代里,更是如此。

尽管我什么都不想说,但在今天这样的场合我必须说话,那我就简单地再说几句。

我是一个讲故事的人,我还是要给你们讲故事。上世纪六十年代,我上小学三年级的时候,学校里组织我们去观一个苦难展览,我们在老师的引领下放声大哭。为了能让老师看到我的表现,我舍不得擦去脸上的泪水。我看到有几位同学悄悄地将唾沫抹到脸上冒充泪水。我还看到在一片真哭假哭的同学之间,有一位同学,脸上没有一滴泪,嘴巴里没有一点声音,也没有用手掩面。他睁着大眼看着我们,眼睛里流露出惊讶或者是困惑的神情。事后,我向老师报告了这位同学的行为。为此,学校给了这位同学一个警告处分。多年之后,当我因自己的告密向老师忏悔时,老师说,那天来找他说这件事的,有十几个同学。这位同学十几年前就已去世,每当想起他,我就深感歉疚。这件事让我悟到一个道理,那就是:当众人都哭时,应该允许有的人不哭。当哭成为一种表演时,更应该允许有的人不哭。

我再讲一个故事:三十多年前,我还在部队工作。有一天晚上,我在办公室看书,有一位老长官推门进来,看了一眼我对面的位置,自言自语道:“噢,没有人?”我随即站起来,高声说:“难道我不是人吗?”那位老长官被我顶得面红耳赤,尴尬而退。为此事,我洋洋得意了许久,以为自己是个英勇的斗士,但事过多年后,我却为此深感内疚。请允许我讲最后一个故事,这是许多年前我爷爷讲给我听过的:有八个外出打工的泥瓦匠,为避一场暴风雨,躲进了一座破庙。外边的雷声一阵紧似一阵,一个个的火球,在庙门外滚来滚去,空中似乎还有吱吱的龙叫声。众人都胆战心惊,面如土色。有一个人说:“我们八个人中,必定一个人干过伤天害理的坏事,谁干过坏事,就自己走出庙接受惩罚吧,免得让好人受到牵连。”自然没有人愿意出去。又有人提议道:“既然大家都不想出去,那我们就将自己的草帽往外抛吧,谁的草帽被刮出庙门,就说明谁干了坏事,那就请他出去接受惩罚。”于是大家就将自己的草帽往庙门外抛,七个人的草帽被刮回了庙内,只有一个人的草帽被卷了出去。大家就催这个人出去受罚,他自然不愿出去,众人便将他抬起来扔出了庙门。故事的结局我估计大家都猜到了——那个人刚被扔出庙门,那座破庙轰然坍塌。

我是一个讲故事的人。因为讲故事我获得了诺贝尔文学奖。我获奖后发生了很多精彩的故事,这些故事,让我坚信真理和正义是存在的。

今后的岁月里,我将继续讲我的故事。

谢谢大家!

\newpage

\section{后记}

编辑《写作只能塑造真实的自己》这本小册子是纯属偶然兴起,看到网友“平安哥”在腾讯文档整理了各位作家关于写作的心得体会,质量非常高。故在此基础上新增了内容,也添加了脚注和排版,使其更加正式,也便于打印。

我也是一名90后,当意识到大数据每天都在给我推送公文写作的时候,我感觉到了文字即将消散。当我坚持每周写作,坚持了4年,我好像什么也没收获,但好像也收获了笔耕不辍。我明白我并非像机器一般只能产出公文,一事无成,而是要追求知识与德行。

我有一个自己的博客站点\href{https://www.macin.org/}{ www.macin.org},如果有一天腾讯公众号死掉了,我的文字还不至于流散。当然也欢迎各位添加微信,多多交流。微信二维码如下:

\begin{figure}[htbp] %H为当前位置,!htb为忽略美学标准,htbp为浮动图形
\centering %图片居中
\includegraphics[width=1\textwidth]{wechat.jpg} %插入图片,[]中设置图片大小,{}中是图片文件名
%\caption{Main name 2} %最终文档中希望显示的图片标题
%\label{Fig.main2} %用于文内引用的标签
\end{figure}


\end{document}